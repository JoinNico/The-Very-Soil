\chapter{不曾忘却(魔法少女和美)}
《魔法少女小圆》电视系列剧中确立了三条主要原则,而这三条原则在不同程度上对本章节讨论的三部衍生作品构成了挑战。首先,魔法少女从起源起就注定会成为魔女。鉴于魔女这一形象最初象征着父权社会对女性力量的恐惧(103),在《魔法少女小圆》中,魔法少女既内在地对现行权力结构构成挑战,同时也容易被解读为一种堕落且具有腐蚀性的元素。其次,《魔法少女小圆》的世界由丘比的系统主导,这一系统通过将宇宙的熵转化为魔法少女的情感崩溃来清除熵,而这一转化过程并未征得她们的同意——这意味着这个权力结构不仅剥削女性,而且对她们极不公正。第三,这一权力结构高度稳固,直到小圆许愿并创造了新的系统,才得以推翻。

暂且搁置小圆的系统是否比丘比的系统更公正的问题(详见第二次插曲和第十七章),第三条原则意味着任何设定在剧集主线时间线或先前丘比系统主导的时间线中的衍生作品,最终都必须将丘比的系统视为不可避免且无法改变的事实。再加上丘比的系统如何限制和剥削魔法少女力量(第二条原则),以及这一力量如何与描绘女性力量为堕落且危险的传统挂钩(第一条原则),衍生作品很容易落入陷阱——描绘魔法少女为危险的存在,并将丘比的系统视为约束她们的必要之恶,甚至更糟糕的是,将其视为一种净收益。

这正是《魔法少女和美魔法记录》漫画系列(由平松正树编剧、天杉贵志绘制)所陷入的陷阱。

《和美魔法记录》开篇时,主人公和美几乎是一张白纸——赤裸裸地被封在一个箱子里,完全失去了过去的记忆。这是一种极端形式的“白房间综合症”,这是恶名昭著的《火鸡城语录》(104)所批评的开场手法,指的是故事开头角色处在一个白房间里,思考自己是如何来到这里的;《语录》对这种手法不甚友好,但或许确实准确,认为这是因为作者尚未弄清故事的背景和人物设定。这个批评是否适用于《和美魔法记录》尚不清楚。一方面,和美失去记忆在剧情上有其重要作用,因为这让她能够作为观众的代入点,逐步发现她的身份和“昴星团”魔法少女小队的真相。然而,《和美魔法记录》是一个相对较长的连载作品,共23个月刊章节,且在《魔法少女小圆》系列完结之前就已经开始连载。漫画中的许多后期揭示都逻辑上依赖于《魔法少女小圆》中的晚期情节,而目前尚不清楚这些揭示是为了避免衍生作品剧透母剧而有所保留,还是因为平松正树和天杉贵志当时并不知道这些剧透,因而不得不在连载过程中调整漫画方向以避免与母剧相悖。

接下来,漫画揭示了和美及其朋友们是魔法少女,隶属于一个由七人组成的团队,名为“昴星团”,并得到一个看似比丘比更友好的变体角色——十兵卫的支持。随着漫画展开,读者逐渐发现这些“昴星团”比最初看上去要黑暗得多。最终揭示,和美及其朋友们原本是一个合作的魔法少女小队,和美曾作为她们的导师,但她最终变成了魔女。她们合力将和美变回了类人形态,同时杀死并将她们当地的丘比转变为十兵卫,并施展了一个覆盖整座城市的魔法,使替代丘比变得不可察觉。“昴星团”误以为十兵卫能够净化她们的灵魂宝石,但实际上他只是制造了净化的假象;在这假象之下,她们仍在慢慢堕入魔女的深渊。和美则是一次次复活原型的最新尝试,但每次尝试最终都会崩溃,和美变成魔女。最终,除两人外的所有“昴星团”成员都被杀死,并形成了一个庞大的魔女合体,而和美则意识到她与原型是不同的人,于是与丘比签订契约,成为魔法少女,拯救了城市和剩下的“昴星团”成员。

这一相对复杂的故事(因采用和美发现真相的顺序而非事件发生的顺序讲述而显得更加复杂)将魔法少女与魔女的统一关系具象化,正如我在第九章中所讨论的那样。然而,漫画并未试图为魔女的形象赋予救赎,而是将魔法少女描绘为某种魔女。视觉上,这体现在和美“普通”状态下的着装与魔法少女形态之间的对比。第二章中,她的普通装扮被描述为适合更年幼的儿童,而她的魔法少女装则显得比电视系列中的服装暴露得多,酷似性感化的万圣节魔女装扮——宽边、高顶的黑帽子、短小的黑裙、裸露的腰部、紧身的上衣。相比之下,小圆的粉色蓬裙、沙耶加的中性盔甲,甚至剧集中最性感化的服装,如麻美的高领束腰裙和紧身裤,或杏子的半解拉链上衣和裙裤,都能显示出漫画在对比纯真儿童与强大、危险、性感的魔法少女时所做的区分。

这正是漫画与系列剧的最大区别:在《魔法少女小圆》中,魔法少女与魔女的统一关系是关于希望与绝望循环的论述,而在《和美魔法记录》中,这一统一关系则反映了“圣母与荡妇情结”。在这个系列中,魔法少女无一例外地隐藏着可怕的秘密——或是她们是魔女(和美),或是她们在过去犯下了可怕的罪行(昴星团成员)。每一位昴星团成员都被她们成为魔法少女之前的某些过失所困扰,欺骗和美,并在她们变成魔女之前帮助捕获并囚禁其他魔法少女。这种转变并非像《魔法少女小圆》中沙耶加所经历的逐渐陷入绝望的过程,而是单纯地因为滥用了魔力;换句话说,魔女并非绝望的产物,而是女性力量的体现。更明显的是,早期的反派是可以直接将女性转化为伪魔女的魔法少女。

当我们最终看到真正的魔女时,她们也与《魔法少女小圆》中的魔女截然不同。在《魔法少女小圆》中,魔女被暗示为无意识的、陷入妄想的、被困在结界中的存在;而在《和美魔法记录》中,她们可以直接与现实世界互动,有些甚至拥有个性和智慧。因此,这些魔女不再是《浮士德》的隐喻,也不再是女性绝望的化身,而是魔法少女身体的丑陋变形。在《和美魔法记录》中,魔法少女与魔女之间唯一的区别在于,魔法少女是“可爱”或“性感”的,而魔女则是畸形或恐怖的。换句话说,魔女的丑陋并非源自她与负面情绪的关联,而是她不再对男性目光具有吸引力。而漫画相比电视系列更频繁地利用了“男性凝视”。

结局同样否认了《魔法少女小圆》中魔女的本质。当和美与她剩下的两位朋友再次在丘比系统内行动时,和美的旁白毫不隐晦地宣称她们不会变成魔女。《魔法少女小圆》中的希望-绝望循环完全被否定了;取而代之的是对女性力量的描绘——只要这种力量被限制在对抗负面女性力量的框架内,就可以继续存在。漫画断言,只要魔法少女们只专注于猎杀魔女而不

试图挑战丘比或其腐朽的系统,她们就可以继续生存并使用她们的力量。

最终,《和美魔法记录》恰巧落入了这样一个陷阱:不经意地(或许无意地)声称丘比的系统是必要的,能够防止女性力量挑战现有秩序。