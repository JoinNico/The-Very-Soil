\interlude{完美一刻}

讨论了这么久《魔法少女小圆》,我们不得不提到 Goethe 的《浮士德》。在动画第二集中,墙上出现的德语涂鸦直接引用了《浮士德》的文本,而 Goethe 对这个古老传说的改编故事,即一个人与魔鬼交易的故事,深刻影响着整部动画。

简单概括一下《浮士德》:浮士德\footnote{译注:关于《浮士德》中出场的人物译名均参考人民文学出版社名著名译丛书《浮士德》,绿原先生的译本。}是一位年老的智者,但他在生活中找不到快乐,于是与魔鬼梅菲斯特签订契约,让他恢复年轻并重新体验生活的种种乐趣。梅菲斯特答应带他体验生活中所有的快乐,但条件是:如果浮士德有一天感到极度的幸福,以至于希望时间永远停止,那他便会立即死亡并堕入地狱。第一部分(1808年出版,1828年修订)主要讲述浮士德追求一位名叫玛加蕾特(有时也简称为格蕾琴)的年轻女子。在他杀死了她的兄弟后,他暂时离开去参加瓦尔普吉斯之夜,这个夜晚在德国民间传说中是魔女和恶魔在布罗肯山上狂欢的时刻。等他回来时,发现格蕾琴已经精神错乱,被关进了监狱。她生下了他的孩子,但孩子被夺走了。他试图解救她,但她因精神错乱无法理解发生了什么,最终他被迫在逃避追捕时抛弃了她。(94)

第二部分(1832年出版,是 Goethe 去世之年)则更加奇特:浮士德再次变老,成为了一个成功富有的人,且是一位强大的巫师。他经历了时间旅行、与特洛伊的海伦有过情史,还通过召唤恶魔大军赢得了一场战争。最终,他做了一件完全为他人着想的善事,体验到了完美的幸福瞬间。他死了,但因为这是为他人所做的善事,他没有立即堕入地狱,而是接受了审判。格蕾琴恳求圣母玛利亚允许她引导浮士德进入天堂,圣母玛利亚同意了。(95)

在《魔圆》中,关于《浮士德》的引用随处可见。除了上述的涂鸦,《浮士德》中的名言也经常以魔女文的形式出现在魔女结界中。更重要的是,这部动画本身包含了许多浮士德式的元素。比如瓦尔普吉斯之夜,虽然在动画中是一位极其强大的魔女,但实际上它就像在《浮士德》中描述的一样,是许多魔女一起展开的邪恶狂欢。魔女结界是通过她们的绝望和疯狂来覆盖现实所形成的牢笼,这与《浮士德》第一部分末尾格蕾琴的经历非常相似。浮士德的一瞬极乐直接通向了地狱,这在动画中的多个角色身上发生:麻美从发现自己有朋友和战友的幸福中瞬间跌入残酷的死亡;杏子父亲因为发现自己的教众是通过杏子许愿而来,陷入了杀人自杀的悲剧;而沙耶香在救了恭介后体验了短暂幸福,然而这却成为她堕入绝望和变成魔女的开端。

更重要的是,《魔圆》的故事可以说是《浮士德》的重绘。丘比显然是梅菲斯特的化身:它最初以一只可爱的动物形象出现,但很快被揭示为一位可怕且强大的捕食者,以愿望交换灵魂。正如梅菲斯特希望浮士德体验到幸福后堕入地狱,丘比则利用魔法少女的情感起伏,期望她们堕入绝望后转变为魔女时释放出的能量。

由于丘比的主要目标是小圆,可能有人会认为她是浮士德的化身,但并非如此。动画更倾向于将她映射为格蕾琴;例如,她的魔女形态被命名为救济的魔女\footnote{译注:救济的魔女英文为 Kriemhild Gretchen,这与格蕾琴(Gretchen)的名字一致。}。丘比在动画的大部分时间里都在尝试让小圆签订契约,但屡屡失败,直到结尾才成功,这与梅菲斯特最初试图腐化格蕾琴时屡屡受挫,直到她最终为浮士德所沦陷的过程相似。最终,小圆的愿望是引导魔法少女,逆转她们成为魔女的命运,这与格蕾琴引导浮士德进入天堂的愿望相呼应。小圆还承担起了救世主和守护者的角色,类似于《浮士德》中与玛利亚联系在一起的永恒女性神圣原则,而格蕾琴也与这一原则联系在一起。(96)

如果小圆不是浮士德,那么谁是?焰是个相当贴切的对象。和浮士德一样,她与恶魔交易来倒转时间,试图纠正她认为自己犯下的错误。(在焰的情况下更为字面化,但浮士德也进行了时间旅行。)她与小圆的亲密关系和想要拯救小圆的强烈愿望反应了浮士德对格蕾琴的感情,而焰停止时间的能力也可能是在暗示浮士德契约中的时间停止条件。最后,像浮士德一样,她最终也意识到自己试图回溯时间的行为反而让事情变得更加糟糕。

然而,《魔圆》也颠覆了《浮士德》的故事。最终,焰的愿望并非一个错误,而是打破循环的关键,小圆(格蕾琴) 不是向圣母玛利亚或上帝,而是向丘比(梅菲斯特) 祈求力量,引导其他人进入天堂。这是因为小圆并不是《浮士德》中的任何角色,也不是一个基督教人物。她真正的角色来自另一个完全不同的神话体系。

正如《魔圆》在表面上呈现出典型的魔法少女元素,但后来揭露其真正的形式。它也提供了一个浮士德式、基督教化的表面解读,掩盖了其背后本质上是一个佛教故事的内核。最明显的是,佛教的核心教义之一是有漏皆苦\footnote{译注:“漏”就是烦恼。烦恼的种类极多。贪(贪欲)、嗔(嗔恨)、痴(不知无常无我之理等第)是三毒,再加上慢(傲慢)、疑(犹疑)、恶见(不正确的见解如常见、断见等),组成六根本烦恼。由于烦恼造种种业。}(97),而这在《魔圆》中也非常明显。所有的愿望最终都会带来痛苦和绝望;情感的高潮总是伴随着情感的低谷。

另一个关键的佛教概念是业,动画中也提到了这一概念。业是一个非常复杂的概念,不同的佛教宗派对此有不同的解释。广义上,佛教中的业可以被非常简略地概括为因果关系:行动种下种子,这些种子(无论是在这一生还是下一生)会产生结果。善行带来好的结果,恶行带来坏的结果,但无论哪种结果,它们都会将人束缚在业力的循环中,因为这些结果会引发进一步的行动,进而带来更多的结果(98)。魔法少女的幸福—绝望循环以类似的方式运作,逐步将魔法少女拖入成为魔女的深渊。

佛教信仰中,业力的重担还会将人束缚在轮回的循环中,让他们一生又一生地重复,承受前世所积累的业力(99),就像焰被困在第十集所描绘的时间循环中一样。开悟,即对世界的本质有了真正的理解,是逃离这一业力循环的唯一途径,也只有在最后一个时间循环中,圆才了解到焰的时间旅行(轮回的循环)以及丘比的真正意图(业的本质)。最后,瓦尔普吉斯之夜的形象与莲花(佛教中象征开悟的符号(100))极其相似,同时齿轮的意象则反映了不断转动的法轮(101)。

正如前面提到的,小圆类似于佛教神话中的一位人物,观音菩萨。观音是一位年轻的女子,她几乎达到了涅槃,但在即将达成时停了下来。她超越了时间和空间,伸出援手,帮助他人开悟,然后才最终自己进入了涅槃。这也帮助解释了她与圣母玛利亚之间的关联,观音菩萨和玛利亚之间的相似性是显而易见的,并且经常被讨论(102)。

因此,作为观音菩萨,经过拯救时空中所有魔法少女的小圆跨越了槛,达到了下一个层次。她成为了一种自然的力量,一种希望的化身,消散了自我意识,达到了涅槃。

不过,正如我们将看到的,她在通往涅槃的路上还有一个小小的转折。