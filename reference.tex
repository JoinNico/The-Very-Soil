\begin{thebibliography}{200}  
    \bibitem{ref1}Aristotle, Poetics (trans. S. H. Butcher) (The Internet Classics Archive, 2009).  \url{http://classics.mit.edu/Aristotle/poetics.html}.
    \bibitem{ref2}Vivien Burr, "Where do you get your personality from?" An Introduction to Social Constructionism (London: Routledge, 1995).
    \bibitem{ref3}Ross Murfin and Supryia M. Ray, "Modernism," The Bedford Glossary of Critical and Literary Terms (Boston: Bedford Books, 1997).
    \bibitem{ref4}Murfin and Ray, "Postmodernism," The Bedford Glossary of Critical and Literary Terms.
    \bibitem{ref5}Ihab Hassan, "Pluralism in Postmodernism," Exploring Postmodernism: Selected Papers Presented at a Workshop on Postmodernism at the XIth International Comparative Literature Congress, Paris, 20–24 August 1985, Mattei Calinesku and Douew W. Fokkema ed. (Amsterdam: John Benjamins Publishing Company, 1987). 
    \bibitem{ref6}Pauline Marie Rosenau, "Affirmatives and Skeptics," The Fontana Postmodernism Reader, Walter Truett Anderson (ed.) (Fontana Press, 1996).
    \bibitem{ref7}Bill Shipley, "Vaclav Havel: Ambassador of Conscience 2003: From Prisoner to President—A Tribute," Art for Amnesty (Amnesty International, 2003). \url{http://archive.today/B9l16}
    \bibitem{ref8}\url{http://jedablue.com/2014/06/25/ the-nutcracker-the-mouse-king-and-the-puella-magi/}
    \bibitem{ref9}Thomas Adajian, "The Definition of Art," Stanford Encyclopedia of Philosophy, Edward N. Zalta (ed.) \url{http://plato.stanford.edu/entries/art-definition/}
    \bibitem{ref10}Eric P. Nash, "Raising the Curtain," Manga Kamishibai (New York: Abrams Comicarts, 2009).
    \bibitem{ref11}"Sakura and the Mysterious Magic Book," Cardcaptor Sakura: The Complete Collection (NIS America, 2014).
    \bibitem{ref12}"The Crybaby: Usagi's Beautiful Transformation," Pretty Soldier Sailor Moon (Vis Media, 2014).
    \bibitem{ref13}"The Black Claw Grips the Heart," Cutie Honey: The Complete Series (Discotek Media, 2013).
    \bibitem{ref14}"Take Care of Yourself," Neon Genesis Evangelion: Platinum Complete (ADV Films, 2005).
    \bibitem{ref15}Nagaru Tanigawa, The Melancholy of Haruhi Suzumiya (New York: Little, Brown and Company, 2009).
    \bibitem{ref16}Hassan, "Pluralism in Postmodernism."
    \bibitem{ref17}Brian Boyd, "Laughter and Literature: A Play Theory of Humor," Philosophy and Literature, Volume 28, Number 1 (April 2004).
    \bibitem{ref18}Noël Carroll, "Horror and Humor," The Journal of Aesthetics and Art Criticism, Volume 57, Number 2 (Spring 1999).
    
    \bibitem{ref19}Tomaki Saito, Beautiful Fighting Girl, J. Keith Vincent and Dawn Lawson (trans.) (Minneapolis, Minn.: University of Minneapolis Press, 2011)
    \bibitem{ref20}Sharalyn Orbaugh, "Busty Battlin' Babes: The Evolution of the Shojo in 1990s Visual Culture," Gender and Power in the Japanese Visual Field, Joshua S. Mostow, Norman Bryson, and Maribeth Graybill (ed.) (Honolulu: University of Hawai'i Press, 2003).
    \bibitem{ref21}Laura Mulvey, "Visual Pleasure and Narrative Cinema," Media and Cultural Studies: Key Works, Meenakshi Gigi Durham and Douglas M. Kellner (ed.) (Malden, MA: Blackwell, 2006).
    \bibitem{ref22}R. W. Connell, Gender and Power: Society, the Person and Sexual Politics (Stanford, CA: Stanford University Press, 1987).
    \bibitem{ref23}Carl Kimlinger, "Magical Girl Lyrical Nanoha DVD Box Set: Review," Anime News Network (December 15, 2008). \url{http://www.animenewsnetwork.com/review/magical-girl-lyrical-nanoha/dvd-box-set}
    \bibitem{ref24}"Sakura's Wonderful Friend," Cardcaptor Sakura.

    \bibitem{ref25}Ninian Smart, Buddhism and Christianity: Rivals and Allies (Honolulu: University of Hawaii Press, 1993).
    \bibitem{ref26}Magica Quartet, Puella Magi Madoka Magica Production Note (SHAFT, 2011). The book is available only in Japanese; however, a rough translation of the relevant page is available at the unofficial Puella Magi Wiki: \url{http://wiki.puella-magi.net/Puella_Magi_Production_Note#Witch_of_cheesecake_.28early_concept_that_later_became_Charlotte.29}.
    \bibitem{ref27}Romit Dasgupta, "Working with homosociality," The Salaryman in Japan: Crafting Masculinities (Abingdon, Oxon: Routledge, 2013).
    \bibitem{ref28}Paracelsus, On the Nature of Things, excerpted in Stanton J. Linden (ed.), The Alchemy Reader: From Hermes Trismegistus to Isaac Newton (Cambridge: Cambridge University Press, 2003).

    \bibitem{ref29}Gen Urobuchi, Fate/zero Volume 1 (Type-Moon, 2006). The book is available only in Japanese; an unauthorized translation of the relevant section is available at the pirate translation site Baka-Tsuki: \url{http://www.baka-tsuki.org/project/index.php?title=Fate/Zero:Volume_1_Postface_1}.
    \bibitem{ref30}Wayne C. Booth, The Rhetoric of Fiction (Chicago: University of Chicago Press, 1961).
    \bibitem{ref31}"The Body," Buffy the Vampire Slayer: The Complete Series (Fox Searchlight, 2010).
    \bibitem{ref32}Karen A. Smyers, "Symbolizing Inari: The Jewel," The Fox and the Jewel (Honolulu: University of Hawai'i Press, 1999)
    \bibitem{ref33}Princess Tutu: Complete Collection (Æsir Holdings, 2011).

    \bibitem{ref34}Sir James George Frazer, "The External Soul in Folk-Tales," The Golden Bough: A Study in Magic and Religion (Franklin Center, PA: Franklin Library, 1982)
    \bibitem{ref35}Jennifer Clarke Wilkes and Jon Pickens (ed.), "Lich," Dungeons \& Dragons: Monster Manual: Core Rulebook III v.3.5 (Renton, WA: Wizards of the Coast, 2003).
    \bibitem{ref36}J. K. Rowling, Harry Potter and the Half-Blood Prince (New York: Arthur A. Levine, 2005).
    \bibitem{ref37}Susan Caringella, "Affirmative Consent Reform Models," Addressing Rape Reform in Law and Practice (New York: Columbia University Press, 2009).
    \bibitem{ref38}Walter Sinnot-Armstrong, "Consequentialism," Stanford Encyclopedia of Philosophy \url{http://plato.stanford.edu/entries/consequentialism/}

    \bibitem{ref39}Robert Kirk, "Zombies," The Stanford Encyclopedia of Philosophy \url{http://plato.stanford.edu/entries/zombies/}. See also Dennet's counter-arguments to the zombie concept in Daniel Dennett, Consciousness Explained (Boston: Little, Brown, 1991).
    \bibitem{ref40}Tanya Krzywinska, "Zombies in Gamespace: Form, Context, and Meaning in Zombie-Based Video Games," Zombie Culture: Autopsies of the Living Dead, Shawn McIntosh and Marc Leverette (ed.) (Lanham, MD: The Scarecrow Press, 2008).
    \bibitem{ref41}Frantz Fanon, The Wretched of the Earth, Richard Philcox (trans.) (New York: Grove Press, 2004).
    \bibitem{ref42}Senko K. Maynard, "Japanese Communication in Global Context," Japanese Communication: Language and Thought in Context (Honolulu: University of Hawai'i Press, 1997).

    \bibitem{ref43}Maynard, "Japanese Communication Strategies: Collaboration toward Persuasion," Japanese Communication.
    \bibitem{ref44}Compare American Psychiatric Association, Diagnostic and Statistical Manual of Mental Disorders, Fourth Edition, Text Revision: DSM-IV-TR (Washington, DC: American Psychiatric Publishing, 2000).
    \bibitem{ref45}Hans Christian Anderson, "The Little Mermaid," The Classic Fairy Tales, Maria Tatar (ed.) (New York: W.W. Norton and Company, 1999). Excerpted from Anderson, Eighty Fairy Tales, R. P. Keigwin (trans.) (Pantheon Books, 1976).
    \bibitem{ref46}Ana Mardoll, "The Little Mermaid," Ramblings. \url{http://www.anamardoll.com/2012/05/disney-little-mermaid.html}
    \bibitem{ref47}American Psychiatric Association, DSM-IV-TR.
    
    \bibitem{ref48}Sharalyn Orbaugh, "Busty Battlin' Babes."
    \bibitem{ref49}Bewitched: The Complete Series (Sony Pictures Home Entertainment, 2013)
    \bibitem{ref50}Sally the Witch (Toei Animation, 1966)
    \bibitem{ref51}Sandra M. Gilbert and Susan Gubar, "[Snow White and Her Wicked Stepmother]," The Classic Fairy Tales. Excerpted from Gilbert and Gubar, The Madwoman in the Attic: The Woman Writer and the Nineteenth-Century Literary Imagination (New Haven: Yale University Press, 1979).
    \bibitem{ref52}Kumiko Fujimura-Fanselow, "The Japanese Ideology of 'Good Wives and Wise Mothers,': Trends in Contemporary Research," Gender and History, Volume 3, Issue 3 (1991).
    \bibitem{ref53}Smyers, "Symbolizing Inari: The Jewel."
    \bibitem{ref54}Noriko T. Reider, "Yamauba, the Mountain Ogress: Old Hag to Voluptuous Mother," Japanese Demon Lore: Oni from Ancient Times to the Present (Logan, Utah: Utah State University Press, 2010).
    \bibitem{ref55}Lafcadio Hearns, "Yuki-Onna," Kwaidan: Stories and Studies of Strange Things (Project Gutenberg, 1998). \url{http://www.gutenberg.org/files/1210/1210-h/1210-h.htm#yukionna}
    \bibitem{ref56}Gilbert and Gubar, "Snow White and Her Wicked Stepmother."
    \bibitem{ref57}Revolutionary Girl Utena: The Student Council Saga (Right Stuf, 2011).
    \bibitem{ref58}Revolutionary Girl Utena: The Apocalypse Saga (Right Stuf, 2011).
    \bibitem{ref59}Janice Delaney, Mary Jane Lupton, and Emily Toth, The Curse: A Cultural History of Menstruation (Urbana, Ill.: University of Illinois Press, 1988 (Revised)).
    \bibitem{ref60}Patrick W. Galbraith, "Moe: Exploring Virtual Potential in Post-Millennial Japan," Electronic Journal of Contemporary Japanese Studies, Volume 9, Issue 3 (October-November 2009). Note in particular the section titled "Otaku discussions of moe."
    \bibitem{ref61}Neera K. Badhwar and Roderick T. Long, "Ayn Rand," Stanford Encyclopedia of Philosophy \url{http://plato.stanford.edu/entries/ayn-rand/}


    \bibitem{ref62}Smart, "Buddhism," The World's Religions (Cambridge, UK: Cambridge University Press, 1998 (2nd ed.)).
    \bibitem{ref63}Smart, "The Formation of Zen," The World's Religions.
    \bibitem{ref64}Vaclav Havel, "The Politics of Hope," Disturbing the Peace: A Conversation with Karel Hvizdala, Paul Wilson (trans.) (New York: Alfred A. Knopf, 1990).


    \bibitem{ref65}Smart, "The Ingredients of Indian Religion," The World's Religions.
    \bibitem{ref66}Adrian Stokes, Michelangelo: A Study in the Nature of Art (London: Tavistock Publications, 1955).
    \bibitem{ref67}Frank Lynn Meshberger, "An Interpretation of Michelangelo's Creation of Adam Based on Neuroanatomy," JAMA: Journal of the American Medical Association, Volume 264, Issue 14 (October 10, 1990).
    \bibitem{ref68}World Health Organization, International Statistical Classification of Diseases and Related Health Problems, 10th edition (World Health Organization, 1994).
    \bibitem{ref69}John Michael Cooper, "The Cultural and Religious Prehistories," Mendelssohn, Goethe, and the Walpurgis Night: The Heathen Muse in European Culture, 1700-1850 (Rochester, NY: University of Rochester Press, 2007).
    \bibitem{ref70}Giacomo Oreglia, The Commedia Dell'Arte, trans. Lovett F. Edwards (New York: Octagon Books, 1982).
    

    \bibitem{ref71}Burton Watson, The Zen Teachings of Master Lin-Chi: A translation of the Lin-chi lu (New York: Columbia University Press, 1999)
    \bibitem{ref72}Johann Wolfgang von Goethe, Faust, trans. Bayard Taylor (Hazleton, Penn.: Electronic Classic Series, 2005). \url{http://www2.hn.psu.edu/faculty/jmanis/goethe/goethe-faust.pdf}. This PDF, produced by Pennsylvania State University, covers only the first part of Faust.
    \bibitem{ref73}Smart, "The Pure Land Movement and the Mappo," The World's Religions.
    %Part two

    
\end{thebibliography}

\newpage
\paragraph{关于作者}~{}

Jed A. Blue 是第三代极客和终身动画迷,乔治·梅森大学的英语学位,目前居住在华盛顿特区,担任技术写作工作。他常在东海岸的粉丝大会上主持有关动画、文学与电影评论等相关话题的讲座。他最喜欢的魔法少女是焰,最喜欢的船长是 Benjamin Sisko\footnote{译注:本杰明·西斯科,是科幻剧《星际迷航:深空九号》中的一名主要角色。},最喜欢的博士是 Sylvester McCoy\footnote{译注:西尔维斯特·麦考伊(1943年8月20日—),苏格兰演员,知名于从1987年到1989年在英国长寿科幻剧《神秘博士》中演出的第七任博士——经典系列的最后一位博士。}。

他的博客在 JedABlue.com 上,目前正在进行一项为期多年的关于 DC 动画宇宙\footnote{译注:粉丝之间的称呼术语,指由华纳兄弟动画推出的一系列受欢迎的电视系列动画片及相关副产品。}的批评研究。
