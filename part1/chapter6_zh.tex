\chapter[即便是纯粹的道德行为(这种事绝对很奇怪啊)]{即便是纯粹的道德行为\protect\footnotemark(这种事绝对很奇怪啊)}

\footnotetext{译注:``Even a purely moral act that has no hope of any immediate and visible political effect can gradually and indirectly, over time, gain in political significance.''
\par
“即使一个纯粹的道德行动,眼下并不显现任何直接或明显的政治效果,但在长期的历史进程中,它仍可能酝酿出深远的政治意义。”
\par
在本章中,作者对《魔圆》剧情中有关“同意”和“自主权”的讨论,阐释了道德行为如何在看似无关痛痒的情况下,产生巨大的影响力。丘比在隐藏了灵魂宝石的真相的情况下与魔法少女们的契约,使得她们在不完全了解自己身体和灵魂状态的情况下就做出了决定。这种行为本质上是一种侵犯同意的行为,它通过信息不对称和隐瞒来操控这些少女的命运,从而实现自己的目标。而这些少女们的反应:恐惧、惊愕、痛苦,正是在体现她们对自己的同意权和自主权的保护,也是对现实社会中同意与权力关系的隐射,尤其是对弱势群体的操控和剥削。}

曾有位大学教授给过我这样一条奇怪的建议:如果你在写文章时卡住了,可以先寻找主题确切的中间点,然后围绕那里展开你能想到的一切。虽然我在《魔圆》的主题上完全不缺乏灵感,但既然这集结尾恰好是动画的中间点,倒是个不错的时机来讨论《魔圆》中有关“同意”的问题。
  
在这集的最后,角色们纷纷对丘比揭露的最新真相表现出恐惧:魔法少女的身体早已死亡,仅仅是一副空壳,但只要灵魂宝石完好,身体就可以无限修复。唯有灵魂宝石受损,才会伤害到魔法少女本身。但小圆将沙耶香的灵魂宝石扔下桥,实际上也导致了沙耶香的死亡(幸好焰及时找回)。  

在此想说明一个争议(虽然不是很有力,但也算是):这些女孩实在是太大惊小怪了。坦白讲,丘比所描述的交易看起来相当不错:魔法少女们大多也并未察觉到这种改变,身体机能实际上几乎与活着时一样,健康、更耐用、又能够治愈任何伤害。而且根据杏子的情况来看,一直吃垃圾食品似乎对这副身体没有任何负面影响。换做是我,一定会毫不犹豫地接受这个条件。  

事实上,灵魂宝石的设定似乎明显参考了俄罗斯民间传说角色 \emph{Koshchyei Byessmyertnuy}(译为“不死者科西切”)。他将自己的灵魂藏在指头里,指头藏在蛋里,蛋藏在鸭子里,鸭子藏在兔子里,兔子又藏在铁盒子里,最后这个铁盒被埋在位于偏僻小岛上的一棵冬青栎树下。科西切是一个诱拐年轻女子的恶棍,只有当英雄找到了那个蛋时,他才会受到伤害\cite{ref34}。这种做法对科西切来说非常有利,这也解释了为何他和他在流行文化中的“后继者”(如《龙与地下城》\cite{ref35}中巫妖的护符匣和《哈利·波特》\cite{ref36}中伏地魔的魂器)会采用类似的保护手段。  

但科西切和沙耶香之间有一个重要区别,那就是以知情为前提的同意。传说中表明,科西切知道自己在做什么,并仍选择这么做。而沙耶香并不知道她的灵魂已经禁锢在灵魂宝石中,也不知道自己身体在违背自身意愿的情况下就被改变了。在之后的剧集中,她不相信自己的“尸体”能够怀孕,她或许曾经这么想。无论如何,沙耶香、杏子和小圆在这集中的恐惧反应,表明了她们三人都将这视作为一种极度的侵犯。  

丘比辩解道,它不理解人类为何如此在意灵魂在哪,这简直无关紧要;它清楚地知道人类在乎,但毫不关心为什么在乎。而且丘比明显隐瞒了这些关键信息,然后在受害者拒绝自己时反过来指责她们。从本质上讲,它就是在为自己的行为辩解:“沙耶香并没有说不。”  

“不就是不”常常作为女权运动中的口号,尤其在涉及同意与身体自主权的议题中\cite{37}。然而,正如丘比所表现的那样,这总比忽视“不就是不”要好,但作为标准来说它并不完整。比起“不就是不”,更重要的是“是就是是”,即我们所说的“肯定性同意”标准\cite{ref37}。拒绝的缺失并不能成为侵犯行为的充分理由,因为这可能意味着那人无法反对,就像沙耶香无法反对契约中她所不知晓的那部分一样。  

与剧中另一幕也隐射着关于尊重他人选择和自主权的问题:小圆与母亲谈起沙耶香情况的场景(当然,以含糊其辞的方式)。这里有两点很重要,首先是母亲指出:做对的事情不一定能带来幸福或好结果。这显然是在明确拒绝结果主义(即认为行为的道德性由其后果决定的元伦理观点\cite{ref38}),这一立场与本剧的整体立场相吻合(因此丘比总是被描绘为一位典型的结果主义者)。  

与母亲对话的这一幕明确拒绝结果主义,本集结尾在桥上的场景也含蓄地否定了结果主义。丘比的立场是结果主义的:灵魂提取对魔法少女们是有益的,她们得以这与魔女战斗并存活,但知道了真相的魔法少女们往往感到不安,因此最好的做法就是在不告知她们的情况下提取灵魂。音乐与构图(尤其是丘比尽管个头小,却其阴影被描绘得盖住了女孩们)明确表明了,动画反对丘比的辩解,与女孩们的恐惧共鸣,进而拒绝结果主义的观点。  

这场母女对谈的第二个重要意义在于对沙耶香的描绘:她常常做好事却往往适得其反。这一描述并不只适用于沙耶香,也包括焰,她不断试图拯救小圆,结果却让她在每次循环中都越发痛苦,成为了更强大的魔女。母亲的建议导致了小圆为她朋友犯错——这不仅是说将沙耶香的灵魂宝石扔下桥,也是说小圆在最后一集成为魔法少女(焰一直在试图阻止)的选择。鉴于背景音乐的选择,这种双重含义并非偶然,似乎很明显:这集母女对谈的背景音乐 \emph{Clementia}(怜悯),在最后一集小圆许愿后所用的 \emph{Sagitta Luminis}(光之箭矢)中得以重奏。  

此外,为了“拯救”别人而犯错的想法在《叛逆》中得到了进一步阐述,在那里小圆与焰为对方采取了看似极不明智的行动,到那时我们再讨论吧。
