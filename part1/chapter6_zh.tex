\chapter[即便是纯粹的道德行为(这种事绝对很奇怪啊)]{即便是纯粹的道德行为\protect\footnotemark(这种事绝对很奇怪啊)}
% 纵然那是纯粹道德之举
\footnotetext{译注:``Even a purely moral act that has no hope of any immediate and visible political effect can gradually and indirectly, over time, gain in political significance.''
\par
“即使一个纯粹的道德行动,眼下并不显现任何直接或明显的政治效果,但在长期的历史进程中,它仍可能酝酿出深远的政治意义。”
\par
本章中,作者通过对《魔圆》剧情中有关“同意权”和“自主权”的探讨,阐释了道德行为如何在看似无关紧要的情况下,产生巨大影响力的。丘比隐瞒灵魂宝石的真相与魔法少女签订契约,导致她们在未能完全理解自身身体与灵魂状态的情况下做出决定。这种行为本质上侵犯了“知情同意”原则,它利用信息不对称和隐瞒来操控少女们的命运,以实现自身目标。少女们表现出的恐惧、惊愕与痛苦,既是对自身同意权与自主权的维护,也隐射了现实社会中有关同意与权力关系的问题,尤其是对弱势群体的操控与剥削。}

有位大学教授曾给过我这样一条颇为古怪的建议:如果你在写作时卡壳了,不妨找到作品主题的确切中点,然后围绕你在那里发现的任何内容展开论述。虽然我在《魔圆》的主题上毫无灵感枯竭之忧,但既然本集末尾恰好是整部剧的故事中点,此刻来讨论剧中涉及的“同意权”问题,再合适不过了。
  
本集最后一幕中,角色们因丘比揭露的真相而惊恐万分:魔法少女的身体并非活体,而仅仅是躯壳;若灵魂宝石完好无损,身体便能无限修复。唯有破坏灵魂宝石,才能真正伤害魔法少女本身——而小圆将沙耶香的灵魂宝石扔下桥,实际上就导致了沙耶香的暂时性死亡(多亏焰及时找回)。  

诚然,不妨说(尽管这论点站不住脚,但总归能提一提):女孩们的激烈反应纯属小题大做。坦白讲,丘比所描述的交易听起来相当诱人:身体的生理体验与活着时几乎无异,大多数魔法少女甚至从未察觉变化,而且这具身体完美健康、更为耐用、又能治愈任何伤病。此外,从杏子的情况来看,这副身体似乎还能毫无负担地随便享用垃圾食品。换作是我,那肯定会立刻答应。

事实上,灵魂宝石的设定显然借鉴了俄罗斯民间传说角色 \emph{Koshchyei Byessmyertnuy}(译为“不死者科西切”)。他将自己的灵魂藏在手指头里,切断手指后将其藏在一枚蛋中,蛋藏在一只鸭子里,鸭子藏在一只兔子中,兔子又被装进一个铁箱里,最后铁箱被埋在遥远岛屿上的一棵绿橡树下。科西切是位诱拐少女的恶棍,只有当英雄找到那个蛋时,才能伤害到他\cite{ref34}。科西切这样做的优势显而易见,这很可能也是其流行文化中的“后继者”采用类似保护手段的原因(其中最广为人知的例子,当属《龙与地下城》\cite{ref35}中巫妖的命匣和《哈利·波特》\cite{ref36}中伏地魔的魂器)。 

然而,科西切与沙耶香间存在一个根本区别,是否获得知情且明确的同意。传说表明,科西切知晓自己在做什么并主动选择这么做。而沙耶香却全然不知自己的生命已系于灵魂宝石,更不知晓自己身体已在违背其意愿的情况下被强行改造。后续剧情中,她提到自己不相信这具新身体能够生育——这或许曾是她对未来生活的美好期许。无论如何,沙耶香、杏子和小圆在本集中表现出的恐惧充分表明,三人都将此视为一种根本性的侵犯。 

丘比的辩解是,它不理解人类为何如此在意灵魂的位置。但这纯属避重就轻的胡扯;既然它心知肚明人类在意,纠结原因本身就毫无意义。丘比在签订契约时故意隐瞒关键信息,随后又在受害者拒绝接受时指责她们。本质上,它是在用“沙耶香从未说过‘不’”来为自己的行为开脱。  

“不就是不”,在争取女性权利、尤其是涉及同意与身体自主权的运动中,常被用作为口号。然而,丘比的例子表明,虽然承认“不就是不”肯定比不承认要好,但这仍是一个不完整的标准。比“不就是不”更重要的是“是就是是”,这即是“肯定性同意”的核心含义\cite{ref37}。沉默并不能成为侵犯行为的充分理由,因为这可能意味着当事人无法反对,正如沙耶香无法反对她根本不知情的契约条款一样。 

这种尊重他人选择与自主权的问题,在本集另一幕中得到了耐人寻味的呼应:小圆与母亲(当然是用最含糊的措辞)谈起沙耶香的近况。这里有两点至关重要:第一,母亲指出了做正确的事并不总能带来幸福或好结果。这实质上是明确拒斥了结果主义伦理立场(一种元伦理观点,即认为行为的道德性取决于其后果\cite{ref38}),这与整部剧所持的立场大体一致(正因如此,丘比始终被刻画为一个纯粹的结果主义者)。 

在母亲对话场景中明确拒绝结果主义,其深意延伸至本集结尾的桥头场景——后者同样以隐晦方式否定了该立场。丘比的立场是结果主义的:灵魂抽取对魔法少女有益,因为她们得以能够对抗魔女并活下来;但得知此事往往令她们痛苦不堪,因此最佳做法是抽取灵魂却不告知真相。该场景的音乐与构图(尤其是丘比虽然体型小巧,但其镜头呈现却使其阴影笼罩着女孩们)清晰地表明,剧作不仅拒绝了丘比的说辞,更与女孩们的恐惧共情,也就是彻底摒弃了结果主义的观点。  

这场母女对谈的第二个深意在于,将沙耶香刻画为“秉正心而为,招悖逆之果”的角色——且这种评价绝非她所独有,也包括焰。她反复尝试拯救小圆,结果却让小圆在每一次时间轮回中承受更多的痛苦,并成为了越发强大的魔女。母亲的建议促使小圆为朋友行差踏错——这不仅指将沙耶香的灵魂宝石扔下桥,也指向她在最终话选择成为魔法少女(这正是焰一直竭力阻止的事)。鉴于背景音乐的选择,这种双重含义并非偶然:本集母女对谈的配乐 \emph{Clementia}(怜悯),在最终话小圆许愿后的 \emph{Sagitta Luminis}(光之箭矢)中以变奏形式再现。  

此外,这种“为拯救他人而犯错”的理念在《叛逆》中也得到了重申,小圆与焰都为对方采取了看似极不明智的行动——不过更多细节,留待我们讨论到那里时再展开。
