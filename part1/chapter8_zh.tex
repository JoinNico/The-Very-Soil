\chapter[熵增是宇宙基本法则(我真是个笨蛋)]{熵增是宇宙基本法则\protect\footnotemark(我真是个笨蛋)}
\footnotetext{``Just as the constant increase of entropy is the basic law of the universe, so it is the basic law of life to be ever more highly structured and to struggle against entropy.''“正如熵增是宇宙的基本法则,生命的基本法则也是不断追求更高的结构,并与熵作斗争。”
\par
熵象征着情感与精神的腐化,灵魂宝石不断地变暗,而她们的心理状态也在不断恶化,这正是“熵增”的一种隐喻。抑郁、绝望、和精神崩溃像熵一样不可避免地在魔法少女们的生活中不断加剧,她们所做的努力与牺牲,似乎始终无法完全抵抗这种“熵增”。尽管如此,少女们依然在不断的斗争,试图找到脱困的希望,抵抗内心的黑暗和绝望,这正是“魔法少女”这一角色的核心冲突所在。正如 Havel 所言,生命的本质是在与熵的斗争中寻求结构和意义,而《魔圆》通过展现这些角色的内心挣扎,揭示了这一哲学主题。}

《魔法少女小圆》第八集是一个典型的情节陷阱,这种陷阱在像《魔圆》这样的作品中常见,尤其是依赖“惊人的转折\footnote{这种风格在日本很常见,但并非独有。它有着悠久的历史,以一种叫做“起承转合”的修辞和叙事结构形式存在。这种结构起源于中国四行诗,其中第一行(起)引入一个话题,第二行(承)展开话题,第三行(转)引入一个惊人的转折或表面上无关的内容,第四行(合)将转折整合到原话题中,从而得出一个新颖或出乎意料的结论。}\cite{ref43}”来推动剧情的作品。具体而言,这一集的一个出色角色篇章最终在结尾时被一个毁灭性的情节发展所压倒。因此,本章我们将忽略这一情节发展,下一章再详细分析,先专注于本集前十八分钟。

这十八分钟构成了本剧中最为紧凑的部分,我们几乎所有的场景都集中在沙耶香身上。即便是其中少数几个不直接与她相关的场景,也都是围绕沙耶香的卫星角色——恭介和仁美展开的,虽然沙耶香并不在其中,但其延续部分却是以沙耶香的视角呈现的。因此,这一集几乎全部致力于描绘沙耶香的最终崩溃,作为魔法少女所承受的压力、得知丘比对她身体所做的事后彻底失去了自尊,以及她拒绝净化灵魂宝石,这些因素共同摧毁了她的精神状态。

这一崩溃不仅揭示了沙耶香所经历的种种,也揭示了在这个世界中成为魔法少女意味着什么:就是不断与抑郁作斗争——而本剧对抑郁症的呈现异常准确,真实再现了该病症患者的内心体验。每当使用魔法时,黑暗就会在她们的灵魂宝石中积聚。这种外部力量,腐蚀并扭曲了她们的世界观和思维过程,形成了一种绝望的循环,让宝石变黑得很快更深。她们生活中的事件并不会直接引发这一循环,但却能恶化、加剧这一过程\cite{ref44}。

在登场的所有魔法少女中,抑郁是一个共同主题,尽管每个人表现抑郁的方式各不相同。比如,杏子在初看之下似乎并没有表现出明显的抑郁迹象,但上一集已经很明确地表明,她已经完全与人断绝联系,埋头沉溺于享乐主义的冲动中,逃避过去的痛苦。她回忆中的画面,比如她经常独自弯腰瘫坐在一个大空间中,或是在黑白剪影中唯一一只色彩的玩偶,清楚地表现了她的孤独感。而且,作为一个曾饱受饥饿之苦的人,杏子将食物与安慰紧密联系在一起,正如她在这一集里临近沙耶香变成魔女之前,给她零食一样;她也在与小圆策划试图将沙耶香从魔女状态中救回时,向小圆提供了食物;以及在下一集里,她把沙耶香尸体藏在公寓里的那堆包装纸和容器中,都可以证明这一点。对杏子来说,食物是建立原本被她所否定的与他人和世界联系的方式,也是她用来治愈情感的方式。换句话说,她不断进食的原因可能是她患有饮食障碍,如果她仍是个人类的话,可能会对身体造成严重伤害。

与杏子形成鲜明对比的是沙耶香,她会向外表露自己的抑郁。对沙耶香来说,反复出现水的意象现在有了双重含义:如同她在这集末尾魔女形态所暗示的,她正扮演着 Andersen\footnote{汉斯·克里斯蒂安·安徒生(1805年4月2日—1875年8月4日),丹麦作家、诗人。安徒生以其童话作品闻名于世,童话中带有人文色彩与哲学思想。}笔下《美人鱼》中的女主角,那位为与自己所爱的男人在一起而牺牲了自己身体的少女,最终却发现那个男人完全不知道她的牺牲,并选择了别人\cite{ref45}。这一形象与如今人们熟知的迪士尼电影\cite{ref46}不同,后者讲述的是一位非人类少女通过极大考验和牺牲最终得以融入她所认为自己应属的种族。这一意象同样也暗示着沙耶香感觉自己正在溺水,被一系列毁灭性的失落和绝望压得喘不过气来。就在这一集开头,她已经经历了恭介从他离开医院的那刻起就对她的忽视,也发现自己并不是那个她曾经希望成为的正义化身,而是比大多数魔法少女都要弱的人。更糟的是,正如抑郁常常会这样做的,她扭曲了自己的感知,以最消极的方式看待所有发生在自己身上以及她自己做的所有事情。因此,在她看来,身体的变化意味着彻底失去人性,将她的一切价值剥夺殆尽;朋友小圆的关心变成了可怜,甚至在火车上那两位性别歧视者变成了一个她觉得不值得再去拯救世界的代表。

沙耶香崩溃的核心是她许愿治愈了恭介,却因而与他没有了任何联系。一个很重要的点:我们并没有听到恭介与仁美的对话,尽管上一集和《叛逆》都暗示他们确实开始交往,但沙耶香并不能知道这一点;她之所以如此认为,是因为她消极地看待所有事情,并且她强烈的自卑感使她将周围的人视为更为高尚。沙耶香无法想象仁美不会得到她想要的,因为在她扭曲的思维中,只有像她这样的人才永远得不到自己想要的,而不是她这样的普通人就总能实现她们的愿望。沙耶香在火车上的爆发同样回应了她对自己和恭介的情感。男人谈到冷酷地抛弃关心他们的女人时,这切中了沙耶香的内心,愤怒促使她(可能)杀害了那些男人,在某种程度上她也是在宣泄对恭介的愤怒,因此她紧接着说自己已经不记得她原本想要为什么而战了。

本集还展示了与沙耶香截然不同的另一种崩溃:焰的崩溃。像沙耶香一样,焰同样在抵抗疲劳。在那么几个转瞬即逝的镜头中,我们曾见到焰偶尔表现出愤怒或悲伤,有时是和其他角色之间的互动时,比如沙耶香和小圆在雨棚中的对话。这些场景具有一定的模糊性;它们显然是焰试图隐藏的情感的短暂呈现,但尚不完全清楚这些是否只是为了观众的利益而呈现的外叙事镜头,还是在这集揭晓焰真实身份后,展示了不同时间线中某些场景略微有所不同的演绎。无论如何,她在射杀丘比之后的绝望与痛苦,尤其是在她恳求小圆不要再牺牲自己要珍惜生命时,我们才第一次在剧中时间线中看到她的真面目。甚至小圆也意识到了这一时刻的重大意义,这是她第一次完全意识到自己曾经见过焰。

我们在剩下的时间里看到焰的冷静、高效则是对“功能性抑郁”或叫做“慢性抑郁症”的描绘,一种比常见的抑郁症不那么严重、但持续时间更长的抑郁症\cite{ref47}。就像杏子一样,她在某种程度上是在自我药疗,但她选择的“药物”是她对目标的不停追求。她没有希望,也不相信自己能够成功,或者认为任何事情会有所好转,但她得以能够坚持下去,是因为她有必须要做的事情。只要她继续前进,继续朝着无法实现的目标努力,她就可以避免去思考她的处境有多么无望,或者她经历了多少痛苦。只要她继续前进,她就不必去想自己有多么孤独,或者她通过努力保护小圆,实际上已经把自己与小圆产生隔阂了。正如沙耶香敏锐地观察到的,焰已经放弃了生活,不同于沙耶香,她仍然有自己的信仰。只要焰能继续说服自己,除了拯救小圆之外其他什么都不重要,她就能相信她对绝望和孤独的感受不重要,也就不会变成魔女。在这方面,焰也是麻美的对立面,因为麻美也是通过全身心地投入一个事业,来抵抗魔法少女孤独与寂寞感的。

即便是小圆,也偶尔表现出一种普遍的抑郁情绪,尤其是低自尊,以及如焰所指出的那种,随时准备为别人牺牲自己(即总是将他人置于自己之上)。这一点与我们最终将在其他时间线中见到的小圆有着很大不同,这暗示着,正如其他时间线的一些片段通过小圆的梦境和闪现出现在她的感知中一样,这种抑郁感同样渗透进了作为魔法少女的小圆生活。这个描写的不完全和不一致性,可以被理解为小圆在这一时间线并未真正成为魔法少女的结果。

这些关于抑郁症的高度真实且多样的描写,强烈反映了创作者的精神健康历史,尤其是考虑到如我在第四章和第五章引用的, 虚淵\, 玄在 \emph{Fate/Zero} 第一卷后记中提到的内容。当然,通过作品去诊断艺术家既不明智也极其不礼貌。区分作品中的隐含作者和实际作者是至关重要的,前者是一种由观众通过作品试图窥视创作人背后个体的构建,后者是一个完整而真实的人,只有真正认识他们的人才可能了解。事实上,创作者是一个庞大的集体,包括编剧、导演、角色设计师、动画师和配音演员,而隐含作者始终是一个单独的个体。尽管如此,隐含的“ 虚淵玄”看起来确实相当抑郁,处于一种绝望的状态,以至于他无法想象一个正面的、有美好事情发生的世界,亦或者是创造出描绘这种世界的艺术作品。

在接下来的剧集里,我们将明确看到丘比的目标是利用魔法少女和魔女释放出的能量来对抗熵增,抵制宇宙的不可避免衰亡,实际上就是将这种衰亡从物理现实转移到魔法少女的情感状态上。这个角色们所生活的宇宙充满了衰落,而这种衰落等同于抑郁和绝望。但这个衰落的宇宙本身存在于一个处于抑郁中的隐含作者的内心里,正挣扎着试图超越自己的局限。换句话说,丘比的做法是一种通过将抑郁从作者转移到角色上,来延缓并驱除抑郁的尝试。

然而,这也是本集中的一个转折点,除了小圆,其他角色开始明确将丘比视为敌人。特别是本集结尾,在构图、音乐和节奏上都非常明确地展现了丘比作为一个彻头彻尾的怪物,他自鸣得意地从这些根本不知道自己到底在同意什么的魔法少女的痛苦中获利。丘比试图操控小圆签订契约去拯救沙耶香,而后是那群性别歧视者在火车上的对话,他们将身边的女性当做是低等存在,讨论着如何操控她们,紧接着沙耶香在之后的那句话:“我已经不记得我当时为了什么而战斗。”将丘比的恶行与一个更大问题联系在一起,即人类是否真的值得被拯救。

然而,如果说丘比是隐含作者处理抑郁过程的守护者,那么这也将作者置于同样负面的光环之下,并进一步暗示他完全意识到并刻意强调这一点。这反过来又表明,他正在寻求另一种解决方案。

然而,要找到解决方案,就必须打破他一开始所建立的系统,也就是整个《魔法少女小圆》宇宙。而要做到这一点,他必须首先摧毁它所属的作品类型。
