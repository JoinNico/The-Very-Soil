\chapter[角色不仅仅被人为安排(怎么可能会后悔)]{角色不仅仅被人为安排\protect\footnotemark(怎么可能会后悔)}

\footnotetext{译注:``The role of the writer is not simply to arrange Being according to his own lights; he must also serve as a medium to Being and remain open to its often unfathomable dictates. This is the only way the work can transcend its creator and radiate its meaning further than the author himself can see or perceive.''

“作家的角色不仅仅是按照自己的想法来安排存在,还必须作为表现方式而存在,并维持难以琢磨的行动原则。只有这样,作品才能超越其创作者,散发出超越作者自身所能看到或感知的意义。”
\par
这一点在《魔法少女小圆》中得到了充分体现,例如,小圆从一开始的软弱和依赖,到逐渐接受自我牺牲的命运,最后选择挑战宿命,这一转变表明了她的内心复杂性和成长;杏子从一开始的冷酷无情到最终为他人牺牲,也反映了她在困境中所作出的非理性、情感驱动的决定。少女们的选择和行动并非简单的反应或遵循固定逻辑,而是充满了冲突、矛盾和不可预见性,保持着“难以琢磨的行动原则”。
\par
丘比作为外部力量,虽然设定了魔法少女的命运框架,然而这些角色在面对这些困境时并不按预定的轨迹行走。她们做出的选择、牺牲和反抗展现了她们作为独立个体的复杂性,进而推动了故事向不可知的方向发展,让故事拥有了更高层次的意义和冲突。
\par
可以说《魔圆》不仅仅是一个悲剧,也是一部充满哲学性反思的深刻故事。它是对命运、牺牲、自我认知以及自由意志的深刻探讨,展现了角色们如何通过她们“难以琢磨”的行动,让作品本身的意义超越了原本的构思,变得更加深邃有力。}

在第五集,第二个篇章的开头继续呼应第一章。正如第一集和第四集都是引子一样,委婉来说第二集和第五集都是在定位角色并探索新世界的本质,意即这一集剧情发展缓慢。

本集以丘比和沙耶香签订为魔法少女的回忆开场,推测应该是发生在第四集中沙耶香向恭介保证魔法存在之后不久。尽管丘比自系列开始以来一直显得阴森恐怖,但这段回忆是目前为止它作为一个(字面意义上)黑暗角色的最长描写,而且镜头构图和光影都明显暗示着沙耶香的死亡场景。正如焰后来说的那样,沙耶香的命运此时已经注定。实际上,她已是个行尸走肉(我们将在几集中了解到这一点)。此外,通过将丘比置于深色阴影中并在它身后设置了一大株植物,这几个镜头看起来像是丘比有着多条尾巴,视觉上让人联想到强大的九尾灵狐\footnote{译注:中国神话生物,山海经《南山经》云:“青丘之山,有兽焉,其状如狐而九尾。”至少在宋代以前,九尾狐都是属于祥兽,约宋代后,九尾狐才开始偏向魅惑、欺骗等形象。及至明代小说《封神演义》成书后,九尾狐为恶兽的观念已大致成定局,狐狸成为狡猾的代名词。在《封神演义》中,九尾狐乃奉女娲之命前去诱惑商纣王,九尾狐化身妲己,做了许多坏事让纣王失去了民心与江山,而后周武王伐纣。},一个(通常为邪恶的)骗子化身\cite{ref32}。

我们还首次看到灵魂宝石的形成过程。签订仪式强烈暗示丘比实际上是从女孩们的心脏中将灵魂宝石拉出来,把它们从已有的某些东西中塑造成型。这与前几集中麻美和丘比所说的“尽管小圆还没有成为魔法少女,但感觉她体内拥有巨大的力量”相吻合,这也解释了为什么丘比不能仅靠自己的力量实现目标:尽管它的能力巨大,但却非常有限。它可以实现愿望、充当心灵感应的中枢、控制谁能看到它,并且(正如我们后来将在系列中了解到的)能够同时存在于多个地方,但却无法像魔法少女和魔女那样直接用魔法改变现实。它的一些言论甚至暗示,能够实现的愿望取决于许愿的魔法少女的力量,考虑到后几集中提到沙耶香并不算很强大的魔法少女,有可能她只治好了恭介的手,而没有治愈他的全身,正是因为她的力量不足以做到这一点。

换句话说,丘比是一个促成者。它使准魔法少女能够触及她们体内已经存在的力量,以便为它狩猎魔女或是相互厮杀。正如我们在本集中所见,它完全乐意制造冲突,尽管它知道杏子即将到来,但还是赋予沙耶香力量;同时,它一边向杏子提供信息,一边让沙耶香蒙在鼓里,因为她们的战斗符合它的目的。

由于丘比的操控,杏子接替焰在第一章中的角色,成为行为可疑的敌对魔法少女。杏子正如麻美所警告的那样:愿意与其他魔法少女战斗,只关心击败魔女的奖励,毫不在意保护见泷原的群众。她愿意让魔女的使魔杀人直到使魔变成魔女,再加上她关于食物链的评论,以及她不停地吃东西,暗示杏子将“吃”视为权力的表达,奉行“强权即公理”的权力哲学。与她对立的沙耶香则接过麻美的角色,成为“正义”的魔法少女,为保护他人而战,坚信强者有义务保护弱者。

当焰打断她们间的战斗时,揭示出她也已经接替了麻美在第一章中的角色。焰不再试图抹去传统魔法少女元素并用《魔圆》这部剧来取而代之;因为这已经实现了。相反,她现在试图阻止故事向下一个逻辑发展,也就是标题中所预示的魔法少女小圆的出现。当小圆恳求她帮助沙耶香时,焰拒绝了,但当唯一的选择是让小圆成为魔法少女时,焰别无选择只能插手。

最后只剩下小圆和丘比。小圆做出了一个有趣而深思熟虑的选择,她没有改变自己的角色定位,正如她在麻美手下是不会魔法的帮手和知己一样,她同样愿意成为沙耶香的帮手。尽管她情理之中地对成为魔法少女感到害怕,但她仍愿意冒着生命危险站在沙耶香身边。正如我们在本集结尾看到的,如果必要,她甚至也愿意成为魔法少女。丘比同样没有改变,而是逐渐向观众揭示它的角色定位是什么,从一个道德和立场成疑的角色逐步转变为明显的操控者,主动帮助杏子并隐瞒她的存在,不让小圆和沙耶香知晓。

但它的角色定位究竟是什么?本集早些时候,沙耶香带着恭介上到屋顶,他的家人将小提琴还给他,恭介演奏了《圣母颂\footnote{译注:由法国作曲家Charles-François Gounod 创作,伴奏取自J.S.巴赫的《十二平均律钢琴曲集》。}》。这一幕之所以引人注目,是因为场景中的一些元素不断将恭介投射到丘比上。首先,这是沙耶香在本集开头与丘比签订契约的同一地点,尽管他们此时位于同心花环的外侧而不是内侧,但沙耶香和恭介的位置与回忆中沙耶香和丘比的位置相同。其次,在某些镜头中,恭介以剪影的形式出现,就像回忆中的丘比一样,而且从沙耶香的视角看,恭介与回忆中丘比所站的位置在同一直线上。最微妙但也是理解恭介-丘比平行关系最重要的一点就是,丘比伸手进入沙耶香的心脏,将她变成魔法少女,正如恭介的音乐触及沙耶香的心,使她对恭介产生了感情,并促使她成为魔法少女。当然,我们会看到在接下来的几集中,恭介对沙耶香相当冷漠,因此他与丘比有着共同点:共同促成了沙耶香成为魔法少女,却对她的感受毫不在意。

场景结束时,恭介放下了小提琴,而《圣母颂》依然作为背景音乐继续播放,贯穿了整场戏,从叙事层音乐(也就是剧中角色能听到的音乐)转变为外叙事层音乐。直到目前为止这种跨越叙事层的能力都被呈现为魔女的能力。为了表现她们的异质性和角色们对这种变化的恐惧反应,其结界通常采用与整部剧截然不同的独特艺术风格。这也是外叙事效果对叙事的反应。此时,恭介是唯三以反方向跨越这一障碍的非魔女角色,也是唯一一个没有任何明显“魔力”却能这样做的角色。

这是因为他的音乐是情感表达,而情感就是魔法;正如我们将在本系列的后期看到的,人类的情感是所有魔法的源泉。因此,人类艺术本质上也是魔法,艺术表达可以重塑现实。在这里确实也是如此:沙耶香因恭介的音乐萌生好感,而这也是她成为魔法少女的原因。魔法少女沙耶香所违反的所有物理法则,都是恭介音乐的结果。

但,如果恭介能够跨越叙事层,且他又与丘比有着深刻的平行关系,那么是否意味着丘比也能如此?确实如此,它是唯三中第二个跨越这一障碍的非魔女角色。丘比是一个在故事中的外叙事层实体。

想一想,丘比创造魔法少女为自己服务,知道她们会遭受痛苦,甚至依赖这种痛苦。它想让小圆成为魔法少女,并为这一最终结果铺设一切行动,因为她可以帮助它解决问题。它设计让杏子与沙耶香战斗,同样不关心对她们的影响,除非这些影响符合他的目标。那么这些目标是什么?就是让它所创造的魔法少女经历悲剧和绝望。

在这一点上,它让人强烈联想到另一部魔法少女作品《萩萩公主》中的反派多罗斯玛亚\cite{ref33}。在那部作品中,多罗斯玛亚最终被揭示为一个拥有将自己的创作赋予生命的作家,他利用这种能力将一个小镇困在悲剧的循环中\footnote{《荻荻公主》第21话:编织者们$\sim$无言歌$\sim$}。然而,因他变为主角的鸭鸭为终结悲剧循环而牺牲了,迫使多罗斯玛亚离开这个世界,进入其他“故事”,迎来了一个幸福的结局。将此情节与我们在上一章讨论过的虚淵\, 玄后记相比较:“我对人类所谓的幸福充满了憎恨,不得不将我倾注心血创造的角色都推向悲剧的深渊。” 以及稍后提到的,“只有具备这种能够唱出人间赞歌的高洁灵魂,才能将这个故事拯救。”\cite{ref29}(参见第20和第21章进一步探讨丘比与《萩萩公主》中多罗斯玛亚的原型,E.T.A. Hoffman的角色多罗斯玛亚(Drosselmeyer)的关系。)

除此之外,《魔圆》还是关于“同意”与“自主”的动画。到目前为止,剧中已经有了几次暗示,但在下一集中,这个主题将变得不可否认。鉴于此,有什么比控制角色行动的人更适合作为反派呢?丘比的所有能力都相符:创造魔法少女、知道她们在想什么、无处不在。它的所有动机,让魔法少女经历情感的起起落落,尽可能让故事世界持续运转,这也与它无意识中扮演的角色一致:《魔法少女小圆》的作者。
