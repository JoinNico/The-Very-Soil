\chapter[并非一时兴起(那真是太令人高兴了)]{并非一时兴起\protect\footnotemark(那真是太令人高兴了)}

\footnotetext{译注:``You do not become a `dissident' \textbf{just because you decide one day to take up this most unusual career}. You are thrown into it by your personal sense of responsibility, combined with a complex set of external circumstances. You are cast out of the existing structures and placed in a position of conflict with them. It begins as an attempt to do your work well, and ends with being branded an enemy of society.''
\par
“一个人之所以成为一名‘不同政见者’,并非仅仅因为他某一天这人忽然决心投入到这个非凡的事业中。他自己的责任感,以及各种复杂的外在因素,驱使他加入这一事业。他被现制度抛弃,而且置身于与之相冲突的地位,事情以努力做好工作的良好愿望开始,以被打成社会的敌人告终。”
\par
角色们“并非一时兴起”选择了成为魔法少女,而是像 Havel 所言,她们的“事业”是由外部环境和责任感驱使,最终走上了必然不平凡的命运道路。例如,小圆最初并不打算成为一名“魔法少女”,更没有意识到她的选择将如何改变她的命运和整个世界。然而,随着小圆被外部压力和责任感推向了这一命运,她的选择逐渐从“非自愿”变成了“必须承担的责任”。同样,焰也在不断地遭遇着外界和内心的拉扯。最初她并没有选择成为“不同政见者”,但责任感和对小圆的保护欲让她不得不与既有的秩序对立。她的斗争也从一个看似平凡的尝试变成了与现状对抗的使命,也促使她在一开始被打上“反派”的标签。
\par
更进一步说,小圆与焰的转变并不是单纯的一个“选择”,而是一种历史进程中的必然。她们的觉醒标志着秩序的“盖子”裂开,是魔法少女世界秩序重建的关键时刻,恰如 Havel 对变革的理解,历史不是由人随意选择的,而是由外部环境、责任感与内心的驱动共同塑造的。角色的命运并不是她们的一个简单选择,而是由内外复杂因素推导出的必然结果。}

魔法少女的变身动画有着一段耐人寻味的历史与作用,这也使得麻美的变身再度成为《魔法少女小圆》第二集开场的不二之选。这个场景最早起源于1973年永井\, 豪\footnote{译注:永井豪(1945年9月6日—),本名永井洁,出生于石川县轮岛市,日本漫画家,东京都立板桥高等学校毕业。代表作有《恶魔人》、《魔神Z》、《甜心战士》等。}的漫画及改编动画《甜心战士》,其以自身兼有的魔法少女与少年(即“为男孩的”设计)元素为卖点\cite{ref19}。这一场景很快就成为该题材的标志性特点,到了1990年代,魔法少女类的作品或多或少都会融入少年元素\cite{ref20}。

在大多数魔法少女作品中,变身动画起到了多重作用。传统场景(以《美少女战士》为代表)通常会展示一连串裸体角色的彩色剪影,在其身上魔法少女服逐渐形成。魔法少女摆出一系列姿势,镜头围绕着她旋转的同时,营造出一个孤独的裸体女孩在众多男性凝视\footnote{这里“男性凝视”的概念指电影和电视镜头构图中一种倾向,即镜头模仿异性恋男性观众的窥视方式,例如通过在镜头中长时间聚焦于女性身材曲线。}\cite{ref21}下,舞动的效果。用叙事理论来说,这一瞬间象征着角色赋权,伴随着背景音乐也唤起了我们类似的情感。然而,镜头的运作同时也将角色作为一个供观看的对象;换句话说,她是被赋予力量的,但这种力量又并不大。她依然是在表演着少女气质,并且服从于霸权男性气概\footnote{简而言之,霸权男性气概指的是一种文化倾向,即将男性气质与权力等同。诸如暴力、支配力和果断等权力的表现被视为固有或适当的男性特征。相比之下,表演性女性气质指的是一种文化倾向,即将女性气质视为一种表演,一种社会游戏,女性通过这种表演为自我表达创造一个有限但相对安全的空间,但代价是迎合霸权男性气概。}\cite{ref22}。这个变身场景基本上是在传达这样一个信息:“这位少女是一位强大的守护者,但她依然不会威胁到父权主义或脆弱的男性自尊;她的存在目的,不过是为了取悦你。”这正是魔法少女题材逐渐与刻板印象中变态“御宅\footnote{例如,\emph{Anime News Network} 上对《魔法少女奈叶》的批评文章,称其为“面向御宅的娱乐,已被推到其逻辑极限”,特别提到其“萌系角色……和色情服务镜头”。}”产生联系的原因\cite{ref23}。

日本精神科医生斎藤\, 環\footnote{译注:斋藤环(1961年9月24日—),日本精神病学家、评论家,筑波大学医学医疗系社会精神保健学教授。专门研究青春期精神病理与精神病志学。}在其著作《战斗美少女的精神分析》中,将“御宅”定义为“技术迷、书呆子、粉丝”,以及还有比这些词汇更为尖酸刻薄的定义,指的是那些能够将他们所消费的虚构作品视为自己身份的一部分,并从中寻找性欲对象的人。换句话说,日本御宅文化的一个重要元素(在美国极客文化中也可以观察到)是作品及里面的角色属于观众,而这种归属感与(通常是异性恋男性的)性欲息息相关。正如斎藤 環所说:变身动画代表了一种加速成熟过程,实际上是动画给予观众的默许,允许他们对角色抱有性幻想。他认为,魔法少女(或者,他所说的“战斗美少女”,把这一概念扩展到科幻等其他题材的角色上)并非赋予女性观众们强大力量的工具,也非用来教育女孩传统女性美德的工具。她并不是一个观众要成为的榜样,而是一个观众要“拥有”的、通常是抱有性幻想的对象\cite{ref19}。

麻美的变身动画比大多数的要温和些,因为她并没有脱去校服,但镜头仍然在胸部和臀部上停留,强调着她的身材与服装。即使麻美在剧中暴力地消灭了威胁小圆和她的敌人,她仍是一个典型的少女形象,温柔而富有母性。在稍后的几分钟,询子也有着类似的变身。这是她在这集中唯一的出场,担任着一个母亲的角色,温柔地责备小圆昨晚没回家。但化好妆后,她便变成了冷静、雄心勃勃的公司高管,考虑着她的同盟与公司内部的权力斗争。麻美和询子的两个场景可以看作是一个过渡,从传统的女性角色:学生或母亲,转向传统的男性角色:战士或胜者。

尤其对于麻美而言,变身动画不仅仅是一个呈现魔法少女的常规元素,它还作为一种打破魔女迷宫异化感的方式,强化了魔法少女题材和父权社会的规范。它也是为了小圆与沙耶香而进行的表演,意味着麻美来拯救她们了。然而,值得注意的是,当麻美在本集结尾再次变身时,并没有采用完整的变身场景,只是短暂光芒与丝带交织之后,她便已变身为魔法少女了。完整的变身场景强调魔法少女之内存在“女学生”这一形象,是供观众消费的(小圆和沙耶香在叙事层\footnote{译注:叙事小说中的叙事层次是多重的。 Gérard Genette 区分了三种“叙事层次”,分别是,\par 外叙事层(extradiegetic level):可以理解为一个不属于故事本身的叙述者层次;\par 叙事层(diegetic level):指故事角色的层次,包括他们的思想和行动;\par 元叙事层(metadiegetic level):通常指“故事中的故事”,例如当故事角色自己讲述一个故事时。}中,男性凝视在叙事层外)。简短的变身场景则恰恰相反,它提醒观众,女性之内存在“战士”,不给观众拥有和消费她的机会。它不是强化社会秩序,反而在某种程度上破坏了秩序,暗示着魔法少女并不需要为男性观众表演或强调她的女性气质,以获得战斗许可。她所需要的,仅仅是一个契机。

但这无疑对沙耶香来说是足够了。麻美作为无辜者的守护者立刻吸引了她,毫不犹豫地支持麻美,讨厌焰。对沙耶香来说,魔法少女代表着一种明确的善恶对抗,她迫不及待地想站在正义的一方。她渴望成为一名魔法少女,为了正义而战,站在麻美身边,因此决定带着武器棒球棒去参加猎杀魔女的任务。与此相反,小圆更倾向于体验成为魔法少女的过程。她想理解麻美和焰,和她们成为朋友。她是为了酷和特别而想成为魔法少女,希望像麻美一样,因此她专门设计了一套魔法服装。换句话说,沙耶香专注于“做”,而小圆则专注于“成为”。沙耶香是积极的,但茫然;小圆则是明确的,但消极。

尽管两人在成为魔法少女目的上有所不同,但两人都纠结于要许下什么样的愿望。出乎意料的是,稚嫩的女孩们竟然去反思了她们的优越生活。作为孩子,她们很难对自己生活质量产生影响。不得不承认,她们的安全、健康、良好的教育背景、较高的社会地位、出众的外貌以及良好的经济条件(住在公寓的沙耶香看起来比小圆稍微不那么富裕些,小圆的房子非常大,但从后续与杏子的互动来看,两人似乎都没有经历过食物或住房问题),完全归因于非常幸运地出生在了特定家庭中,而非她们自己的努力。正如我们稍后看到的,沙耶香在恭介身上看到了随机事件如何轻易打乱一个人的生活。小圆也认同她提出的观点。两人都感到自己太过幸运,根本没有任何愿望值得付出丘比所要求的代价。

还有一个隐含的主题贯穿本集:魔法少女是某种生态系统的一部分。杏子在几集后明确指出了这一点,但在本集中我们已经了解到了:在这个生态系统中,魔女以人类为食,而魔法少女则以魔女为食,丘比则是分解者,它从捕食行为的副产品为食。但从社会角度考虑,它却是一个纯粹的捕食者,专门挑选群体中最脆弱的成员,利用她们为自己谋取利益。从麻美、焰和杏子回忆她们与丘比签约的经历中可以看出,丘比专门寻找那些正处于痛苦与绝望中的女孩,她们不仅愿意立刻接受契约,而且也很可能迅速转化为魔女。而小圆与沙耶香迟迟没有接受它的提议,正是因为她们太过幸运,要操控那些无欲无求的人来说是格外困难的。

但这通常应该不是魔法少女所面临的问题。丘比寻求着那些为它战斗的女孩以达成契约,在这两集的背景下,它的行为看起来比《魔卡少女樱》和《美少女战士》中的相应角色要好一些。在这两部作品中,小樱和月野兔都没有选择的余地。小樱被告知,因为她不小心释放了库洛牌,有责任收回这些卡片\cite{ref11};而小兔则只是被告知她注定要成为美少女战士。正如在现实生活和虚构作品中常见的那样,女性往往被期望接受她们将承担某些角色的事实,无论她们是否同意\cite{ref22}。(这并不是说男性就不会面临类似情况,只是他们更可能被给予选择的机会,并且更容易抗拒自己不得不战斗的命运。例如《新世纪福音战士》中,真嗣不断挣扎着反抗作为EVA驾驶员的命运,而小兔只能顺从接受了成为魔法少女的命运。)

当然,了解后续剧情之后,丘比狡诈的本质变得显而易见。它根本不是在寻求知情同意,而是在以令人震惊的方式欺骗和操控女性。即使在开头的剧情里,借助有关与魔鬼交易和骗子精灵的故事(这一点通过魔女受害者涂鸦中引用的 Goethe\footnote{译注:约翰·沃尔夫冈·冯·歌德(1749年8月28日—1832年3月22日),德国戏剧家、诗人、自然科学家、文艺理论家和政治人物,出生于神圣罗马帝国法兰克福,为古典主义最著名的代表。}的《浮士德》得到了强调),也暗示了丘比并所表现的那样可信。它那张难以捉摸的面孔进一步加剧了这种不安感。最初看起来或许是可爱或友好的,但随着观众对它的怀疑增加,那种缺乏情感的表情开始显得像是一种面具,让它更加难以琢磨,又进一步加深了观众的猜疑。小圆和沙耶香也并不是生活在《魔卡少女樱》或《美少女战士》那样的世界中(尽管小圆在这一集开头醒来的场景与《魔卡少女樱》第二集\footnote{在《魔卡少女樱》第二集开头,樱像《魔圆》第一次开头的小圆一样从床上醒来,以为前一集的事件只是一个梦,直到她意识到床边的布娃娃根本不是布娃娃。}\cite{ref24}有些相似)。但无论她们是否与丘比签订协议,魔法少女的世界很快也将渗透并侵蚀她们原本安全、舒适、优越的生活。

麻美希望保护无辜,维护着魔法少女题材应有的内容,但她终究会失败。尽管她将沙耶香的球棒变成典型的魔法少女形象:一根粉红色、卡通的法杖,但这个法杖在本集结尾部分的魔女战斗中并没有任何用处,最终只是帮助沙耶香与小圆暂时脱离险境,而麻美则继续战斗。同时,\emph{Magia} 这首预示着黑暗版《魔圆》的曲子在战斗中响起,魔女将其超现实的迷宫投射到了小圆世界里。尽管麻美再次战胜魔女,但她在那精彩的战斗中显露的过度自信(尤其在她陷入蝴蝶绳索陷阱时),表明了麻美始终在为注视着她的观众表演,而非专注于眼前来自魔女的威胁。

在本集的结尾,麻美已经明确了自己作为沙耶香与小圆守护与引导者的角色定位。只要她还在,她就会阻止住 \emph{Magia} 版《魔圆》不让其吞没掉《明天再见》版的魔法世界。她保护着小圆那幸福、安稳、子供向的小世界不被魔女那强烈情绪、复杂性和威胁所打破。然而,这一集也揭示了她的弱点。麻美必须离开才能推动剧情的发展的原因也不再模糊了,显而易见,她会因为自己的过度自信成为牺牲品而离开舞台。
