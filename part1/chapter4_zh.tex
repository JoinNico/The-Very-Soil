%Completed
\chapter[常存质疑性(奇迹、魔法,都是存在的)]{常存质疑性\protect\footnotemark(奇迹、魔法,都是存在的)}

\footnotetext{译注:``I think theatre should \textbf{always be somewhat suspect}.''“戏剧应常存质疑性。”

\par
这句话揭示了一种深刻的艺术理念:优秀的作品不仅仅是为了娱乐观众,也要引发思考,甚至让人质疑现实、质疑自身。《魔圆》正是这样一部作品,表面上看似是一部典型的魔法少女动画,但随着剧情的发展,它逐渐解构了这一题材,揭示出人性中更为复杂和黑暗的层面,诸如失落、抑郁、绝望和牺牲。
\par
从第四集开始,观众进入了该剧的“真实”核心,动画不再仅仅是一个充满希望和魔法的童话世界,而是一场深刻的心理与情感考验。虚渊玄的故事让我们质疑魔法少女这个角色的真正意义:她们的牺牲是否值得?她们的力量是否带来幸福,还是更多的痛苦?这正是作品“质疑性”的体现——通过扭曲和颠覆观众对魔法少女题材的预期,让观众对善恶、救赎以及牺牲的本质产生疑问。通过这种方式,这部动画的确达到了“常存质疑性”的艺术高度。}

《魔法少女小圆》第四集开启了该剧的第二个篇章。然而,这一篇章的主要焦点:成为魔法少女的沙耶加和新角色佐仓杏子直到集尾才出现的。本集的大部分内容更像是对第一篇章的尾声总结,同时也为第二篇章做了引入。

此外,我们可以认为这一集是该系列的首个“真正”的集数。也就是说,这是该剧首次公开展现其主导美学风格的一话,而非像前三集那样试图伪装成一部平凡的魔法少女作品。因此,这一集也是对整个系列的引入,因为它率先呈现了该剧最重要的主题之一:抑郁。

当谈论《魔圆》时,经常会引用编剧虚淵\, 玄的一句话,尽管这句话出自他早期作品 \emph{Fate/Zero} 第一卷的后记。部分原话是这样的:“事实上,我曾经不是这样的。我也写过即使没有完美结局,但到了最后一章,主角依然会坚信‘虽然前路艰难,但仍要坚持下去’的作品。然而不知从什么时候起,我再也无法写出这样的作品了。”\cite{ref29}接着他讨论了熵与失败的必然性,言辞中透露出某种抑郁的迹象——或者说是一种绝望。

当然,基于作品来揣测作者精神状态或个人观点总是有问题的。“隐含作者\footnote{译注:隐含作者,由美国文学理论家韦恩 $\cdot$ 布斯在《小说修辞学》中首次提出此概念,指文本呈现而经读者建构的作者形象,不同于实际存在、有血有肉的真实作者。故纵使真实作者否认,文学评论仍可将文本呈现的价值观、要素与手法归因于文本的隐含作者。}\cite{ref30}”与实际作者必然不同,后者作为真实的人(并不像那些诸如“角色”或者“隐含作者”这类虚拟人物),拥有完全不可知的主观内心世界,他人无法从外部感知或推断,但我们仍必须假定其(隐含作者)存在。换句话说,虚淵\, 玄很可能对自己的生活十分满意,只是单纯喜欢并且知道自己擅长写抑郁的角色和绝望的情境。具体情况是什么样,我们无法得知。

话虽如此,《魔法少女小圆》作为一部关于抑郁的故事还是很容易理解的。剧中的魔法少女们都表现出不同的抑郁症状,例如麻美的孤独感,焰的情感冷漠,以及沙耶香的整个剧情,失去挚爱后逐步绝望,最终走向自杀的故事。

这一集作为该剧的首个“真正”集数,到处充斥着抑郁和失落的情绪。剧集的前半部分大多围绕小圆试图平复失去麻美学姐的痛苦展开。在麻美死后的第二天早晨,早餐里的蛋黄都让她想起了麻美的发色,味道让她忍不住流泪。这一场景让我想起了另一部备受喜爱的魔法少女(magical girl)作品中的经典台词\footnote{译注:出自《吸血鬼猎人巴菲》第五季第16集的26分53秒至27分30秒,下文台词摘自其字幕。剧中的“魔法少女”非传统日本“魔法少女”,单单意为有魔力的少女。}:

“但我不明白!我不明白为什么事情会发生,也不明白要怎么挺过去,但她现在只是……,只剩躯壳了。我不明白她为什么不能回到身体里然后活过来,不要死去。这很愚蠢,又极端又愚蠢。……我喝着果汁时在想:她再也无法喝到果汁了,也不能吃煎蛋,打哈欠,或是梳头发了,再也不会了。但是没有人跟我解释为什么。”\cite{ref31}

不过,这一集中的失落不仅仅局限于麻美的死亡和小圆的悲伤。恭介只能听音乐,无法演奏的痛苦和愤怒同样也极具感染力。在得知医生建议他接受手腕永久性残疾的事实后,沙耶加决定成为魔法少女去拯救他,尽管她已经知道了这条道路充满了危险。即使她之后会展现出复杂的动机,但在这一刻,她是英勇而无私的。然而不幸的是,恭介感受不到自己自残手臂时的疼痛也预兆了:不久后,沙耶香也会陷入麻木的绝望状态,这种状态不仅将她的痛苦屏蔽,同时也隔绝了一切其他感受。此外,她选择接替麻美也预示了其相似的行为:贸然去拯救他人而不是先处理好自己的悲伤和失落。

当然,还有自杀团体。我们在剧中很少看到仁美出场,她突然地加入自杀团体,但剧情暗示了是魔女将脆弱的人们聚集成了一个自杀团体。然而,究竟是什么让仁美变得如此脆弱,我们并不清楚,就像第二集中的那位无名女性一样。尽管如此,有一些线索:在此集之前,仁美被描绘地非常繁忙,不仅要应对优等生的考试准备课程,还要接受日本上层社会的文化训练,如传统舞蹈或茶道课。在此集之后,这些课程再也没有被提及,或许暗示她在这次事件后减轻了自己的负担。另一种可能是她对自己暗恋恭介的害羞:直到她向沙耶香坦白这一情感之前,甚至连她的朋友都不知道;而在那之后,她马上主动接近了恭介。

让我们把更多注意力集中在小圆的脆弱上,魔女利用她未能拯救麻美的自责来攻击她。结合自杀团体中那位提到自己工厂破产的男人,以及仁美可能的脆弱点,这位魔女的攻击方式似乎是针对失败感和自卑感的情绪。

至少在对小圆的攻击中,魔女选择的手段是通过电视屏幕播放上一集的片段,迫使小圆面对她的无能,小圆的形象也开始模糊和扭曲,角色与背景之间的通常界限逐渐消失。换句话说,魔女让小圆直面她是动画角色的事实,剥夺她的身份。她不再是一个人,只是动画中的一个元素,因而她开始逐渐融入动画背景中。

然而,有一个奇特的事实,即使角色是虚构的,但她们产生的情感依然是真实的。即使我们知道小圆是虚构的,我们仍然会关心她,并为她被沙耶加所救感到高兴。至少在这一刻,这足以让她回归动画中的“现实”。至少在动画中,小圆依然是一个独立个体,而不是一个模糊概念。尽管如此,《魔圆》的这一集也已经为结局埋了伏笔:小圆将最终散于无形。

到了本集的尾声,第二个篇章已经完全展开:沙耶香成为了魔法少女,她的战斗服象征着一位英勇的骑士,誓言要拯救恭介、小圆以及世界免受魔女的侵害。不幸的是,与她为敌的是她那完美的对手:经验丰富对天真幼稚;红色对蓝色;自我中心且贪得无厌对自我牺牲保护他人。至少表面上是这样,但正如我们将看到的,这部剧中许多对立的双方并没有那么大的不同。
