%Completed
\chapter[常存质疑性(奇迹、魔法,都是存在的)]{常存质疑性\protect\footnotemark(奇迹、魔法,都是存在的)}

\footnotetext{译注:``I think theatre should \textbf{always be somewhat suspect}.''“戏剧应常存质疑性。”

\par
这句话揭示了一种深刻的艺术理念:优秀的作品不仅为娱乐观众,更要引发思考,甚至促使观众质疑现实与自身。《魔圆》正是如此。它表面是典型的魔法少女动画,但随着剧情推进,逐步解构这一类型,揭示出人性中更为复杂幽暗的层面,如失落、抑郁、绝望与牺牲。
\par
从第四集起,观众得以触及该剧的核心。动画不再呈现充满希望与魔法的童话世界,转而成为一场深刻的心理与情感考验。虚渊玄的故事迫使我们质疑魔法少女角色的本质:她们的牺牲是否值得?其力量带来的是幸福,还是更深重的痛苦?这正是作品“质疑性”的体现——它扭曲并颠覆观众对魔法少女题材的固有期待,引发对善恶、救赎及牺牲本质的深层叩问。通过这种方式,本作确实践行了“常存质疑性”的艺术理念。}

《魔法少女小圆》第四集开启了该剧的第二个篇章。然而,此篇章的核心焦点——成为魔法少女的沙耶香与新角色佐仓杏子——直至集末才登场。本集主体实则是承上启下,既充当了第一篇章的尾声,又同时为第二篇章埋下引线。

此外,本集可以说才是系列“真正的”第一话。相较于前三集对传统魔法少女风格的表面效仿,本集首次鲜明地呈现了作品主导的美学风格。因此,它亦可视为全剧的正式引介,因为它首度揭示了贯穿系列的核心主题之一:抑郁。

在论及《魔圆》时,经常会引用编剧虚淵玄的一段言论(尽管原文出自其早期作品《Fate/Zero》第一卷后记)。他写道:“事实上,我并非向来如此。我也曾创作过结局未必完满,但主角至终章仍秉持‘纵使前路艰险,亦须坚守初心’的作品。然而不知何时起,我再难创作此类故事。”\cite{ref29} 他继而又论及熵增与失败的必然性,言辞间流露出某种抑郁,或至少是消极的迹象。

诚然,依据作品揣测作者的精神状态或个人观点风险重重。“隐含作者\footnote{译注:由美国文学理论家韦恩·布斯在《小说修辞学》中提出,指文本所呈现、由读者建构的作者形象,区别于真实存在的血肉之躯。故即便真实作者否认,文学批评仍可将文本蕴含的价值观、要素与手法归于其隐含作者。}”必然有别于实际作者\cite{ref30}。后者作为真实个体(迥异于角色或隐含作者这类虚构存在),其主观内心世界全然不可知,无法由外感知或推断,尽管我们仍需假定其存在。换言之,虚淵玄本人或许生活顺遂,创作抑郁角色与绝望情境,纯粹出于对此类题材的偏好与驾驭能力。真相如何,我们无从知晓。

尽管如此,《魔法少女小圆》解读为一部关于抑郁的故事依然顺理成章。剧中的魔法少女均呈现不同的抑郁征象:麻美的孤独、焰的情感淡漠,以及沙耶香贯穿始终的失落与层层累积的绝望,最终走向了自杀的结局。


作为系列的首个“真正”剧集,抑郁与失落弥漫其间。前半部分着重刻画小圆如何应对失去麻美的创痛。麻美逝后翌晨,早餐蛋黄的色泽令她忆起麻美的金发,滋味更是催人泪下。此情此景,令人联想到另一部备受喜爱的魔法少女真人剧中的经典台词\footnote{译注:出自《吸血鬼猎人巴菲》第五季第16集(约26分53秒至27分30秒)。主人公巴菲面对自身能力限制,而无法挽救母亲离世所说的台词。}:

“但我不明白!我不明白为什么事情会发生,也不明白要怎么挺过去,但她现在只是……,只剩躯壳了。我不明白她为什么不能回到身体里然后活过来,不要死去。这很愚蠢,又极端又愚蠢。……我喝着果汁时在想:她再也无法喝到果汁了,也不能吃煎蛋,打哈欠,或是梳头发了,再也不会了。但是没有人跟我解释为什么。”\cite{ref31}

然而,本集中的失落远不止于麻美之死与小圆的哀恸。恭介丧失演奏能力后,只能聆听音乐,其痛苦与愤怒同样撼动人心。得知医生断言其手腕残疾已成定局,沙耶香毅然决定成为魔法少女以拯救他,即便深知其中凶险。尽管后续剧情会揭示她更复杂的动机,但此刻的她无疑是无私而英勇的。不幸的是,恭介自残手臂时的麻木感已成预兆:沙耶香很快也将堕入一种麻木的绝望,这种状态屏蔽痛苦的同时,也隔绝了一切其他感受。此外,她选择成为麻美的继任者——即在未及处理自身悲伤与失落前,便贸然投身于拯救他人——也预示了她们同样的行为模式。

当然,还有那个自杀团体。仁美在剧中戏份寥寥,其加入显得十分突兀,但剧情暗示是魔女将脆弱个体聚拢成此自杀团体。究竟是何使仁美变得如此脆弱?剧中未明言,如同第二集中那位无名女子的遭遇。但仍有迹可循:此前仁美被刻画为极度忙碌,不仅是优等生必备的备考补习,还有日本上流阶层的文化修习,如传统舞蹈或茶道。此集后这些课程再无提及,或暗示她在事件后减轻了负担。另一可能是关乎她对恭介的暗恋情愫以及羞怯心理——她在本集后不久向沙耶香发出最后通牒前,此情甚至不为密友所知,之后她便立即主动接近了恭介。

让我们把更多焦点集中在小圆的脆弱性上,魔女利用她未能拯救麻美的自责来发动攻击。结合自杀团体中那位自述经营工厂失败的男人,以及仁美可能的脆弱点,此魔女的作案模式似乎是针对目标对象的失败感与无能感。

至少对小圆,魔女选择的攻击手段是:通过电视屏幕播放上一集的片段,迫使她直面自己的无能。此时小圆的形象开始模糊扭曲,标志角色与背景区分的轮廓线逐渐消融。换言之,魔女向小圆揭示了她身为剧中角色的本质,剥夺了她的身份。她不再是人,而是一件物品,是动画中的一个元素,因此开始消解,融入动画的背景之中。

然而,一个有趣的事实是:纵然角色是虚构的,她们所激发的情感却是真实的。即便知晓小圆是虚构的,我们依然会关心她,并为沙耶香的救援感到欣慰。至少此刻,这足以让她在剧中“现实”里恢复完整。在动画的领域内,小圆仍是一个独立的实体,而非弥散的概念。尽管如此,这集“真正”的《魔圆》已为其结局埋下了伏笔:小圆终将化作星辰,散于天际。

至本集尾声,第二篇章已全然展开:沙耶香成为魔法少女,其战斗服宛如英勇的守护骑士,她誓言要拯救恭介、小圆以及世界于魔女之害。不幸的是,与她针锋相对的是其完美的对立面——经验老道对应天真懵懂,炽烈红色对应沉静蓝色,自我中心的贪婪对应自我牺牲的守护。至少表面如此;但如我们所见,剧中众多对立组合,她们终究并非那般不同。
