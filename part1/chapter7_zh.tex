\chapter[失败的唯一原因(你能面对真正的心情吗?)]{失败的唯一原因\protect\footnotemark(你能面对真正的心情吗?)}
\footnotetext{译注:``The only lost cause is one we give up on before we enter the struggle.''“失败的唯一原因,是我们在斗争伊始前便已放弃。”
\par
“失败的唯一原因”并非外敌,而是我们内心的自我放弃。沙耶香深陷绝望与自我物化的泥潭,正是因为她选择了视己为“僵尸”,放弃了人之为人的主体性,不再试图挣脱困境。在她眼中,未来已被烙上无可逆转的悲剧印记。面对恭介的冷落与身体的异变,她放弃了“斗争”,放弃了反思自我、质疑现状与追寻解脱的可能,将自己彻底抛向无望的深渊。Havel 的箴言,恰恰点明了在困境与挑战面前,永不放弃、持续斗争的价值,这也与沙耶香本集中的精神状态及其对自我物化的屈从形成了鲜明对照。}

上集揭示了丘比如何通过封锁信息来剥夺魔法少女的主体性。本集开篇延续此主题,丘比继续向观众与沙耶香展示其彻底的冷漠。为佐证“灵魂抽离肉体有益”的论调,它利用沙耶香的灵魂宝石来施虐,以此证明:若无宝石缓冲,她与杏子战斗中承受的第一波攻击便足以令其彻底瘫痪。然而,丘比对她的苦痛漠不关心,仅将其视作向观众演示的教具,这无疑暴露了其所谓“为魔法少女着想”的虚伪。

“僵尸”一词在本集反复出现(该词源自剧中原话\footnote{译注:日文“ゾンビ”即英文“zombie”的发音。},沙耶香在对话中多次清晰提及),其意涵耐人寻味。僵尸的本质是形似活人,实为死物。哲学意义上的“僵尸”,指行为类人却无内在体验或生命的造物——例如,戳刺僵尸会令其喊“痛”,但它并无真实的痛感\cite{ref39}。更为人熟知的则是影视中行走的尸骸,它们虽早已死去,却能在战斗中被我们“击倒”或“杀死”。这为观众提供了一种幻想与他人搏杀的快感,却无需背负戕害不辜的道德重负\cite{ref40}。换言之,无论哲学僵尸还是影视僵尸,皆是极端物化的典范。在此类情境中,人的主体性被剥夺殆尽,沦为可被任意使用与践踏的物件。有论者指出,僵尸传说本身便根植于个体遭受不公利用、非人化对待或被压倒性外力扼杀的感受\footnote{后殖民理论家 Frantz Fanon 认为,僵尸是殖民压迫下原住民群体对去人性化与禁锢性压迫的表达。}\cite{ref41}。这无疑是沙耶香当下处境及其所受丘比影响的贴切写照。

物化的阴霾笼罩本集始终。丘比自诩洞悉沙耶香所需,却令其深陷苦痛;杏子亦复如是。在杏子回忆成为魔法少女的经过中,她的家人与周遭世界皆以玩偶、傀儡、玩具的形象示人——宛如僵尸般徒具人形,毫无主体意志可言。杏子曾许下她自以为父亲所渴求的愿望,却因未曾与之沟通而铸成大错。这愿望非但未能惠及父亲,反将他连同整个家庭,乃至杏子自身,一同拖入了绝望的深渊。她未曾敞开心扉与父亲对话,仅凭臆测断定他的心意,实则将他视作观察的客体,而非能言说自身欲求的主体。诚然,日本文化中含蓄间接的表达根深蒂固 \cite{ref42},然则这份不坦率终酿成弥天误解,至今仍在看似无动于衷的杏子心中回荡不散。

然而,杏子从此经历中汲取的教训全然错误。她并未尝试与沙耶香沟通理解,反而武断地将彼此等同,认为沙耶香的症结在于未能洞察恭介的心意。杏子彻底摒弃共情,拒斥所有人际联结与社会规范,沉溺于食物慰藉的纯粹物欲。这对沙耶香而言无法接受,因为她的问题核心并非物化他人。

在沙耶香与住院的恭介多次对话中,我们清楚听到他坦言渴望手臂康复,而无法愈合正是他绝望的根源。沙耶香并未臆测恭介所想,她亲耳听其倾诉。相反,她的谬误在于(如麻美在第二集所暗示的)未能厘清自己真正的渴求——她确实希望恭介康复快乐,但更深层的愿望是与恭介相守。换言之,沙耶香的错误在于过度自我牺牲,反而将自己物化了。

恭介那自我中心的冷漠,对沙耶香的精神状态更是雪上加霜。她频繁探病,他却从未思及缘由,只视作理所当然,也从不考虑应告知一声自己出院或返校的日期。他对沙耶香的内心世界与动机毫无兴趣,亦不体察其感受,只将其视为自己生活中的配角,而非她自身生命的主角——恰如他在剧中本就是个配角。

吹散本集物化迷雾的良方,在于角色们将彼此视作拥有独特主观经验的自主个体。关键在于坦诚沟通,不幸的是,本集中的沟通极度匮乏。角色们大多自说自话,浑然不觉言辞的杀伤力。唯一例外是小圆与沙耶香的对话:崩溃的沙耶香泪流满面地哭诉,因身体的异变,她已然无法与恭介建立浪漫关系了。

触发这场对话的,恰恰是一个关乎理解与尊重他人主体性的难题。仁美在接近沙耶香时已恪尽礼数;她本无义务推迟向恭介告白或给予沙耶香优先权,但她依然选择如此,只因体谅沙耶香的心情,不愿伤害她。不幸的是,她无从知晓沙耶香正经历的其他磨难——沙耶香也显然不会告知——因此,她全然不知自己对沙耶香的坦白,正刺痛后者因身体状态而生的新创。仁美无法感知自己如何伤害了沙耶香,同样不解沙耶香为何不向恭介表露心迹;她自然没有理由相信沙耶香会反对她与恭介交往。

耐人寻味的是,最令沙耶香痛苦的并非失恋,而是对话中那转瞬即逝的悔意——后悔当初从魔女手中救下仁美。沙耶香为自己设定了过高的正义标准,无法接纳人性中刹那的恶意本就是内心的一部分,这些念头未必外化为行径。反而,她将这瞬间的恶念当作自己确已沦为“僵尸”的铁证,认定自己不再是人,而是一具构造异常的躯壳。

最终,沙耶香自我物化的倾向走向了极致:她刻意麻痹感知,令自己在与魔女战斗时无知无觉,仅靠治愈魔法修复伤痕。她的自我价值感已跌至谷底,不再在意自我保护,亦拒绝接受他人援手。在她看来,自己仅是件物品,那失去恭介理所应当,只是时间问题罢了。此刻,绝望与深沉的抑郁构成了沙耶香眼中唯一的未来,与魔女战斗则成了她仅存的意义。

至此,《魔圆》中章高潮的舞台已然搭就。如同首章以挣脱传统魔法少女题材的桎梏作结,这第二章终将以该题材的消亡落幕。
