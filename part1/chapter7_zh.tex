\chapter[失败的唯一原因(你能面对真正的心情吗?)]{失败的唯一原因\protect\footnotemark(你能面对真正的心情吗?)}
\footnotetext{译注:``The only lost cause is one we give up on before we enter the struggle.''“唯一的失败原因就是我们在未曾斗争之前就放弃了。”
\par
真正“失败的原因”并不是外在敌人,而是我们自己内心的放弃。沙耶香的最大困境是深陷绝望和自我物化的泥潭。沙耶香的绝望正是源自于她的自我放弃,选择了视自己为“僵尸”,失去了作为人的自主和能动性,没有试图挣脱这种状态。在她眼中,未来已经被定义为不可逆转的悲剧。面对失去恭介和身体的变化,她放弃了继续“斗争”,放弃了反思自己、质疑现状和追求自我解脱的可能性,将自己投向了无望的未来中。Havel 的这句话正是强调了在面对困境和挑战时,不放弃、持续斗争的价值,而这一点与沙耶香在本集中的精神状态及她对自我物化的接受形成鲜明对比。}

在上一集,我们看到了丘比如何通过限制魔法少女获取信息来剥夺她们的自主权。本集开篇延续了这一主题,丘比继续向观众和沙耶香展示它的漠不关心。为了证明将魔法少女的灵魂移里身体是对她们有益的论据,它利用灵魂宝石来折磨沙耶香。如果没有宝石的缓冲,她在与杏子战斗中受到的第一波创伤就会让她完全失去行动能力。然而,丘比对她痛苦的漠视,完全将其当作向观众展示的教学工具,暴露了它所谓的“为魔法少女着想”言论的虚伪。

“僵尸”一词的反复使用(这词来自剧中原话\footnote{译著:日文中“ゾンビ”正是``zombie''的发音。},本集可以明显听到沙耶香几次在对话中使用这个词)在这里很有说法。僵尸本质上看起来像活人,实际却是死物。哲学上讲,所谓“僵尸”是一种行为像人却没有内在体验或生命的物——比如戳一下僵尸,它会喊“哎呀”,但它并没有痛感\cite{ref39}。更为熟悉的是电影中,行尸走肉的僵尸。虽然它们实际上早已死去,但在战斗中可以被我们击倒或杀死。这给观众带来了想象与其他人战斗和杀戮的刺激,而不必担心让一个真人死去的道德问题\cite{ref40}。换句话说,无论是哲学上的僵尸,还是电影上的僵尸都是极端物化的案例。在这些案例中,一个人的自主性被剥夺,只剩下可以任意使用和虐待的物体。可以说,僵尸的传说甚至源自于人们被不公正地利用、被非人性化对待或是被压倒性的外部力量所扼杀的感受\footnote{后殖民主义理论家 Frantz Fanon 认为,僵尸是殖民压迫下,原住民群体对去人性化和禁锢性压迫的表达。}\cite{ref41}。这无疑是对沙耶香目前状态和丘比对她的影响的合理描述。

本集充斥着物化。类似丘比的,自认为这是对沙耶香好却造成了她极度痛苦,还有杏子。在她讲述自己成为魔法少女的故事中,杏子家庭和周围人都以娃娃、玩偶的方式出现,就像僵尸一样,都是没有自主性的人形物体。杏子许下了一个她自认为是为了父亲的愿望,但因为没有与父亲沟通,她搞砸了,最终不仅给父亲带来了痛苦,还让整个家庭乃至她自己也陷入了困境。由于没有与父亲直接交流,而只是单纯地假设自己知道他想要什么,杏子将父亲视为她观察的对象,而非一个能够表达自己需求和愿望的主体。尽管日本文化中间接委婉交流很普遍\cite{ref42},但这种不坦率最终造成了严重误解,至今依然困扰着似乎无动于衷的杏子。

然而,杏子从这段经历中得到的教训也是完全错误。她没有与沙耶香沟通尝试理解她,反而假设自己和沙耶香是一样的,认为沙耶香的问题是她没有了解恭介的真实想法。杏子完全放弃了同理心,拒绝了所有人际交往和社会规范,专注于以食物为慰藉的纯粹物质需求。这对沙耶香来说是无法接受的,因为它的问题并非是将他人物化。

在沙耶香与恭介住院期间的多次对话中,我们看到他明确表示希望自己的手臂能够恢复,而无法愈合给他带来了极大的绝望。沙耶香并没有自以为自己知道恭介想要什么,恭介直接亲口告诉她了。相反,沙耶香的错误在于(正如麻美在第二集所暗示的)没有搞明白自己真正想要什么。她确实想要一个恢复健康、快乐的恭介,但更是为了能够与他在一起。沙耶香的错误换句话说,就是过于自我牺牲,反而将自己物化了。

恭介有些自我中心的冷漠对沙耶香也是没有一点帮助。沙耶香去看望他时,从不思考她为何如此频繁地来看望自己,他只认为这是理所当然的,因此从未停下来思考沙耶香是否希望知道他何时离开医院,或者他何时会回到学校。他完全不关心她的内心世界和动机,也不考虑她的感受。他把沙耶香当成自己生活中的配角,而不是她自己生活的主角,就像他自己在剧中也仅是一个配角。

解决本集物化现象的关键是让角色们把彼此当作拥有独特主观经验的自主个体来看待。关键在于坦诚的沟通,而不幸的是,本集中的沟通极为匮乏。大部分角色仅仅在自说自话,完全没有意识到自己言辞所带来的影响。唯一例外是小圆与沙耶香的对话,沙耶香在崩溃之际,泪流满面地哭诉,因为她的身体状态已经发生了改变,无法和恭介浪漫地交往了。

这一对话的触发点是一个关于理解和尊重他人主观性的难题。仁美在接近沙耶香时做了一切正确的事情;她没有必要推迟向恭介告白的时间,也不需要给沙耶香抢先的机会,但她还是这样做了,因为她理解沙耶香的感受,不想伤害她。不幸的是,由于她不了解沙耶香正在经历着其他问题,而沙耶香显然也不会告诉她。她并不知道自己向沙耶香坦白喜欢恭介的心情,实际上触发了沙耶香对自己身体状态问题的痛苦,正在伤害沙耶香,也不知道沙耶香为何不将自己的心情告知恭介;因此,仁美没有理由相信沙耶香会反对她与恭介约会。

有趣的是,沙耶香感到最痛苦的并不是失恋,而是那一瞬间,她开始后悔当初从魔女手中救了仁美。沙耶香对自己设定了过高的标准,从而无法认识到瞬间的恶意想法也是人内心的一部分,比较这些想法不一定会表现为外在行为。沙耶香反而把这一瞬间的恶念当作自己已经变成“僵尸”的证明,认为自己已不再是一个人,而是一个有着不寻常身体构造的东西。

最终,沙耶香物化自我的倾向在故意麻木自己中达到了顶点,让她在与魔女战斗时感受不到一点痛苦,依赖治愈魔法修复着她所遭受的伤害。她对自我价值的认知已跌至谷底,已经不再关心自我保护,也不再愿意接受他人的帮助。在她眼中,已将自己视为一物,自己理所当然得注定会失去恭介,只是时间问题罢了。此时,绝望和深沉的抑郁成为沙耶香所能看到的唯一未来,而与魔女战斗也成为她剩下的唯一目的了。

这一切为《魔圆》中章的高潮奠定了基础。正如第一章以跳脱传统魔法少女题材束缚结束,这第二章将以该题材的死亡而告终。
