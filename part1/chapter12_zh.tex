\chapter[滋养人类希望的东西(我最好的朋友)]{滋养人类希望的东西\protect\footnotemark(我最好的朋友)}

\footnotetext{``Isn't it the moment of most profound doubt that gives birth to new certainties? Perhaps hopelessness is the very soil that nourishes human hope; perhaps one could never find sense in life without first experiencing its absurdity.''
\par
“难道正是最深刻的怀疑时刻孕育了新的确定性吗?或许,绝望恰是滋养人类希望的土壤;或许,没有先经历生命的荒谬,就永远找不到其中的意义。” 
\par
小圆并非自愿牺牲,正是她在荒诞与绝望中看到了人类存在的可能性,通过超越个人欲望、抛弃自我,最终达到了启示与超越的境界。小圆的选择象征着一种新的理解:通过自我消亡与超越,生命的意义才得以真正显现。Havel 的思想提供了一个哲学框架,帮助读者理解为何小圆的牺牲不仅仅是悲剧,而是一种超越困境的希望种子,“绝望恰是滋养人类希望的土壤。”
\par
同时,焰也经历了类似的成长和转变。她从最初的绝望中获得了继续战斗的动力,即使一切看似无望,依然存在希望的火花。焰并没有放弃,而是通过对小圆无私的爱和责任感,继续在新的世界中寻找意义,这也是一种通过绝望发现新希望的表现。
\par
这种关于希望与绝望辩证关系的哲学反映了作品的核心主题:希望并非永远是天真或直接的,它往往是在经历了巨大痛苦和怀疑之后,才以一种更深刻的形式得以诞生。}

在《魔圆》的最后一集里,失败即胜利,既是衰落也是凯旋。

故事开始时,我们看到与上一集结束时相同的画面。四个身影依然存在,她们是这场末日中的关键人物,因为这的确是末日:我们看到的每一个时间线都以与瓦尔普吉斯之夜的战斗告终。在与她战斗后没有任何未来,因为焰会在未来到来前重置宇宙。焰是第一个身影,破碎而流血,像个悲伤的小丑,她无休止地牺牲自我在一个荒诞又冷漠的宇宙中绝望地寻找意义。她被她的失败工具——瓦尔普吉斯之夜,那个象征荒诞的丑角哈利昆嘲笑着。两者之间是丘比,一个导演、作家、大师级的操纵者,它编排着她们的舞蹈来取悦自己看不见的观众,从她们的情感变化中获取力量与养分。

但接下来是小圆。她曾一直处于被动的状态,是其他人争夺的奖品,但现在她终于做出了自己的选择:死亡。她将成为死亡,世界的毁灭者,在每个魔女诞生的瞬间将她们消灭,直到最终只剩下自己,而她也将结束自己的生命。

但小圆并不是沙耶香。这不是自杀,这是超越性的死亡,是自我消失的死亡,通往永恒与统一的途径\cite{ref62}。小圆不再是小圆;她是一个没有开始与结束的存在,在她之中,万物都是一体的。

但首先,是一块承诺已久的蛋糕。毕竟,这是曾经她与麻美间的约定:如果小圆找不到任何想许的愿望,就许愿要一个大蛋糕吧。但小圆刚刚许下了她的愿望,那么为什么她还要和麻美一起吃蛋糕?

因为小圆解决了这个悖论。为了获得启示,为了摆脱魔法少女所陷入的希望与绝望的因果循环,必须舍弃一切欲望。但如果舍弃了所有欲望,包括超越的欲望,那为什么还有人会去超越呢?小圆找到了答案:消亡的自我,消除自我与他者的区别,正是消除欲望,因为欲望的主语与欲望的对象已经合为一体。“逢佛杀佛。”\footnote{出自唐代一位有影响力的僧侣,临济义玄。}\cite{ref71}尽管她刚刚许下愿望(在她现在等于遍布时间中的存在下,“刚刚”这一概念已经没有实际意义),她不再有任何愿望,所以她同时得到了启示与蛋糕。

然而,这一场景中不止是蛋糕。麻美,一个传统魔法少女的象征,把小圆的服装设计笔记本还给了她。麻美死后,小圆不再谈论成为魔法少女来寻求目的;反而,它变成了她不断考虑为他人牺牲的事。通过归还笔记本,象征着麻美将守护魔法少女传统的角色交给了小圆,同时恢复了这样一种观念:魔法少女不仅仅是一种可悲的牺牲,也是一种令人鼓舞的使命。

因为,小圆绝对不是殉道者。她不是基督的化身,不是通过痛苦与死亡来吸收他人罪孽的存在;她明确地摧毁了自己身上承载着痛苦的魔女的那一面。她无我、超脱世俗,因此不再受苦。在这个浮士德式故事中,她类似是格雷琴(89),更像是圣母玛丽亚的象征,纯洁无暇,代表着为他人争取恩典的存在。即便如此,这个形象依然不完全准确,因为小圆并没有为魔法少女们代求或为她们祈求怜悯。她们依然会变成魔女并死去;唯一改变是她们的魔女形态不再出现在这个世界,因为小圆在她们诞生的瞬间就将其抹去。她们依然会受苦、绝望、死去——因为这就是活在这个世界的本质。小圆的角色是引路人,是一个引领者,在她们对世界构成威胁之前,引领魔法少女们走出这个世界。在她的净土\footnote{净土宗佛教大致上的教义是,阿弥陀佛创造了一个“净土”,信徒死后将升往此地,在那里更容易达到涅槃。}\cite{ref73}中,她们学习,或许有一天会如她一样超越。与此同时,地球上的一切依旧不完美,但确实变得比以前更好。

“大丈夫\footnote{译注:日语,意为没有问题,没关系。}。”这是小圆在最终以圆神(尽管圆菩萨\footnote{观音菩萨是一个重要的女菩萨。}\cite{ref73}更为合适,因为她更像是菩萨而非神)的形象出现前对焰说的那句话。这是在该剧中有很强象征意义的一句话语;在《魔卡少女樱》中,它是女主角在系列结束时施下的最终咒语,表达了一种近乎无限力量的希望\footnote{《魔卡少女樱》第70话:真正的心意}。小圆已经在履行她作为守护者与魔法少女们过去的职责。然而,她的变身动画简短又非性化,并强烈暗示她的服装由安东尼(在第一集中,魔女迷宫中占主导地位的使魔)制成,这是该剧首次引入的奇异且野性的美学风格。

小圆成为了连接旧魔法少女与新魔法少女间的桥梁。她用旧类型的肯定话语向新类型的代表人物焰发出安慰。她还将由被一再与麻美做类比的询子选中的缎带交给了焰。这不是对魔法少女传统的完全恢复,新世界仍然充满黑暗,成为魔法少女依旧充满危险并且很可能以死亡告终,这是部分的恢复,承认在过去魔法少女作品中有着优秀的故事、好的角色,以及真切的主题。

其中最重要的主题就是希望,天真的希望,乐观地认为事情会变得更好,这是确实一种牢笼。任何坐等救世主或幸运降临的人注定会失望。宇宙的熵性本质是,如果事情可以变得更糟,它们就会变得更糟,且总是有更糟的时候。但是,存在另一种形式的希望,这是焰在结尾所拥抱的希望。如果事情能够且必然会变得更糟,那么这就意味着此刻,宇宙并未达到最坏的状态;此时,世界中一定有些好东西存在。那种美好是可以被寻找的,可以被争取并暂时保存的。在局部范围内,熵是可以被逆转的。

小圆已经获得了启示,脱离了这个腐朽的世界。但她并没有抛弃它;她所创造的世界并非完美,但确实变得更好。因为完美的世界是没有故事的。魔法少女们依然不可避免地会死去,但每个人都会死;重要的是,他们现在更明白这个系统是如何运作的,并且与彼此和孵化者之间有了更好的关系——尤其是那些魔兽们会掉落的小魔法恢复品,而不是一个大的,鼓励魔法少女们合作。团队将成为这个新世界的常态,而不是旧世界里孤独的魔法少女们。由于孵化者无法再从魔女的绝望中汲取能量,他们也就没有动机让魔法少女们受苦或对她们隐瞒系统的运作原理。

连询子也在新的环境中出现。我们之前见过她坚定且决心的模样,关心且体贴,但这最后一幕是我们第一次看到她展现出真正的快乐。我们可以推测这一场景暗示询子在新时间线中是个完全不同的人,少了过去的坚持与多了怀旧感,但我们没有足够的理由相信这一点。看起来小圆不太可能用另一个女人取代她的母亲,而更可能的是,这是她和家人一起外出时的放松状态。她和小圆父亲之间的关系保持不变——他照顾达也,而询子则处理外部事务——所以很可能她仍然是家庭的主要收入者,是一个充满动力的高管。只是,现在我们看到了她也包含了怀旧、宁静和对过往的留恋;她同样是复杂的,就像魔法少女与魔女们包含着诅咒与祝福一样,这个结局也既是喜悦也是悲伤,是胜利也是失败。

看到询子和达也帮助焰明白了即使在对她来说是黑暗且没有小圆的世界里,也仍然有值得珍惜的美好事物。她继续前行,在爱、责任和 Havel 式的希望中得到确认。无论她是否能成功,她的生活只要继续奋斗就能找到意义。虽然她未能拯救小圆,但在失败中,她成功地赋予了小圆拯救自己的力量。小圆通过自我牺牲拯救了自己,而焰失去了她,但总有一天,当焰耗尽最后的能量,在作为魔法少女的最后一战中失败时,她将再次与小圆团聚。

但这不仅仅是焰一个人的事。另有一人正在努力,试图抵抗衰落,但愈发担心自己的努力注定要失败。正如虚淵玄在他的后记中所写:“为了给故事写出一个完美的结局,你必须扭曲因果律,颠倒黑白,甚至拥有逆转宇宙法则的力量。”\cite{ref29}那个注脚的隐含作者,和本剧的创作者,正是一个深陷绝望的个体,正陷入由绝望引发的创作深渊。

“只有一个能为人类歌唱赞美之歌的、天真纯洁的灵魂,才能拯救这个故事。”而现在,在小圆之中,一切事物已经合而为一。这是一个虚构的,在创作者心中孕育出来的作品(即使是合作创作所隐含的整体创作者);那位创作者就是那个一体的存在。小圆曾经爱过这个世界中的某个东西,认为它值得拯救,她也是那位创作者的一部分。焰将接受这种爱,作为继续前行和努力的理由,她也是那位创作者的一部分。正如焰在新世界中并没有突然变得满面笑容一样,这并不是一个万能灵药,但足够让她再坚持一段时间。

——不曾忘却。世界的某处,那人一直为你而战。

——只要将她存在心中,你就并不孤单。

放映机缓缓停止了转动。
