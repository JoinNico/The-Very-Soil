\chapter[也许绝望是(最后残留的路标)]{也许绝望是\protect\footnotemark(最后残留的路标)}

\footnotetext{``Isn't it the moment of most profound doubt that gives birth to new certainties? Perhaps hopelessness is the very soil that nourishes human hope; perhaps one could never find sense in life without first experiencing its absurdity.''“或许,绝望恰是滋养人类希望的土壤;或许,没有先经历生命的荒谬,就永远找不到其中的意义。”“难道正是最深刻的怀疑时刻孕育了新的确定性吗?也许绝望恰恰是滋养人类希望的土壤;也许只有先经历生命的荒谬,人们才会找到其中的意义。”  
\par
人类在最绝望、最不确定的时刻,往往能够找到重新振作、追求意义和希望的动力。《魔圆》中深刻探讨的正是这样的主题:焰通过反复重置时间、尝试保护小圆,虽然屡次失败,但她也最终成就了小圆的巨大牺牲与升华。在魔法少女们的绝望与苦难中,尽管看似一切都失去了希望,然而正是在这些最黑暗的土壤中,也有可能孕育出新的转机、成长和希望。这揭示了《魔圆》中的哲学层面:生活的荒谬感、无望感和痛苦其实是推动人物成长和希望产生的根源,也进一步深化了对生命的深刻反思和对人类不屈的礼赞。}

命运之轮\footnote{法轮、命运,是佛教的经典象征。}\cite{ref62}贯穿结局的始终。无论在哪,都可以看到命运之轮的影子。瓦尔普吉斯之夜那巨大转动的齿轮代表了希望与绝望的循环,这些循环把魔法少女(再)转化为魔女,焰所创造的时间回溯的无尽循环。它们都象征着命运之轮,正如焰用来穿越时间的圆形时钟盾牌。

在本集末尾,丘比解释到是焰促成了现在的小圆。每一次重生,小圆将前一个循环的因果带入下一个循环\footnote{一般来说,这就是印度教和佛教中轮回的概念:前世的身体或记忆痕迹并未保留,但其业力负担依然存在。}\cite{ref65}。换句话说,尽管小圆在不同的时间线中并不携带具体的记忆,只有些许模糊的印象,无疑这些印象源自她与那个记得过去时间线的唯一人物之间深厚的联系,并且尽管她在每个时间线中都会被重新创造,她依然保持着与前一个时间线世界的联系。而既然这些世界是为她而创造并毁灭的,那么她承载的便是整个世界的因果。

这些重担,最终连接到一个迅速走向毁灭的世界,只会导致痛苦\cite{ref62}。世界不断增加的熵反过来影响小圆,将她从一个经验不足、但外向和自信的魔法少女,转变为一个优柔寡断又弱小的配角,而其身边的朋友们一个个死去。

为什么会发生这种情况?因为焰试图保护她。她是一个基督徒(或者至少上过基督学校)。像杏子一样,她怀着这样一个信念:一个人有能力拯救另一个人。即存在某种可以拯救的东西,也存在一个可以拯救的地方。这是一个根本的二元论命题:这里是坏的,而那里是好的。但正如丘比通过它的因果平衡论所阐明的那样,我们其实处于一个佛教的世界观中。丘比和焰都未能理解的是,在佛教宇宙中,物质世界的衰落是一种幻象,因为所有的区别都是幻象\cite{ref63}。过去即是现在即是未来。衰落即是生命。魔法少女即是魔女。物质是非物质的,彼此与自我没有界限。焰无法拯救小圆,因为根本没有需要拯救的小圆,也没有什么东西可以拯救她。万物合一,而既然这是一个虚构的故事,那么这合一的存在就是故事本身,进而是那个塑造这个故事的内在实体,隐含作者。

我们不再能把焰和麻美作对比了,但我们有在早期剧集中与麻美不断平行的角色——询子,她在这集有两个非常重要的场景。在第一个场景中,她与小圆的英语老师讨论了沙耶香失踪以及这件事对小圆的影响。在这一场景中,她似乎更像沙耶香而非麻美:她自责自己无能为力只能袖手旁观,而是坚持做而非仅仅成为,镜头从老师的视角拍摄,她沐浴在蓝色光芒中,焦点集中在发夹上,甚至看起来像沙耶香。

在这段与老师交谈的过程中,询子头顶上方悬挂着 Michelangelo\footnote{译注:米开朗基罗(1475年3月6日—1564年2月18日),意大利文艺复兴时期杰出的雕塑家、建筑师、画家、哲学家和诗人,与列奥纳多·达芬奇和拉斐尔·圣齐奥并称“文艺复兴艺术三杰”,以人物“健美”著称,即使女性的身体也描画得肌肉健壮。}的《创造亚当》的复制品。值得注意的是,两个角色的位置使得询子与上帝对位,而老师与亚当对位;与此同时,老师一侧的红色光照和询子一侧的蓝色光照,使得神的红色披风(学者们曾将其比作子宫\cite{ref66}和大脑\cite{ref67})几乎不可见;取而代之的是亚当显得被红色的温暖包裹着。这一形象预示着询子在这一集中的第二个场景:在庇护所中,她意识到必须停止保护小圆,去信任她。这是一个悲剧性的场景,最终呈现的是可能是本剧最言简意赅的镜头,一个简单的画面:询子穿着妈妈款式的牛仔裤和毛衣,镜头聚焦在她的腹部和手上,将小圆母亲简化为一个子宫。那只手伸出,仿佛要抓住小圆,将她拉进来,然后又停了下来,我们回到询子的脸上。这一刻,母亲的形象没有被理想化(即没有被简化和物化);她压制了自己作为母亲的本能,主动选择放弃行动,让小圆承担她所拥有的生活的风险。我们感受到询子放手时的痛苦,但我们也感受到她对女儿深深的尊重和信任。

然而,焰一直在做相反的事情,追求基督教的救赎理想,其中“上位者保护下位者”,如同一个永恒的子宫。她试图阻止小圆做出的自我牺牲。如果询子对小圆表现出尊重和信任,那么焰对她又表现出了什么?

正如所有人中最没有同情心和情感的丘比所指出的那样(这再一次表明它有足够的智力产生同情心,只是没有情感上的共情,这是对反社会人格来说的一种不完全、不准确的一阶描述\footnote{特别注意ICD-10中关于反社会人格障碍的描述项一:“对他人感受的冷漠无情”。}\cite{ref68}),问题早已不再是小圆。焰自己曾将她保护小圆的经历称为“迷宫”,而她公寓内部看起来几乎和一个魔女迷宫没有什么不同。故事外,魔法少女源于魔女;而在剧内,魔法少女的末路是魔女。焰能够超越时间,在她身上,一切是一体的:过去、现在和未来成为了一个完整的轮回。换句话说,她既是魔法少女,也是魔女,困在一个时间迷宫中,那迷宫正是本剧的叙事本身。作为魔法少女/魔女,她既带来了愿望,也带来了诅咒;既保护小圆,又导致了她的毁灭。

但是,魔法少女变成魔女的转变,正是从希望到绝望的转变,从相信事情会好转,到意识到死亡和衰落是不可避免的。而焰从未经历过这种转变。她是一个弱小且体弱多病的孩子,因为绝望和责任感成为魔法少女,想要成为小圆的保护者,而非她的救世主。即使在她的愿望中,她也没有想象过自己会成功,只是会不断地尝试。她本质上是绝望的,因此免疫于绝望——像所有二元对立一样,希望与绝望是合一的。因此,最终打破焰的并非是绝望,而是她意识到自己正让事情变得更糟,这让她第一次开始怀疑她的道路。对她来说,从坚定到怀疑的转变,才是威胁着将她变为魔女的根本原因。

而嘲笑她的,是瓦尔普吉斯之夜,魔女的安息日\footnote{德国民间传说认为,在圣·瓦尔普吉斯节(5月1日)前夕,女巫和恶魔会在布罗肯山山顶上相会跳舞。}\cite{ref69},命运之轮,一位丑角哈利昆。这位戏剧人物,履行了第一集开场时升起的幕布承诺——同样,剧院每天晚上都会上演同样的剧,只有些微的变化,反映了命运之轮和焰的无尽循环。在古老的意大利即兴喜剧\footnotetext{译注:16世纪在意大利出现的一种戏剧,其特点是戴着面具的角色。其中有许多的定型角色,例如愚笨的老人、狡猾的仆人、看似充满勇气的军官等。}中,哈利昆是一个恶作剧人物,嘲笑所有的权威和秩序,尤其是对注定悲剧的浪漫故事\cite{ref70}。像哈利昆一样,瓦尔普吉斯之夜将永远与焰共舞,嘲笑她,夺走小圆,她的爱,挑战规则、束缚,是生命中不可预测和混乱的象征。面对一个恶作剧者,没有高低之分,只有人类。规则被打破,体系崩塌。

而就在此时,小圆登场并许下了她的愿望。
