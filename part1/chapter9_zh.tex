\chapter[时间酿出深远意义(这种事情、我不允许)]{时间酿出深远意义\protect\footnotemark(这种事情、我不允许)}

\footnotetext{``Even a purely moral act that has no hope of any immediate and visible political effect can gradually and indirectly, over time, gain in political significance.''“即使一个纯粹的道德行动,眼下并不显现任何直接或明显的政治效果,但在长期的历史进程中,它仍可能酝酿出深远的政治意义。”

标题借用了哈维尔的话,恰恰是在强调一种逐步积累、间接产生意义的过程。这种过程反映在魔法少女这个类型的演变上,尤其是在与“魔女”这一形象的关系上。文章讨论了魔法少女(如沙耶香)与魔女形象之间的联系,魔法少女看似只是一个关于女孩子拯救世界的表面故事,但通过《魔法少女小圆》的深层叙事,作者揭示了这一类型如何在多个层面逐渐积累起政治和社会意义。
例如,魔法少女的力量通常被限制在幻想世界里,与现实中的社会问题、性别不平等等没有直接关联。通过表现魔法少女的力量如何与女性气质、社会角色期待交织,作品间接地批判了压迫女性的性别规范和社会期望。在这个意义上,魔法少女这个看似单纯的类型,逐渐获得了“政治意义”,成为对女性角色、性别权力和社会结构进行深刻反思的载体。
沙耶香转变为魔女,尤其是她如何挑战性别角色和社会期望,恰恰是一个在看似毫无希望的情况下,通过时间和反思积累政治意义的过程。她的转变不是因为外部的直接“政治”冲突,而是由于她对自我价值、责任感以及对“保护他人”这一社会角色的挑战,这些行为从表面看只是情感和道德上的选择,但最终在《魔法少女小圆》的宏大叙事中,它们逐渐揭示了一个深刻的社会政治问题:谁拥有权力、如何使用这种权力,女性如何在一个男性主导的社会体系中挣扎和反抗。
丘比作为一个“系统的象征”,操控了所有魔法少女的命运,他的行为虽看似没有直接的政治含义,但其背后对女性角色的压迫、对自我表达的束缚却隐含了巨大的政治意义。通过丘比和魔法少女们的互动,作品探讨了个人选择、道德行动如何在长期的社会文化结构中逐渐积累起对政治和社会秩序的挑战。这与哈维尔所说的“即便是纯粹的道德行为,随着时间推移,仍能获得政治意义”高度契合。
总结:
作者选用这个标题,是为了强调《魔法少女小圆》这类看似单纯的叙事形式如何随着时间的推移,逐渐积累出政治和社会层面的重大意义。这与哈维尔的观点相符,暗示着个体行为或看似无关紧要的道德选择,能够在历史的长河中激发更深层的政治反思和变革。在这篇分析中,魔法少女的变身、权力与性别角色的讨论,正是在这个逐渐积累的过程中,形成了对社会秩序和性别权力的深刻政治意义。}

魔法少女其实一直以来是魔女。

从外叙事的、历史的角度上讲,这显然是正确的。魔法少女这一类型的出现,直接源于《家有仙妻》这部剧在日本那一代女学生群体中大受欢迎\cite{ref48}。萨曼莎是典型的魔法少女,既符合传统女性美,又拥有巨大的力量,但这些力量被紧密地限制在一个狭小的范围内,一个焦虑男性气质的牢笼。她当然可以,也确实经常有,坚持自我个性,但最终她还是受困于那个时代电视剧规范的限制中,必须以完美的妆容、发型、可爱的裙子,以及社会认可的妻子和母亲角色来展现女性气质\cite{ref49}。

虽然在她之前也有类似的漫画,但第一位动画魔法少女也是一位魔女,她就是《魔法使莎莉》中的莎莉。她当然比《家有仙妻》中的萨曼莎要年轻得多,也并未非妻子或母亲的角色,而是表现出一种孩子气的女生气质,尽管拥有强大的魔力,她甜美可爱,但从根本上来说,她是友好的\cite{ref50}。

这一题材不断发展,就这样到了《美少女战士》时,魔法少女的类型基本已经定型。魔法少女,像魔女一样,从异界获得力量,无论是由镜中世界的统治者赐予,还是不小心从一本神秘的书中意外释放,或者是因为前世是月亮王国的皇室而与生俱来。魔法少女像魔女一样,有她们的使魔,那些以动物形态出现有意识的生物。当然,也拥有强大的、各式各样的魔法。

但奇怪的是,魔法少女的力量总是用来对抗同来来自异界的敌人。实际上,敌人往往与她们的起源有着密切联系。通常来说,魔法少女并不会去与政府腐败、企业不当行为,甚至男性英雄常常对抗的街头犯罪作斗争。换句话说,她们的力量不仅来自幻想,而且只能对抗幻想。

这也是所有魔法少女都是魔女的另一个含义。魔女这一形象是女性力量的恐惧象征;在一个男性气质与霸权和支配力挂钩的世界中,女性气质必须等同于无能、顺从或克制\cite{ref51}。在日本文化中,这一思想的传统表达是良妻贤母这一形象,指的是一个为家庭牺牲的理想女性,像《魔圆》中的仁美那样。拥有极强社会智慧的良妻贤母精通许多技艺,虽然在家庭领域内占据主导地位,但这些全都为家庭成员的娱乐或支持而服务,尤其是支持丈夫并抚养孩子。而她的丈夫,则是唯一一个在家庭之外可以行使权力的人\cite{ref52}。日本民间和流行文化充斥着那些“坏”女人的故事,她们超越了“她的位置”,从原本是狐妖的妻子\cite{ref53}到食人的山中老妖婆\cite{ref54},再到从情人身上汲取生命的雪女\footnote{然而,如果雪女选择的话,至少最后这一点可能是仁慈的。}\cite{ref55}。

当然,好女孩和坏女孩这一二元对立并不局限于日本文化。在西方民间传说和流行文化中,它通过等待被拯救的公主和威胁她和英雄的邪恶女巫(以及许多其他代表)得到了体现。拥有力量本身就意味着坏女孩、魔女,是对现状的威胁。\cite{ref56}

魔法少女的力量通常以两种方式被“净化”。首先,如前所述,她的力量不允许影响任何可能被观众识别为现实事物的部分,而是几乎无一例外地集中于与幻想敌人作斗争。第二,正如我们在第二章讨论的那样,她必须不断表演女性气质(记住,霸权男性气质的对立面正是表演性女性气质),穿着带花边或暴露的服装,摆出精心设计的姿势,当然,还有变身场景中的裸体舞蹈,所有这些都无时无刻提醒着任何可能感到威胁的男性观众,她仍然服从男性凝视,依旧是符合社会规范的“好女孩”。

所以意料之中,沙耶香变身为魔女的事件正好紧随她对一对性别歧视者使用魔法之后发生。她已经越出了“好女孩”的界限,并挑战了现状,因此她变成了坏女孩,邪恶的魔女。

但这一集以多种方式审视并最终颠覆了这一二元对立。最引人注目的是在杏子与沙耶香变成的人鱼魔女的战斗,我们看到蓝色和红色的血液漩涡形成了沙耶香和杏子的形象,之后漩涡交织汇聚成了一朵玫瑰,这与《少女革命》\cite{ref57}的开头极为相似。那部剧同样有位公主,玫瑰新娘(同样身边有一位挥着剑的假小子肩并肩,她拒绝成为传统女性角色,站在了保护者的位置上),她最终变成了魔女,从顺从的“好女孩”转变为强大且叛逆的“坏女孩”,摆脱了整个体系的束缚\cite{ref58}。这是一个普遍的说法,因为这些血液随后在画面中溅洒下来,镜头的构图看起来就像是从杏子的腿间流下。

在西方文化中,月经有时被看作是对女性天生“恶”的一种特殊惩罚\footnote{有关月经禁忌与女性主义的详细探讨,见参考文献\cite{ref59}},因为表演女性气质的本质就是坚持做自己,也就意味着“坏”,最终被贴上魔女的标签。因此所有女性都有“坏”的一面,也就是说,她们寻求表达自我。

但如果魔法少女与魔女本质上是,并且一直都是一体的,那么我们应该如何解读沙耶香的变身呢?幸运的是,这集给了我们答案:这是由丘比所施加的系统所导致的。魔法少女们的整个世界观都是由重要的唯一男性角色丘比强加给她们的,它对她们有绝对的霸权。它的论点是,她们已经同意参与他的系统,显然这是荒谬的,因为它故意隐瞒了关键信息;这里有着明显强奸文化的隐喻,男性霸权利用复杂且荒谬的标准来定义什么是同意,并通过操控这些定义来将责任归咎于受害者。

但请记住,丘比是隐含作者的象征,它并不是真正的虚渊玄,而是由编剧、导演、角色设计、动画师、声优、作曲等各方创作者共同努力形成的一个整体存在,是由整个行业的创作者所组成的集体意识。它在这一集中成功诱使杏子将沙耶香视作需要拯救的人,尽管它明知沙耶香无法被拯救。毕竟,如果她一直都是魔女,那又有什么需要拯救的呢?在成功操纵杏子接替本来先前由沙耶香尝试承担的守护者角色后,它又试图说服小圆做出类似的自我牺牲。

在二次元文化中,有一个词汇专门用来形容这种唤起保护欲的角色特征:萌[ 译注:此用语的起源目前众说纷纭、仍未可知,一般推测是在1990年前后开始普及。](萌え)。它源自日语“燃え”,意思是“燃烧”或“灼热”,后来演变为指对无助或处于危险中的他人所产生的一种强烈保护欲。2000年代末期,这种美学的表现形式在整个日本二次元文化,尤其是在魔法少女类作品中占据了主导地位,尽管斋藤环指出这一趋势最早可以追溯到90年代早期,尤其是御宅族对《美少女战士》\cite{ref19}中土萌萤的反应。根据这一美学的定义,角色的价值在于她们能否引发这种保护欲,包括无助、可爱、情感脆弱以及柔弱,同时具备传统和典型的女性吸引力\cite{ref60}。这一点,当然是延续了将萨曼莎置于她那温和丈夫的掌控之下,并迫使月野兔在获得力量之前必须脱光衣服的过程;它使角色变得无害,从而成为“好女孩”,不会对霸权男性气质的固有焦虑构成威胁。

丘比被揭露为一个剥夺女性权利的系统代表,迫使她们为它的利益表演,同时又将她们置于一个她们的痛苦被视作需要被保护的证明的位置,从而进一步剥夺了她们的力量。换句话说,它代表了性别角色本身。然而,他仅仅是一个更广泛系统中的一环,这个系统远远超出了它个体的范围。《魔圆》本身也深陷其中,尽管它可以批判为经济利益所作出的丑陋选择,但它无法完全摆脱这些选择。

尽管如此,这集依旧保持了惊人的一致性。沙耶香试图拯救他人、扮演守护者角色,却让她变成魔女。杏子和小圆试图拯救沙耶香,却因此丧命。丘比试图延续的宇宙,实际上却是一个毁灭的、悲惨的系统。正如我们在下一集中将看到的,焰试图拯救小圆的努力同样是注定失败的。

似乎,试图保护或拯救他人本身就意味着剥夺他们的力量。但与此同时,动画一再抨击丘比缺乏同情心,所以它不也可能是在支持客观主义,即 Ayn Rand\footnote{译注:爱丽丝·欧康纳(1905年2月2日—1982年3月6日),以笔名安·兰德为人所知,俄裔美国籍作家和哲学家。}的哲学,拒绝将利他主义视为一种道德行为\cite{ref61}。那么或许,帮助与拯救之间存在某种区别?或者,正如虚渊玄在Fate/zero后记中所写,是否只是因为我们无能为力,一切注定会变得更糟,正如所有系统、宇宙、社会和心理一样,都在迈向热寂\cite{ref29}?
