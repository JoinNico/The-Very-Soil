\chapter[一股秘溪潺潺流淌(似乎在梦里见过,那样...)]{一股秘溪潺潺流淌\protect\footnotemark(似乎在梦里见过,那样...)}
% 静水流深这个翻译也不错
\footnotetext{译注:``Life cannot be destroyed for good, neithercan history be brought entirely to a halt. \textbf{A secret streamlet trickles} on beneath the heavy lid of inertia and pseudo-events, slowly and inconspicuously undercutting it. It may be a long process, but one day it must happen: the lid will no longer hold and will start to crack. This is the moment when something once more begins visibly to happen, something truly new and uniquesomething truly historical, in the sense that history again demands to be heard.''
\par
“生命永远不会彻底毁灭,历史也无法完全停滞。在惰性与伪事件的沉重覆盖之下,一股秘溪潺潺流淌,缓慢而不显眼地侵蚀着这一切。这个过程或许漫长,但终有一日,覆盖层再难维系,开始崩裂。此时,某种真正崭新、独特且具有历史意义的事物将再次显现。历史正重新呼唤想被倾听。”
\par
% 《魔圆》本来的世界就像 Havel 所说的“惰性与伪事件的沉重覆盖”。然而,故事中的魔法少女们逐渐意识到,外表看似平静的世界下,潜藏着一股巨大的变革力量,不断侵蚀压抑的现实。小圆通过自我牺牲彻底打破原有结构,创造出一个全新秩序,象征着那层“覆盖”的破裂。她的行动根本性地改变了魔法少女的命运,推动了世界的演进。
}

《魔法少女小圆》的剧情可划分为三个相互交织又各有侧重的篇章。每个篇章聚焦于特定角色(或角色组合),并深入探讨一个核心主题。第一章涵盖前三集,重点着墨于麻美。透过她,作品揭示了一个根本的内在冲突:即本剧在魔法少女题材传统框架中的定位,与其渴望创新突破、成为全新叙事形态间的矛盾。这种矛盾可进一步理解为两种截然不同作品风格间的斗争:一类是本剧最初呈现的,深受《魔卡少女樱》影响的经典童话风格(尽管已然透露出些许黑暗);而随着剧情推进,它逐渐演变为另一类,承袭《少女革命》与《新世纪福音战士》血脉、令人深感不安的后继之作。简而言之,这是“虚假剧集”(以蓝光版前两集甜美温柔的片尾曲《明天再见》\footnote{译注:原标题为『また\, あした』,蓝光版第 1、2 话片尾曲、小圆角色歌。}为标志)与“真实剧集”(以第三集及之后的片尾曲“Magia”\footnote{译注:``magia''源自希腊语的 ``$M\alpha\gamma\varepsilon\acute{\iota}\alpha$'',意为“魔法”。蓝光版第 3 $\sim$ 8、11 话片尾曲,第 1、2、10 话插入曲。}为标志)之间的对抗。因此,第一章的核心任务,是在剧情最终倒向后者之前,为观众构筑一种二元对立:一端是对编剧虚渊玄风格尚且不熟的观众所预期的传统魔法少女剧;另一端则是他意图创作的黑暗解构之作。

动画伊始,以定格动画呈现、非传统动画形式的纸幕,缓缓升起。艺术与非艺术之别,关键在于“框架”\footnote{译注:迄今可考有关框架(frame)的最早论述系 1955 年 Bateson 的论文\emph{A Theory of Play and Fantasy} ,其中界定“框架”的概念为“个人组织事件的心理原则与主观过程”。}:一篇没有“框架”的故事只是谎言;一幅没有“框架”的画作只是涂鸦,正是“框架”定义了艺术。这本质上是制度主义定义:艺术是艺术家创作、并呈现给“艺术公众”的人工制品。所谓“艺术公众”,是指具备艺术理论与历史知识、能识别艺术品的群体\cite{ref9}。换言之,艺术是能以其编码方式被观众解码为艺术的存在。由此观之,《魔圆》中运用定格剪纸动画,以及远超常规动画范畴的艺术风格,或许正是作品对自身的定义与宣告:它强调自身作为独立作品的个性,而非公式化作品中的一个普通实例。

同时,纸幕令人联想到日本的“纸芝居”\footnote{译注:一种二十世纪初的日本街头表演艺术,类似于中国的皮影戏。}。这种20世纪初的艺术形式类似于欧洲的流动木偶戏,说书人游走于城镇之间,借助便携式舞台上的连续画片讲述故事\cite{ref10},并对漫画与动画的发展产生了深远的影响\cite{ref10}。通过纸艺的运用,《魔圆》既致敬了动画类型与艺术媒介的起源,也宣示了自身作为超脱于传统之外的独特存在。

这种诠释颇为契合,但让我们先回到剧情。开篇的视觉母题是小圆在一个扭曲广袤的内部空间奔跑,其棋盘格局令人联想到埃舍尔\footnote{译注:莫里茨·科内利斯·埃舍尔(1898年6月17日—1972年3月27日),荷兰著名版画艺术家。} 笔下的错位空间。黑白在此被严格区分,形成清晰的二元结构。注意,这个空间并非现实。与小圆梦境后半部分不同,它并不直接对应任何时间线上我们看到的事件,而仅仅预示着她即将遭遇首个魔女的场所。

梦境后半部分(正如我们在动画后期所了解)实则是前一个时间线的记忆。真正的片尾曲“Magia”在此处作为背景音乐响起,象征着这才是故事的真相。而蓝光版中,本集和下集的结尾并未播放,取而代之的是一首看似欢快(若不深究歌词)的歌曲以及少女嬉戏的画面(电视版则仅以职员表滚屏)。直到第三集,“Magia”才作为片尾曲登场,那一刻,作品彻底撕下了“正常”魔法少女动画的伪装。

视觉上,这段梦境以多层次灰阶构筑基调,几乎摒弃了纯黑、纯白与鲜亮色彩。废墟都市中心,一棵巨大的枯树耸立,象征着自然与文明的双重衰亡。瓦尔普吉斯之夜现身,她那小丑般的魔女形态与巨大齿轮暗示着人造秩序,却被癫狂的尖笑声所戳穿。万物皆被倒置。在《魔圆》的世界里,二元对立只是幻象;所有矛盾终将崩解,复归统一。后续我们将再次探讨此点。

本集动画竭力抑制这一片段所暗示的内容。作为梦境呈现,其处理方式与《魔卡少女樱》——影响《魔圆》最明显的作品之一——对首个预示性场景的表现如出一辙\footnote{《魔卡少女樱》第一集:“小樱与不可思议的魔法书”。}\cite{ref11}。随后的片头曲亦符合典型魔法少女动画风格(再次强调,只要忽略歌词),其中小圆的变身场景极具《美少女战士》\footnote{《美少女战士》第一集:“爱哭鬼小兔的华丽变身”。}\cite{ref12}或《甜心战士》\footnote{《甜心战士》第一集:“一只黑色爪子抓住了心脏”。}\cite{ref13}的特色,服装设计也令人不禁联想到小樱的几套魔法服饰。

事实上,本集其余部分大多都致力于将小圆的生活与性格描绘得尽可能善意且平凡。她的家庭近乎完美:一位温暖友善的全职主夫,一位事业心强、职场打拼的母亲,还有一个可爱的弟弟。之后,正如《美少女战士》第一集那样,小圆叼着面包跑出家门,险些上课迟到,此类例子不胜枚举。(实际上,这种“叼面包赶课的迟到女生形象”,在《新世纪福音战士》最后一集真嗣的幻想场景中,被用以传达“普通”绫波丽的日常生活\footnote{《新世纪福音战士》第二十六集:“在世界中心呼唤爱的野兽”。 }\cite{ref14}。)

即使焰的出现对小圆而言打破了常态,但对观众来说却完全在预料之中。神秘转校生早已是陈词滥调,甚至在以恶搞多种动漫为卖点的《凉宫春日的忧郁》等小说中也被直接点名\cite{ref15},这几乎是日本青少年遭遇超自然事件后引出具体故事的共同模板。焰在“小圆梦境”片段后出场,并迅速在高中生活的各个方面斩头露角,这进一步强化了我们的刻板印象。她那高冷的外表\footnote{译注:作者用 persona(表面形象) 一词,原指戏剧面具,现指个体在不同情境中展现的公开形象或角色伪装。}(作为一种防御性伪装,既符合该词的经典定义,亦契合现代用法)和无所谓成绩的态度,只会让她显得更加符合这种神秘设定。

除了一处例外,魔法少女世界侵入小圆生活的每一次情节都呈现出双重意义上的定式化——既缺乏新意,又符合该类型的常规套路。她救助受伤的丘比便是又一例证,该情节与《美少女战士》中救助露娜的场景\footnote{《美少女战士》第一话:“爱哭鬼小兔的华丽变身”。}高度相似,同样以获得魔法力量为条件,换取与邪恶战斗的承诺。那么,除了我们尚未谈及的魔女入侵,本集中最令人耳目一新、摆脱俗套的时刻,是小圆向仁美和沙耶香讲述梦境的片段,现实的介入打破了魔法氛围。在本剧最为平淡却极具智慧的场景之一中,三人认真探讨了多种对梦境的解释,从“或许小圆之前见过焰”的合理推测,到荒谬的“前世记忆”(当然,此解释最接近真相)。沙耶香甚至指出了未来的情节走向,这在动画与剧场版中均有所暗示——一个微妙的伏笔,表明在所有变为魔法少女的角色中,她是最努力维持理想化“正义战士”形象的人。

接下来让我们讨论本集中最不符合典型魔法少女动画风格的场景:魔女袭击。这既是魔法世界对小圆平静生活的侵入,也是非动画元素对动画元素的侵入。定格动画与手工纸艺风格爆发式地呈现怪诞又令人不安的景象,迫使小圆和沙耶香直面此前被屏蔽的陌生存在,感受她们已知世界之外的宽广与野蛮。异质的艺术风格强化了她们的迷失与恐惧,同时透出一丝荒诞乃至滑稽感。尤其是“安东尼”\footnote{译注:蔷薇魔女的使魔,其任务是创造庭园。}们——那些留着胡须的棉球,类似品客薯片商标\footnote{译注:品客(Pringles)为美国薯片品牌,其商标形象为一个留有褐色大胡子和红色领结的人脸。}的形象——既荒唐又令人不安。此种效果在后现代作品中屡见不鲜:将来自某一语境中的元素置于另一语境中,会制造出一种冲突与不协调的效果\cite{ref16},而幽默\cite{ref17}与恐怖\cite{ref18}皆依赖于此类不协调感。

焰除了叹道“偏偏是这种时候……”之外,对此番魔女袭击无能为力。她知道这是不可避免的。作为真《魔圆》的代理人(注意,她是小圆梦境中出现的唯一魔法少女),任何试图维持这种更安全、更舒适的虚假表象的努力都注定失败(我们将在第十集和《叛逆》中再次见证)。此种保护只能由焰的对立面——另一位角色——来实现,且仅能维持一小会。

即麻美,登场。

她刚出场就将自身定位为焰的敌人,通过威胁以保护丘比。此时的丘比仍是那个负责唤醒女孩魔法潜能的、可爱又讨喜的吉祥物兼引导者。只有当本作撕下伪装,它才会显露其操纵人心的恶魔本质,接管反派角色。如我们所见,麻美与焰有一共同点:二者皆为经验丰富、使用枪械的魔法少女。但除此之外,她们几乎完全对立。焰是新来的转校生,而麻美作为学姐资历更深。焰的枪械是现代武器,麻美的则是魔法火枪。焰的形象是直线、深色与冷色调的紫色;麻美则是曲线(不仅体现于身材,亦见于发型)、白色与暖色调的黄色。焰封闭、神秘、看似敌对;麻美则开明、友好。

最关键的是,麻美拥有恢复假《魔圆》世界的能力,而焰则无。在我们甚至未见其人之时,她的第一个举动便是在魔女结界中为小圆和沙耶香划定出一个安全区域。她是能够驱散真实《魔圆》的代理人,包括魔女与焰。在动画后期,我们将看到另一条时间线中,麻美得知魔女的真相——这是“真”与“假”《魔圆》之间的关键分歧点——她的反应既残忍又果决:第一枪击杀了杏子,随即又束缚住了焰,险些把她杀了。最终被小圆付出巨大的代价击败,迫使该时间线(在某种意义上可视为结局的一个废弃草案)被放弃。若有麻美在,那真正的《魔圆》根本无从实现。

麻美在本集中的角色定位是恢复秩序,将艺术风格回归熟悉的动画范式,治愈受伤的丘比,让它重新扮演类似《美少女战士》中露娜的角色,为小圆和沙耶香提供魔法契约。凭借她积极的态度、坚定的决心和强大的实力,麻美成为了过往魔法少女的有力代表,正如我们将在这一篇章中所见,她承载的是魔法少女题材中所有的经典主题。只要她挺身捍卫,虚假的美好世界便不会崩塌。

那么她,唯有“掉头”。
