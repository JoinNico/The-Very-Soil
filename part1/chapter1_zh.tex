%Completed
\chapter[一股秘溪潺潺流淌(似乎在梦里见过,那样...)]{一股秘溪潺潺流淌\protect\footnotemark(似乎在梦里见过,那样...)}

\footnotetext{译注:``Life cannot be destroyed for good, neithercan history be brought entirely to a halt. \textbf{A secret streamlet trickles} on beneath the heavy lid of inertia and pseudo-events, slowly and inconspicuously undercutting it. It may be a long process, but one day it must happen: the lid will no longer hold and will start to crack. This is the moment when something once more begins visibly to happen, something truly new and uniquesomething truly historical, in the sense that history again demands to be heard.''
\par
“生命永远不会彻底毁灭,历史也无法完全停滞不前。在惰性和伪事件的沉重盖子下,有一股秘溪在悄然流淌,缓慢而不显眼地侵蚀着它。这可能是一个漫长的过程,但总有一天它会发生:那盖子再也无法保持完整,开始出现裂缝。这一刻,某种真正的新事物开始再次显现,某种真正的历史事件开始发生,因为历史再次要求被倾听。”
\par
《魔圆》本来的世界就像 Havel 所说的“惰性和伪事件的沉重盖子”。然而,故事中的魔法少女们逐渐意识到,外表看似平静的世界下,实际上隐藏着一股巨大的改变力量,侵蚀着压抑的现实。小圆通过自我牺牲打破了这一切,创造了一个全新的秩序,象征着那个“盖子”的裂开,她的行动彻底改变了魔法少女的世界,推动着世界的发展。}

《魔法少女小圆》可以被理解为三个不同篇章,尽管这三个篇章的主题相互交融在一起,但每章都侧重于某特定角色或一对角色组合,并探讨一个特定主题。第一章涵盖前三集,重点着墨于麻美。通过她,揭示了本作的一个根本内在斗争,即《魔圆》在魔法少女类型中的定位与其追求成为另一种新事物之间的挣扎。这可以进一步理解为两个对立作品之间的斗争:最初是一部经典童话风格(尽管有些黑暗)、深受《魔卡少女樱》影响的动画;另一部则是它所演绎的令人极度不安的作品,类似于《少女革命》和《新世纪福音战士》的后继。简单来说,这可以理解为一种冲突:虚假的魔法少女剧(以蓝光版前两集,相当甜美温和的片尾曲《明天再见》\footnote{译注:原标题为『また\, あした』,蓝光版第1、2话片尾曲、鹿目圆角色歌。}前为代表)和真正的魔法少女剧(以第三集片尾曲后为代表)。换句话说,第一章的作用是,在剧情偏向后者之前,建立一种二元对立:一方是那些对虚渊玄不太熟悉的观众所期望的传统魔法少女剧,另一方则是他所创作的黑暗解构性作品。

动画的第一集,伴随着一块幕布的升起《魔圆》的开始了。然而,这并不是传统动画,而是纸幕升起的定格动画。鉴于艺术和非艺术之间的区别就在于“框架\footnote{译注:迄今可考有关框架(frame)的最早论述系1955年 Bateson 的论文\emph{A Theory of Play and Fantasy} ,其中界定“框架”的概念为“个人组织事件的心理原则与主观过程”。}”:一个没有“框架”的故事只是谎言;一幅没有“框架”的画作只是涂鸦。可以坚定地揣度,正是“框架”定义了艺术。这本质上是艺术的制度主义定义,即艺术是一件由艺术家创作并向艺术公众展示的工艺品,所谓“艺术公众”是指那些掌握艺术理论和艺术史,并能够识别作品为艺术的人\cite{ref9}。换句话说,艺术就是以一种能被观众所识别的方式对自己进行编码的东西。在这种情况下,我们必须考虑,定格剪影动画,以及更广泛的、远远超出动画范畴的艺术风格,或在某种意义上是《魔法少女小圆》的定义,这代表了它作为独立作品的个性,而非仅仅作为某一类型中的一个实例。

同时,纸幕让人联想到日本的紙芝居\footnote{译注:纸芝居,意为“纸上戏剧”,一种日本街头戏剧,类似于中国的皮影戏。},这是一种20世纪早期的艺术形式,类似于欧洲的流动木偶戏。说书人从乡到镇,通过在一个便携舞台上用戏偶和背景插图来讲述故事\cite{ref10}。这一形式也对漫画和动画的发展产生了重大影响\cite{ref10}。通过纸工艺的使用,《魔圆》同时宣示了其媒介和类型的起源,并强调了其作为一个独立于类型的个体作品。

正如我们将看到的,这种演绎方式非常适合。但首先我们必须处理小圆在一个扭曲的、广阔的内部空间中奔跑的场景,这让人联想到 Escher\footnote{译注:莫里茨·科内利斯·埃舍尔(1898年6月17日—1972年3月27日),荷兰著名版画艺术家,知名于其视错觉艺术作品,在平面视觉艺术有极大成就。} 的棋盘世界。这个空间并不真实,但黑与白清晰的边界,呈现出明确的二元对立。与梦的后半部分不同,它并不直接对应我们在任何时间线上看到的事件,而是对她第一次遭遇魔女地方的投射。

然而,梦的后半部分实际上是之前一个时间线的记忆(正如我们在动画结尾附近所了解到的)。动画中由 \emph{Magia\footnote{译注:《魔圆》第3 $\sim$ 8、11话片尾曲,第1、2、10话插入曲。``magia''源自拉丁文,而拉丁文来自希腊语的 ``$M\alpha\gamma\varepsilon\acute{\iota}\alpha$'',意为“魔法”。}}这一真正的片尾曲作为背景音乐来标志。在蓝光版中,本集和下一集的结尾并没有播放 \emph{Magia},而是放了一首看似欢快(只要不注意歌词)伴随着少女们嬉戏插画的歌曲(在电视播出版中,结尾只放了职员表而没有片尾曲)。直到第三集,\emph{Magia} 才作为片尾曲播放,届时该剧将彻底抛弃其“正常”魔法少女剧的所有伪装。

视觉上,这部分梦境是各种灰色的阴影,几乎很少有黑色、白色或是彩颜色。在破败的城市中央,耸立着一棵巨大的枯树,象征着自然与文明一同衰落。瓦尔普吉斯之夜出现了,她那魔女形态和巨大齿轮暗示着疯叫声所掩盖的不自然和秩序。一切都是她的敌人。在《魔圆》中,二元性是幻象;所有的二元对立最终都会统一,稍后我们会回到这一点。

动画这一集极力抑制住这一短暂场景的暗示。作为梦境呈现,正如最明显影响《魔圆》的动画之一,《魔卡少女樱》也以同样的方式演绎着自己的第一个预示性场景\cite{ref11}。接下来的片头曲也是非常典型的魔法少女剧(再一次,只要不去关注歌词),其中小圆的变身场景非常像《美少女战士》\cite{ref12}或《甜心战士》\cite{ref13},服饰也让人不禁联想到樱的几套魔法装。

事实上,本集的其余部分大多是为了将小圆的生活和性格表现得尽可能善意且普通。她的家庭几近完美:一个温暖、友善的全职家庭主夫父亲,一个事业心强的职场母亲,外加一个可爱的弟弟。而后是,就像《美少女战士》第一集中那样,她嘴里叼着面包跑出家门,差点上课迟到,还有无数其他例子。(实际上,这种“叼着面包跑向学校的迟到女孩形象”,在《新世纪福音战士》最后一集的真嗣幻想场景中被用来传达“普通”绫波丽的平凡生活\cite{ref14}。)

即使晓美焰的到来对小圆来说打破了她的正常生活,但对观众来说也是意料之中的。神秘的转校生早已经是陈词滥调了,甚至在部分以恶搞多个动漫为卖点的《凉宫春日的忧郁》等小说\cite{ref15}中被直接点名,这或多或少都是日本青少年遇到超自然现象的共同点。老生常谈,焰在小圆的梦境之后出现,也迅速在高中生活的各个方面都表现出色。而那高冷(chilly)的外表(作为一种防御性伪装,既符合该词的经典定义,也符合现代定义)和无所谓成绩的态度让她显得更加神秘。

除了一个例外,每一次魔法少女的世界入侵小圆生活的情节都是较为定式的,既不独特,也是这一设定的常规套路。她救助受伤的丘比便是又一个例子,这情节与《美少女战士》中救助露娜的场景\footnote{《美少女战士》第一话:爱哭鬼小兔的华丽变身。}很相似,同样是以解锁魔力为条件来换取和邪恶战斗的承诺。除了我们尚未提及的,本集中最令人耳目一新的、非俗套的时刻,是小圆向仁美和沙耶香讲述她的梦境时,现实的介入打破了魔法的氛围。在本剧中最平淡却又极具智慧的场景之一,三人认真讨论了几种对梦的解释,从“也许小圆以前见过晓美焰”的合理推测,到荒谬的“前世记忆”(当然,这个解释离真相最近)。沙耶香甚至指出了未来情节走向,这点在动画和电影中都得到了暗示的一个微妙伏笔,即动画描绘出了,在所有变成魔法少女的角色中,她是那个最努力去维持理想化“正义战士”形象的人。

接着是本集中最不像典型动画的场景:魔女来袭。这是魔法世界对小圆平静生活的入侵,也是非动画元素对动画元素的入侵。定格动画和手工纸艺风格的爆炸性、令人不安的怪诞图像,迫使小圆和沙耶香直面此前被屏蔽的陌生事物,感受她们所知世界之外的广阔与狂野。异样的艺术风格强调了她们的迷惑与恐惧,同时给观众也带来了一丝荒诞甚至恐慌感。尤其是“安东尼\footnote{译注:蔷薇魔女的手下,任务是创造庭园。}”们,那些留着胡子的棉球,类似薯片品牌 Pringles\footnote{译注:品客,美国薯片品牌。其商标是一个有着咖啡色大胡子和红色领结的人脸。}的商标,既荒唐又令人不安。这种效果在后现代作品中并不罕见,将来自其他情境中的元素放置于另一情境中,会产生一种冲突、矛盾的效果\cite{ref16},而诙谐\cite{ref17}和恐怖\cite{ref18}都依赖于这种不协调感。

焰除了只能说“偏偏这时候”,对这次魔女袭击无能为力,她意识到这是不可避免的。作为真正《魔圆》中的代理人(请注意,她是唯一出现在小圆梦境中的魔法少女),她任何试图维持这种更安全、更舒适的虚假表象的努力注定会失败(正如我们将在第10集和《叛逆》里再次看到的那样)。这种保护只能由焰的对立面,即另一位角色来实现,而这也只能维持一小会。

也即,麻美登场。

她立刻将自己定位为焰的敌人,威胁她以保护丘比,此时的丘比仍是那个可爱、讨人喜欢的吉祥物,带领女孩们觉醒魔法的使者。直到动画逐渐揭开丘比的真面目,一个操控他人的恶魔角色,接替了反派的位置。正如我们所看到的,麻美和焰有一个共同点,她们都是使用枪械、经验丰富的魔法少女,但除此之外,她们几乎完全是对立的。焰是转校生,而麻美作为学姐。焰的枪械是现代武器,而麻美的则是魔法火枪。焰是直线、深色和紫色的代表;麻美则是曲线(不仅体现在身材上,还有头发)、白色和黄色的代表。焰是封闭、神秘、看似敌对的;而麻美则是开放、友好的。

最重要的是,麻美拥有维持虚假《魔圆》世界的力量,而焰却没有。在我们甚至还没有看到麻美之前,她的第一个举动就是在魔女迷宫中为小圆和沙耶香划定了一个安全空间。她是能够赶走真正《魔圆》世界中的代理人,包括魔女和焰。在动画的后半部分,我们将看到另一个时间线,麻美得知魔女的真正本质,这是“真正”和“虚假”的《魔圆》之间的一个关键分歧。她的反应既残忍又高效,用第一枪杀死了杏子,接着束缚了焰,差点杀死她。她最终被小圆付出了巨大的代价亲手击败,这也迫使焰放弃这条的时间线(这条线某种意义上也可以看作是被废弃的“草稿”版本动画结局),因为如果她还在的话,真正的《魔圆》是根本无法实现的。

她在这一集中的角色定位是维持秩序,将艺术风格维持在传统的动画形式上,治愈受伤的丘比,使它重新回到类似于《美少女战士》中露娜的存在,为小圆和沙耶加提供魔法力量的形象。凭借她积极的态度、坚定的决心和相当强大的力量,麻美成为了过去魔法少女的有力代表,正如我们将在这一篇章中看到的,她带来了魔法少女题材的所有经典主题。只要她站出来捍卫这一切,虚假的世界就不会崩塌。

那她只好“掉头”了。





