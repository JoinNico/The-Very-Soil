\chapter[绝不是乐观主义(不再依靠任何人)]{绝不是乐观主义\protect\footnotemark(不再依靠任何人)}

\footnotetext{``Hope is definitely not the same thing as optimism. It is not the conviction that something will turn out well, but the certainty that something makes sense, regardless of how it turns out.''    
“希望绝不是乐观主义。它不是相信某件事情会变得更好,而是确信它是有意义的,无论最终结果如何。”

这一段话表达了一种深刻的对“希望”的理解,与传统的乐观主义(即期待事情会变得更好)有所不同。希望是对存在的某种意义的信念,而不依赖于外部结果的好坏。这种观点正与《魔法少女小圆》所讨论的主题高度契合,尤其是与故事中角色的内心挣扎和他们所面临的无常与痛苦相呼应。

从文章内容来看,作者选用这个标题是为了强调焰在故事中的核心情感态度,她的“希望”并非传统意义上的乐观主义或对未来一切都会好转的期待,而是源于她对小圆命运的深刻理解与对她的保护欲。焰不断重复穿越时间,不是因为她相信事情最终会变得更好,而是因为她坚信无论如何,这个过程本身有意义:即使她所做的一切可能并不带来预期的“成功”,她依然愿意为小圆付出一切,哪怕最终的结果是痛苦和失败。

焰的希望来自于对“意义”的执着,而非简单的对乐观结果的期待。她并不确定小圆的命运会改变,但她确信这一切的努力、痛苦和牺牲是有意义的,这种“有意义”并不依赖于外部的成功或失败,而是一种对自我选择与爱的坚定信念。尽管在每条时间线中,她都面临着失败和绝望的边缘,但她依然保持着不放弃的决心。

在此,文章的标题与故事的主题形成了对比,暗示了焰对于小圆的保护欲是建立在对命运深刻理解的基础上,而不是对简单乐观未来的幻想。她的行为并非建立在乐观主义的期待上,而是基于她的内心信念——即使她知道无论结果如何,她的选择本身依然充满意义。因此,标题不仅仅是对哈维尔名言的引用,也是对焰复杂情感的准确表达:希望是一种深刻的存在感,而不是盲目的乐观。}

让我们从头开始。

除了诸多其他方面的内容外,《魔圆》也是对萌文化的批评,萌属性指是二次元中那些柔弱的、激发人们保护欲的女性角色。正如我们在沙耶香的故事中看到的那样,这种保护欲是徒劳且危险的,因为被保护意味着被物化,被视为不具备自由意志的“他者”,而不是一个完整的人类个体。

就像魔法少女其实是魔女一样,成为保护者就是成为摧毁者,而这一点对焰(其姓“晓美\footnote{译注:``whose family name, Akemi, means `fire', ...'',作者这里搞错了,实际上是名“ほむら”,意为火焰。}”意为“火焰”,暗示与萌属性的“燃烧”有关)和沙耶香身上同样适用。注意焰的愿望,她希望与小圆交换位置,成为那个保护小圆的人,而不是被她所保护。通过把自己设定为小圆的保护者,把被保护的身份强加于小圆,这也是小圆遭受如此多痛苦的根本原因,但这并不让人感到意外,因为过度保护某人很容易导致对方失去信心。

这集中的佛教根源尤为重要,因为焰被她自己的观念所束缚——她没有去询问小圆真正想要什么,而是将她自己认为她想要的事物强加给她,而这实际上是基于她自己个人的愿望。焰的欲望因此带给她痛苦,就如佛陀所教导的那样\cite{ref62}。她也因相信自己强加给世界的观念本身就是真实的而迷失了方向;相反,那个变化无常的世界(包括她自己的愿望与行动的影响)否定了她试图划分的界限,比如未来与过去、生与死、人类与魔法少女与魔女。这正是禅宗的观点:语言和思维妨碍了我们看到普遍真理的能力——这个真理不是由独立的实体和状态组成,而是宇宙作为一个永恒流动的“一在其中”的状态\cite{ref63}。

焰最初努力从死亡中拯救小圆,因为她认为死亡和生命是不同的;接着她努力阻止小圆变成魔女,因为她认为魔女和魔法少女是不同的;最后,她努力阻止小圆成为魔法少女,因为她认为魔法少女和普通人类是不同的。但因为她根本不理解自己身处何种故事,值得注意的是,她到来到见泷原中学前曾在基督学校学习过,意料之中,她无法认识到自己的行为只会让事情变得更糟。

简而言之,这根本没用。

让我们从头开始。

上一章我讨论了魔法少女题材的起源,以及“好女孩与坏女孩”(或魔法少女与魔女,或圣女与荡妇)二元对立如何将女性力量的表达局限在某些不会威胁到男性霸权的表达方式中,或者将其压缩成边缘化的角色,无论是深居森林中的魔女,还是永远只与别人看不见的怪物战斗的魔法少女。

第十集通过多个时间线来描绘焰的角色,并展示她从无威胁性的角色到边缘角色的演变。在第一条时间线中,她是一个毫无力量的无辜者,弱小到足以成为魔女的受害者。尽管如此,这也预示着她最终冷静、决绝的决心,即使身体虚弱,她是剧中唯一一个没有完全屈服于魔女精神控制的魔法少女。随着时间线的推移,她变得越来越强大,也变得更加反叛。在第二条时间线的开始,她显然是最弱、技巧最差的魔法少女;她的时间停止能力纯粹是一种辅助能力,没有明显的攻击性应用,对魔女的伤害能力微乎其微。然而,到这一条时间线的末尾,她已经能够使用自制炸弹,并结合时间停止能力来击杀魔女,小圆则变为了辅助角色。在第三条时间线中,焰变得更加强大且更具反叛性,从在家里做炸弹到从黑道那里偷枪,使她能与人鱼魔女一战。

在第四条时间线中,焰终于成为我们熟悉的那个形象:冷漠无情。她剪掉了少女气的辫子,扔掉了象征“弱小”的眼镜,摒弃了自己的不确定性,去抢夺霸权男性气质的最终象征:军队,凭此孤身一人战斗。

但这些焰的形象并不是按照从过去到未来的顺序展开的,而是跨越了多个时间线,这意味着在某种意义上,所有这些焰是同时存在的。“局外角色”所处的“外部”是一个对男性霸权没有威胁的表达方式,它们之间的表面二元对立就像魔法少女和魔女之间的对立一样是虚幻的;麻美和焰不过是(恰是)同一枚硬币的两面。

这意味着焰并不是对魔法少女类型的指责,也不是对其依存的安全、无挑战性框架发出挑战。她只是这一框架的另一种表现形式,一个让观众感到熟悉、舒适的魔法少女的替代形象。

简而言之,这根本没用。

让我们从头开始。

在第一章中,我提出了《魔圆》有三个篇章,每章关注一个特定角色或一对角色,并强调不同的主题。第九集闭幕了沙耶香与杏子间的故事,这一故事主要探索了抑郁,并了结了魔法少女题材一直以来的循环。因此,第十集就负责引入主要着墨于焰的最后一章。

因此,我们得以从她的视角出发,跟随她一次次穿越时间,试图逆转小圆的命运。在这一过程中,焰很大程度上免疫了曾在第二章击垮沙耶香绝望;焰唯一一次差点成为魔女是在本集第三条时间线中,当她提议和小圆一起变成魔女,抹去整个世界及所有悲伤时,这一情节也预示着小圆最终的愿望以及《叛逆》的主要情节。

第三条时间线有什么特殊之处?在第一条时间线中,焰只是在瓦尔普吉斯之夜杀死小圆后才成为魔法少女;在第二条时间线中,焰似乎没有受到任何伤害的迹象表明她几乎没有参与与瓦尔普吉斯之夜的战斗;在第四条时间线中,小圆在瓦尔普吉斯之夜击败焰后才成为魔法少女。

焰差点成为魔女的时间线正是她真正实现愿望的时间线,即在与瓦尔普吉斯之夜的战斗中保护了小圆。这就是沙耶香所说的希望与绝望的平衡。高潮直接通向深渊。

焰通常对绝望免疫,因为她所怀希望不同寻常。她依靠的是自己的决心和对小圆的爱;既然这两者并非天生的美好情感,也就不会带来绝望的平衡。换句话说,她体现了 Václav Havel 所述的希望的概念:“希望绝不是乐观主义。它不是相信某件事情会变得更好,而是确信它是有意义的,无论最终结果如何。”\cite{ref64}

正如她在这一集结尾所说的,只要是为了小圆,不论多久她可以坚持。但那么,第八集中的她崩溃又该如何解释呢?小圆似乎在每条时间线中都决定牺牲自己,这对焰的计划构成了唯一的、最大的威胁。焰实际上处于一种功能性抑郁的状态,也就是说,虽然她还能继续活动,但她依然处于抑郁之中,只需要轻轻一推,就足以让她崩溃。

简而言之,这根本没用。

让我们从头开始。

第十集为我们呈现了一系列与主线相关的不同时间线。每当时间线到达了终结的灾难:瓦尔普吉斯之夜,焰就会重置自己回到她离开医院的那一天,再次尝试。然而,看起来她不仅仅是回到过去,还在穿越着不同的时间线,因为每条时间线中的元素似乎会根据时间线的变化而变化。例如,在第一条时间线中,小圆在焰离开医院之前便成为了魔法少女。

在这四条不同的时间线中,有一些事件与动画主时间线的相呼应(或预兆,取决于你的观点),从第一条时间线中小圆转身面对焰的镜头(这是第一集中焰转身面对小圆镜头的呼应)到杏子拒绝接受人鱼魔女完全取代了她认识的沙耶香的情节。

但更有趣的是那些更微妙的呼应。比如,焰在医院的反复醒来与第三集中的夏洛特的魔女种子在医院孵化的情节相呼应,这强调了它们作为预示整个动画走向的角色以及麻美敌人的关系。小圆恳求焰阻止她与丘比签订契约,反而也将焰置于与小圆一样犯下错误的角色位置,呼应了第六集中小圆母亲的忠告。甚至焰逐渐失去纯真,转变为一个更黑暗、更成熟、更具反叛性的角色,也与沙耶香整章的发展轨迹相呼应。

这些呼应不仅帮助我们理解这些时间线之间的关系;它们还重新赋予了这些事件新的语境。从焰的角度来看,这些并不是呼应,而是预示(正如某些事件也对观众来说是预示一样,比如麻美是第一个攻击其他魔法少女的暗示着《叛逆》的情节);这一切都发生过,且很可能还会再次发生。

这两集彻底改变了整个故事。第一次观看时,这两话让人有种:原本似乎是关于某件事情的故事,突然转向了完全不同的发展。它们就像镜子一样,安置在动画的两端,第三集和倒数第三集。这是两集中的第二集。

如果不谈别的(但确实有很多其他方面),《魔法少女小圆》无疑是一部结构精巧的动画。剧中的每一刻都经过精心安排,在惊人的紧凑空间里推动如此复杂的故事和人物发展进程。这种精心编排的例子之一便是剧集中的对称性,以及大大小小的场景在恰当的时间重复出现。

简而言之,这根本没用。
