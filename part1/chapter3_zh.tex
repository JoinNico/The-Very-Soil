\chapter[打破秩序,出乎意料(已经没什么好害怕了)]{打破秩序,出乎意料\protect\footnotemark(已经没什么好害怕了)}

\footnotetext{译注:``Drama assumes an order. If only so that it might have - \textbf{by disrupting that order - a way of surprising}.''“戏剧预设了一种秩序,而这样做的目的,便是通过打破这种秩序,制造出一种出乎意料的效果。”
\par
动画从一个看似平凡的“魔法少女”世界出发,引导观众熟悉其现有的节奏,然而随着逐渐引入的黑暗元素,再通过打破这种秩序,制造冲突(麻美的死亡),彻底颠覆观众的预期。这种颠覆不仅让剧情发生了巨大变化,也打破了原本观众对剧集的期待,带来了情感上的强烈震撼,让动画充满张力和吸引力,故事也由此进入了真正的核心戏剧冲突。}

《魔法少女小圆》即便抛开其他特质不谈(何况这些又是如此丰富),也堪称一部结构精巧的杰作。每一帧画面都经过精心设计,力求在有限的篇幅内推进复杂的叙事,深度刻画人物。贯穿全剧的对称性便是明证——大大小小的情节在精准的时间点上彼此呼应、回响。

剧中有两集彻底扭转了整个故事的走向。初观时,它们犹如惊雷,令作品摇身一变,将一种叙事陡然颠覆为了另一种。这两集恰似镜像,分列于剧集的两端:第三集与倒数第三集。本章聚焦的,正是这关键的第三集。

本集开篇便为剧中三个主要篇章里最长的一章(聚焦沙耶香与杏子)埋下伏笔。沙耶香探望少年上条恭介,将她和小圆在第一集购买的 CD 送给他。两人一起聆听音乐之际,沙耶香回忆起儿时听恭介演奏的情形,那是她初次领略真正艺术之美,也是第一次见证人类创造力的时刻。我们虽未确知恭介的病症,只晓他手部功能受损,无法再拉小提琴;但关键在于,恭介昔日创造的美,如今已成绝响,遥不可及。这是该剧首次显现其核心主题之一,亦是其佛教底蕴的印记:衰败与消逝的必然。恭介创造的美终将逝去,因为万物皆然。时间,是万物的终结者\footnote{此处“时间为终结者”之喻,意在为更熟悉基督教背景的西方读者提供理解佛教“无常”观的桥梁。详见文献\cite{ref25}。}。恭介音乐的失落与麻美家人殒命车祸、宇宙在熵增的洪流中走向热寂并无本质不同,唯有规模之差罢了。

幸而,片头曲总能将我们从忧郁思绪中拉回!毕竟,此刻仍是那个“虚假”的《魔圆》,那个安然、舒适的魔法少女童话。虽偶露暗影,但片头曲或麻美总会在黑暗降临前及时地拯救我们。

然而,黑暗已然如画框边缘的利齿,悄然渗入。麻美追忆自己许愿求生的一幕,暗藏太多未尽之言:她独居,经济来源不明,且明显身处车祸车辆的后座。她仓促许愿求生,如今则告诫沙耶香与小圆:务必深思熟虑,确保所求即所愿。

这强烈地暗示着:她本可许愿全家生还,却只求自己活命。剧集的副文本\footnote{译注:文学术语,指围绕作品正文本的辅助性文本,如标题、序跋、注释、附录等,共同构建作品意义。在这里可以指代动画设定集。}暗示,夏洛特(或说至少是早期草稿中的“奶酪魔女”)亦然:她本可祈求母亲不死,却选择了与母亲共尝最后一块蛋糕\cite{ref26}。

但夏洛特并非麻美在此唯一的平行角色。上一集,我们看到麻美的魔法少女变身与小圆母亲变身为另一类“变身”形象——雄心勃勃的职场攀登者——形成对照。本集,我们目睹母亲被典型日本上班族生活难以逃避的一部分击垮:工作后的宿醉狂欢\footnote{值得注意的是,这种带有刻板印象的男性化应酬,突显了母亲在动画中性别倒置的形象。}\cite{ref27}。她踉踉跄跄回到家,既预示麻美即将遭遇魔法少女的必然宿命(即惨烈死亡),也意外促成了小圆与父亲间一次至关重要的对话。

小圆自然发问:母亲为何乐在其中?作为利润机器中一颗野心勃勃的齿轮,她究竟在实现何种梦想?父亲解释,母亲的梦想并非“做什么”,而是“为什么”;她为工作本身而工作,珍视的是努力付出的过程。

这恰与麻美向沙耶香提出的核心问题遥相呼应:沙耶香是想帮助恭介,还是想“成为帮助过恭介的人”?她是否在为自己谋求某种私利?若是如此,就该直接许下那个愿望;抑或她真正渴望的,就是“帮助”这一行为本身?正如小圆对“成为魔法少女”感兴趣,而沙耶香执着于“打击邪恶”,此刻的沙耶香仍专注于“想做的事”,而非“想成为的状态”。她以为行动自会带来理想,却未曾道出内心真正的渴望。

这与麻美对沙耶香提出的关键问题相呼应:沙耶香是想帮助恭介,还是想成为帮助过恭介的人?她是否想要为自己得到某种东西,如果是这样的话,她就该许下那样的愿望;或者她真的只想要帮助别人?就像小圆感兴趣的是成为魔法少女,而沙耶香则想打击邪恶一样,专注于她想做的事情,而不是她想达到的状态。她以为自己的行动会带来那个理想状态,但她还是未能表达出自己真正的愿望。

同样,麻美在与小圆的最后对话中袒露心迹:她憎恶因愿望而陷入的孤独境地。尽管她曾表示现况好过死亡,却仍懊悔当初未能许下更明智的愿望,并深感彻骨孤独。但在本集尾声,小圆如同全剧结局时那样,提醒麻美她并非孤身一人,并承诺成为魔法少女来帮助、支持她。

恰恰是这一刻,小圆“害死”了麻美。得知自己不再孤独的喜悦,令麻美比上一集更加忘乎所以。她低估了夏洛特的威胁,由此丧命。更重要的是,如同恭介的音乐,她的欢愉注定无法持久。它必将终结、衰败、变味,这是身处时间洪流中一切存在的必然宿命。

唯有一个例外:奶酪。

从副文本和《叛逆》电影中可知,夏洛特痴迷于奶酪,永无止境地追寻着它。而奶酪为何物?不正是从腐败中诞生,既美味又滋养生命的存在吗?它是变质的牛奶,却升华为人间至味和生命养料,是炼金术中“腐化”概念的完美诠释。无论在物质抑或精神层面,死亡皆生命之源\footnote{摘自\emph{The Alchemy Reader: From Hermes Trismegistus to Isaac Newton}:“腐化乃万物之变迁与死亡,是其初始本质之毁灭;由此再生萌发,一种远胜过往千百倍的新生命得以诞生”}\cite{ref28}。腐烂令人作呕,但那丑陋蠕动、遍布霉菌与蛆虫的腐败温床,却生机盎然,滋养着出多美丽可爱的生灵;若无腐化,生命亦将不复存在。

通过吞噬麻美,夏洛特寻获了她的奶酪。这场死亡与腐化催出了新生——此刻,《魔圆》彻底冲破其题材的桎梏,奔涌的潜能终于喷薄而出。就在麻美初次攻击夏洛特后仅数分钟后,天地已然翻覆:麻美殒命,焰救了沙耶香和小圆。当她们在医院停车场失声痛哭时,丘比冷眼旁观,毫无慰藉之意。而当 \emph{Magia} 终于作为真正的片尾曲响起时,有一点已昭然若揭:《魔法少女小圆》的传奇,由此正式拉开序幕。
