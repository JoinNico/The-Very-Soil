\chapter[打破秩序,出乎意料(已经没什么好害怕了)]{打破秩序,出乎意料\protect\footnotemark(已经没什么好害怕了)}

\footnotetext{译注:``Drama assumes an order. If only so that it might have - \textbf{by disrupting that order - a way of surprising}.''“戏剧预设了一种秩序,而这样做的目的,便是通过打破这种秩序,制造出一种出乎意料的效果。”
\par
本章标题解释了该集的重要性:动画从一个看似平凡的“魔法少女”世界出发,引导观众熟悉其现有的节奏,然而随着逐渐引入的黑暗元素,再通过打破这种秩序,制造冲突(麻美的死亡),彻底颠覆观众的预期。这种颠覆不仅让剧情发生了巨大变化,也打破了原本观众对剧集的期待,带来了情感上的强烈震撼,让动画充满张力和吸引力,让故事真正进入了核心的戏剧冲突。}

《魔法少女小圆》如果没有什么特别的(事实上,它有不少),至少是一部结构极为精巧的作品。它的每一秒都被精心安排过,以便在相对短小的篇幅中推进复杂的剧情,深入塑造角色。一个能展示这种精心编排的例子就是贯穿全剧的对称性:大小情节在恰当的时机重复,彼此呼应。

在整部作品中,有两集彻底改变了整个故事。初观时,这两集尤为突出,因为它们标志着这部作品从看似讲述这种故事,突然转向了完全不同的叙事。它们镜像地出现在整个剧集的两端:第三集和倒数第三集。我们现在来讨论第三集。

这集的开头为动画中三个主要篇章中最长的第二章埋下了伏笔,这一篇章主要着墨于沙耶香和杏子的关系上。沙耶香探望了一个叫上条恭介的男孩,并把她和小圆在第一集中一起买的 CD 送给了他。两人一起听 CD 时,沙耶香回忆起小时候听恭介演奏的情景,那是她第一次真正感受到美的时刻,也是她第一次见证人类创造力的时刻。虽然我们并未确切知道恭介得了什么病,只知道他手受伤了,无法再拉小提琴;但无论如何,关键是恭介曾经创造的那种美现在已是过去式,遥不可及了。这是该剧首次明确展现其主要主题之一,也是其佛教思想的明显体现:衰败与消逝的必然性。恭介曾经创造的美终将逝去,因为万物皆会如此。时间是万物的终结者\footnote{旨在让那些只熟悉基督教的西方读者更容易理解,关于佛教中“无常”的概念可以参考文献\cite{ref25}。}。这种美的消逝与麻美一家在车祸中丧生,或是宇宙被不可阻挡的熵增摧毁并无本质上的不同,它们只是在尺度上有所差异。

不过,幸好有片头曲把我们从这种忧伤中拯救出来!毕竟,现在还是那个虚假的《魔圆》,那个安全、舒适的魔法少女剧,虽然偶有黑暗的影子,但总会有片头曲或麻美学姐在关键时刻出现,把我们从中拉回。

然而,黑暗正如“框架”边缘上的利齿一般,悄然吞噬着整部剧集。麻美回忆起她当初的愿望:活下来。这段回忆隐含着很多未说出口的东西:麻美一个人住,生活经济来源不明,坐在车子的后座上出了车祸。她匆忙许愿求生,现在便告诫沙耶香和小圆,要认真思考自己的愿望,确定自己真正想要的是什么。

很明显的暗示,她本可以许愿让全家一起活下来,但只许愿自己能活下去。副文本\footnote{译注:指围绕在作品正文本周围的一些辅助性文本,主要包括标题、副标题、序、跋、作者署名、插图、图画、题辞、注释、附录、广告等。}也暗示,夏洛特(或者说至少是早期草稿中那个“奶酪魔女”)也是如此:她本可以许愿让母亲不死,但却许愿和母亲一起吃最后一块蛋糕\cite{ref26}。

不过,夏洛特并不是麻美唯一的平行角色。上一集我们也看到了类似麻美的魔法少女变身的小圆母亲的另一种“变身”形象:雄心勃勃的职场攀登者。而这一集,我们看到小圆的母亲在典型日本上班族生活中,难免要经历的宿醉\footnote{值得注意的是,这是一个带有成见的男性化活动,强调了询子性别颠倒的特征。}\cite{ref27}。她踉踉跄跄回到家,不仅暗示麻美也即将迎来魔法少女的必然结局(凄惨地死亡),同时也促成了小圆和父亲之间的一次关键对话。

小圆问了一个意料之中的问题:母亲为什么喜欢她的生活?她作为利润机器中一个充满野心的齿轮,究竟是在实现什么梦想?父亲解释道,小圆母亲的梦想不是做什么事,而是在做什么事;她为了工作而工作,珍视的是努力本身。

这与麻美对沙耶香提出的关键问题相呼应:沙耶香是想帮助恭介,还是想成为帮助过恭介的人?她是否想要为自己得到某种东西,如果是这样的话,她就该许下那样的愿望;或者她真的只想要帮助别人?就像小圆感兴趣的是成为魔法少女,而沙耶香则想打击邪恶一样,专注于她想做的事情,而不是她想达到的状态。她以为自己的行动会带来那个理想状态,但她还是未能表达出自己真正的愿望。

同样,麻美在与小圆的最后对话中透露她讨厌自己因愿望而陷入孤独的处境。虽然她先前表明,现在的生活方式比在那时死掉要好多了,但她依然后悔未能许下更好的愿望,并且感到极度孤独。但在本集结尾,小圆提醒麻美,她并不孤单,承诺要成为魔法少女来支持、帮助麻美。

正是这个瞬间,小圆“害死”了麻美。麻美在得知自己不再孤单后感到无比快乐,这种快乐促使她比上一集更加自满,以至于她低估了夏洛特的威胁,结果因此丧命。更重要的是,就像恭介的音乐一样,她的快乐无法持久。它注定会结束、衰败、变得苦涩,因为这是桎梏于时间中事物的必然结局。

除了一个例外:奶酪。

从副文本和《叛逆》电影中,我们知道夏洛特痴迷于奶酪,永不停歇地寻找着它。而奶酪是什么呢?不就是从腐败中诞生的,既美味又富有营养的东西吗?它是腐坏的牛奶,被升华为美食和营养的来源,是炼金术中“腐化”概念的完美例子。无论是从物质上还是精神上,死亡是生命的源泉\footnote{摘自\emph{The Alchemy Reader: From Hermes Trismegistus to Isaac Newton}:“腐化是所有事物的变化与死亡,是所有自然事物初始本质的毁灭;因此,便出现了一种再生,一种比以往好千倍的新生命。”}\cite{ref28}。腐败是令人厌恶的,但丑陋、蠕动的霉菌和蛆虫却充满了生机,滋养出了更加美丽、更加可爱的生物;没有腐化,就没有生命。

通过吞噬麻美,夏洛特找到了她的奶酪。这种死亡与腐化带来了新的生命,因为这一刻标志着《魔圆》超越了其所属题材的束缚,开始展现其潜力。就在麻美第一次攻击夏洛特后的几分钟内,一切都变了:麻美死了,焰救了沙耶香和小圆。当她们在医院停车场痛哭时,丘比毫无安慰之意。而当 \emph{Magia} 终于作为真正的片尾曲响起时,有一点已经非常清楚:《魔法少女小圆》正式开始了。
