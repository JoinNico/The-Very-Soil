\introduction{万物结构之尺}
《魔法少女小圆》(以下简称《魔圆》,以区别于主要角色:鹿目圆\footnote{虽然日语的命名习惯是姓氏在前(如,鹿目\, 圆),但《魔圆》的官方翻译是使用英语的命名顺序,即先名字,后姓氏。为了避免混淆,本书遵循剧中的顺序。同时为了保持一致性,其他作品中的角色或真实人物,本文也将使用英语顺序。译注:中文显然是没有这类问题的。}, 后者在本文中简称“小圆”)是多重含义的。最显而易见的,它是由动画公司Shaft制作的魔法少女题材的动漫系列,但除此之外,它也是对该类型的批判和致敬;是一个迷失在黑暗中的作家自我救赎的尝试;是对浮士德的佛教式重述。它是一项对比研究,既是对同意、特权和性别角色等问题的女性主义考察,同时也是对未成年女性形象的性别歧视的客体化。它包含了牺牲与自私、绝望与希望、爱与死亡、腐朽与生命。当然,这些看似对立的概念实则是统一的。

但最重要的是,它是关于人的故事,这些人做出选择(通常是错误的),然后受苦、奋斗、成功、失败。这是关于角色们彼此相爱、相恨、相漠不关心,有时甚至皆有的故事。

这本书将讨论一些宏大的主题、哲学和符号,因此在开始前,我想提醒大家:这是一个关于人的故事,因为所有故事本质上都是关于人类的\footnote{ “摹仿者所摹仿的对象是行动中的人。”请注意,在《诗学》第一部分中,亚里士多德表明他将戏剧和叙事也视为诗歌的一部分,而在今天,他所使用的“摹仿的对象”一词,我们称为作品的主题。}\cite{ref1}。 
\paragraph{为什么研究《魔圆》?}~{}

在十二集,每集只有短短二十二分钟的电视动画以及一部电影中(虽然有三部上映,但前两部是电视动画的总集篇,只有少量的改动),《魔圆》呈现了格外丰富的故事,刻画了饱满的角色、复杂的主题以及极为悲伤、恐惧、愤怒和喜悦的瞬间。对它进行分析非常有成就感,每次仔细研究后都能揭示出更多的层次。尤其是它作为一部后现代主义作品,非常适合用后现代主义的视角进行分析。
\paragraph{“后——”什么?}~{}

“后现代主义”这个术语的定义相当复杂,但至少在谈到它时,我一定会尝试去解释清楚。在哲学上,顾名思义,后现代主义是现代主义的进一步延伸。

简而言之,现代主义的核心认识是:符号本质上是任意的。也就是说,符号与其所代表的事物之间没有固有的联系,只有观看符号的人脑海中的关联。例如,一个红色六边形意味着“停”,并不是因为红色、六边形与停止之间有某种逻辑关联,而是因为有人决定红色六边形应该代表“停”,并让其他人接受了这一点。这就是社会建构主义的核心概念,即我们理解世界的概念和符号是由社会环境和关系构建而成的。因此,这些符号和社会一样是流动的,没有固定的客观意义\cite{ref2}。

既然所有艺术(和语言)在后现代主义思想中都是一系列符号,那么任何作品的意义在某种程度上都是任意的,是社会建构的产物,而不是逻辑必然关系的结果。传统上,艺术通过使用约定俗成的文化符号来解决这一问题,如语言或具象图像(例如用苹果的图像代表一个苹果)。这些符号作为“导轨”,帮助观众从一些熟悉的符号开始接触艺术,再逐步构建出更多意义。现代主义则以拒绝传统和表现失落感、解体感为主要特征,现代艺术经常抛弃部分或全部“导轨”,激烈挑战艺术是否应该拥有非任意意义的概念。例如,James Joyce\footnote{译注:詹姆斯·乔伊斯(1882年2月2日—1941年1月13日),爱尔兰作家和诗人,20世纪最重要的作家之一。代表作包括短篇小说集《都柏林人》(1914)、长篇小说《一个青年艺术家的画像》(1916)、《尤利西斯》(1922)以及《芬尼根的守灵夜》(1939)。尽管乔伊斯一生大部分时光都远离故土爱尔兰,但早年在祖国的生活经历却对他的创作产生了深远的影响。他的大部分作品都以爱尔兰为背景和主题。他所创作的小说大多根植于他早年在都柏林的生活,包括他的家庭、朋友、敌人、中学和大学的岁月。乔伊斯是用英文写作的现代主义作家中将国际化因素和乡土化情节结合最好的人。}的作品以意识流的形式呈现,打破了正常的句子结构;Mondriaan\footnote{译注:彼得·蒙德里安(1872年3月7日—1944年2月1日),荷兰画家,风格派运动幕后艺术家和非具象绘画的创始者之一,对后代的建筑、设计等影响很大。}的画作则用抽象的色块和形状代替了平常的物体。然而,现代主义常常试图重新创造它认为已经失去的秩序\cite{ref3}。例如,Joyce 很好地将一个普通日子描绘成一场史诗般的斗争,Mondriaan 则将他的抽象色块安排成网格状的图案。

与此相反,后现代主义拒绝将社会建构的意义等同于“没有意义”。现代主义说:“符号与其所指的关系是任意的,因此一切都没有意义。”而后现代主义则说:“符号与其所指的关系是任意的,因此我们可以自由决定一切的意义。”因此,后现代主义作品不需要重新引入失去的秩序,因为那种秩序从未真正存在过。正因为如此,后现代时期对流行文化作为艺术表达的媒介的接受度也更高\footnote{但请注意,Murfin 和 Ray 在他们的定义中强调了负面和实验性方面,而淡化了 Hassan 认为是核心特征的多元化。}\cite{ref4}。现代主义拒绝传统的意义、形式和技巧,便创造一种新的秩序来取代旧的;而后现代主义则将传统与新事物结合起来,以追求怪异、有趣的东西\footnote{Hassan 提出了十一项相互重叠的标准。我的定义在很大程度上依赖于他的“建构主义”、“去经典化”、“讽刺”、“狂欢化”和“混合化”标准。}。\cite{ref5}

现代主义通过消除“导轨”,而后现代主义则通过强调“导轨”来创作。这主要通过去情境化和再情境化的过程实现,将熟悉的符号从其通常的情境中移除,并将其放置在其他情境中,例如通过混合多种类型的元素,创作出常见作品的集锦,或打破“第四面墙”\cite{ref5}。所引起的迷失感一般用于制造幽默或恐怖效果,或者引发思考(当然,也可以同时既有趣又可怕,并且发人深省)。

后现代主义可以分为“积极”倾向,强调玩乐和自由,以及“消极”倾向,强调混乱和犬儒主义或绝望\cite{ref6}。《魔圆》很大程度上属于后者,并大量运用了后现代主义技巧来讲述它的故事,例如(我们将在下一章更详细讨论)让“魔女”的形象采用外来、侵略性的艺术风格,取代了其熟悉的动画风格。
\paragraph{关于本书}~{}

本书分为三个部分。第一部分是电视动画,十二章每章分别讲述动画《魔圆》的一集,并以捷克剧作家、活动家和政治家 Václav Havel\footnote{译注:瓦茨拉夫·哈维尔(1936年10月5日—2011年12月18日),捷克作家及剧作家,持不同政见者、天鹅绒革命的思想家之一、首任捷克共和国总统。} 的名言片段作为标题,括号里是该集的动画标题。随着本书的展开,读者自会明白我之所以选择引用 Havel 名言的原因。书名也同样:“或许,绝望恰是滋养人类希望的土壤;或许,没有先经历生命的荒谬,就永远找不到其中的意义。\footnote{译注:原文为``Perhaps hopelessness is the very soil that nourishes human hope; perhaps one could never find sense in life without first experiencing its absurdity.'' }\cite{ref7}。 ” 

第一部分之后是第一个插曲《完美一刻》,讨论该系列如何运用了歌德《浮士德》中的叙事元素。

第二部分是最短的部分,每章分别探讨在电视动画播出期间及之后发布的三部漫画衍生作品。这些章节的标题均取自电视动画中的一句台词。

接下来的第二个插曲《奶酪之尸》则探讨在剧场版《魔法少女小圆新篇:叛逆的物语》(以下简称《叛逆》)中,奶酪的含义与炼金术中腐化的概念。

最后一部分由七章组成,探讨《叛逆》这部电影。我只探讨这部是因为前两部是电视动画的总集篇,几乎没有新画面,更谈不上什么新剧情。反观《叛逆》才是电视动画的续作\footnote{电影开头有些模糊,看起来像是一个非常奇特的“平行时间线”,类似于本书第二部分讨论的漫画中探索的时间线,但在电影上映前的宣传材料中提到有两部回顾影片和一个《魔圆》故事的“延续”。在电影本身中,几个角色能够记住最终集中的事件(该集发生在晓美焰重置所有时间线与瓦尔普吉斯之夜战斗之后),例如麻美和焰谈论幽灵,丘比记得焰告诉它,小圆为了创造新系统而牺牲的事情,小圆以“圆神”的形式出现在最后。}。这七章,每章讨论一个《叛逆》中反叛的目标,因此其标题的形式都是“叛逆的\underline{\makebox[1cm]{}}”。

本书是一部分析性的著作,而非导读。剧情、角色和制作细节只在分析相关内容时描述,一般情况下,本书假设读者对所讨论的作品已经熟悉。撰写本书时,在美国电视动画可以通过 DVD、Blu-ray 光盘和 Netflix 与 CrunchyRoll 流媒体服务观看;所讨论的漫画也已经出版,或正以英文翻译出版。由于《叛逆》仅进行了短暂的影院上映,因此要看起来比较困难;不过可以购买附带英文字幕的日本版 Blu-ray 光盘,并且预计会在2015年4月发行带有英文字幕和配音的 Blu-ray 版。鉴于读者对以上这些作品熟悉的假设,本书不对所讨论的作品进行引用,仅在涉及《魔法少女小圆》系列之外的参考文献时进行。

最后,本书的一些章节,尤其是第2章和第9章,深入分析了动画中特别是整个魔法少女类型中的某些女性主义问题。在这一讨论中,请牢记所使用的女性主义理论来源于西方,基于欧洲的文化历史和假设。因此,虽然它适用于西方观众的观看体验,但不一定适用于日本文化——即该动画的创作文化和主要观众。因此,这些章节的讨论主要是从西方角度来分析该动画,不一定能普适用于日本的文化。
\paragraph{致谢}~{}

本书的完成离不开很多人的帮助。首先要感谢我的朋友兼封面设计师 Viga Gadson,正是她最早向我介绍了《魔圆》。其次是 Unnoun,他将我博客上关于该动画的文章不停地转载到各种论坛,让这些文章成为我写作生涯中浏览量最高的内容之一,也促使我写了更多关于《魔圆》的文章,最终汇成了这本书。第三要感谢我的编辑 Kit Paige,本书中学术上的修饰都是她的努力,而其他任何遗留的错误都是因为我的固执,并非她的疏忽。第四要感谢 MarkWatches.net 的 Mark Oshiro 和 SFDebris.com 的 Chuck Sonnenburg,他们对动画每集(极为不同)的评论让我获得了急需的批评性建议。最后但同样重要,感谢我博客 JedABlue.com \footnote{译注:很遗憾,因为一些原因,只能在 Internet Archive 上浏览这些博客网站了。}上的读者们,是他们对我初稿的深刻评论指出了内容的缺陷、需要澄清的部分,或是提供了我未曾考虑过的相反观点。特别要感谢 Alan Jacobs Richardson(01d55),我们关于《叛逆》与 E.T.A. Hoffmann\footnote{译注:E.T.A 霍夫曼(1776年1月24日—1822年6月25日),德国作家及作曲家,是浪漫主义运动的重要人物。作品多神秘怪诞,以夸张手法对现实进行讽刺和揭露,所描写的人际关系的异化和采用的自由联想、内心独白、夸张荒诞、多层次结构等手法与后来的现代主义文学有很深的渊源。}《胡桃夹子和老鼠王》关系的长篇讨论\footnote{相关讨论在我的博客上。}\cite{ref8}启发了第 20 章和第 21 章中的许多内容。
