\introduction{万物的尺度}
《魔法少女小圆》(以下简称《魔圆》,以区别于主角:鹿目圆\footnote{日语习惯是姓氏在前(如:鹿目圆),但《魔圆》官方英译是先名后姓的顺序。为避免混淆,本书沿用剧中顺序。当涉及其他角色或真实人物时,为求统一,亦采用此顺序。}, 后者简称“小圆”\footnote{译注:中文和日文姓氏格式一致,故不做考虑。})内涵极为丰富。表面上看,它是由动画公司 Shaft 制作的魔法少女动画。但深入探究,它既是对该题材的批判与致敬,也是一位迷失于黑暗中的作者寻求自我救赎的尝试,更可视为佛教视角下的《浮士德》重述。它探讨着牺牲与自私、绝望与希望、爱与死亡、腐朽与新生的辩证统一,也包含对同意、特权与性别角色问题的女性主义审视,亦是对未成年女性形象性别歧视和物化现象的考察。

然而,其核心终究是人的故事\footnote{译注:这也是导言标题的含义。来自古希腊哲学家 Protagoras 提出的命题——``人是万物的尺度。"}。故事中人做出抉择(常是错误抉择),历经磨难、奋斗,体验成功与失败。角色间或爱或恨,或漠不关心,甚或情感交织。

本书将探讨诸多宏大主题、哲学命题与象征符号。读者请谨记:一切故事皆关乎人,因故事本质即摹写人之行动\footnote{ “摹仿者所摹仿的对象是行动中的人。”亚里士多德《诗学》第一部分将戏剧与叙事纳入诗学范畴,其所谓“摹仿的对象”即今之“作品主题”。}\cite{ref1}。 
\paragraph{为什么研究《魔圆》?}~{}

在短短十二集(每集仅二十二分钟)的 TV 动画加一部剧场版(虽有三部剧场版,前两部实为 TV 总集篇,改动甚微),《魔圆》却容纳了惊人的故事体量:角色塑造饱满,主题深邃复杂,情感冲击强烈——深沉的悲伤、惊惧、愤怒与喜悦交织。分析这部作品极具文学价值,每次深入研读都能发掘出新意蕴。尤其它本身即是后现代主义力作,以后现代视角剖析尤为契合。
\paragraph{何谓“后现代”?}~{}

“后现代主义”概念繁复,但讨论时必先阐明其要义。哲学上,顾名思义,后现代主义乃现代主义之延伸。

现代主义的核心洞见在于:符号本质是任意的。换言之,能指(表意之物)与所指(被指之意)间并无天然纽带,其关联仅存于观者意识之中。例如,红色六边形意指“停”,并非红、六边、停止三者存在逻辑关联,实因有人规定此形此色表“停”并获众人认可。此即社会建构主义内核:我们理解世界的概念与符号,皆由社会环境与关系构筑而成。故符号意义如社会关系般流动不居,并无固定客观标准\cite{ref2}。

后现代主义视域下,一切艺术(及语言)皆为符号系统。因此,任何作品的意义在某种程度上皆具任意性,乃社会建构之物,非逻辑必然之果。传统艺术倚赖约定俗成的文化符号(如语言、具象图像——以苹果图片代指苹果)规避这一问题。此类符号作为“引导轨道”,帮助受众从身边熟悉的符号起步,开始接触艺术,逐步构建出更多意义。现代主义则拒斥传统,弥漫幻灭与解体感,其艺术常摒弃部分乃至全部“引导轨道”,激进地质疑艺术应具非任意意义的观念。例如,James Joyce\footnote{译注:詹姆斯·乔伊斯(1882年2月2日—1941年1月13日),爱尔兰现代主义作家和诗人。} 的作品以意识流打破句法常规;Mondriaan\footnote{译注:彼得·蒙德里安(1872年3月7日—1944年2月1日),荷兰画家,风格派运动领袖。}的画作以抽象色块取代具象物体。然而矛盾的是,现代主义常试图重建其感知中已失落的秩序\cite{ref3}。例如,Joyce 将平凡一日升华为史诗鏖战,Mondriaan 则将抽象色块纳入规整网格中。

后现代主义反其道而行,反对“社会建构即无意义”之论。现代主义断言:“能指与所指关系任意,故万物无意义。”后现代主义主张:“能指与所指关系任意,故万物可赋义。”因此,后现代作品无需复现虚妄的“失落秩序”。正因解构了意义本源,后现代主义更拥抱流行文化作为艺术表达媒介\footnote{注意,Murfin 与 Ray 的定义侧重消极与实验性,淡化了 Hassan 强调的多元性核心。}\cite{ref4}。现代主义摒弃传统以求新秩序;后现代主义则糅合传统与新异,追寻奇趣之境\footnote{Hassan 提出了十一项交织标准,本文定义主要借鉴其“构成主义”、“ 非原则化”、“反讽”、“狂欢”、“混杂”等理念。}\cite{ref5}。

若说现代艺术拆除“引导轨道”,后现代艺术则刻意凸显“引导轨道”。其核心手法是去语境化与再语境化:将熟悉的能指剥离原境,植入异质语境——如混搭多类型元素、拼贴经典作品、打破第四面墙\cite{ref5}。由此引发的错位感,常被用于制造幽默或恐怖效果,或激发思辨(三者亦可并存)。

后现代主义有“积极”一脉,崇尚游戏与自由;亦有“消极”一支,渲染混乱、犬儒与绝望\cite{ref6}。《魔圆》深植后者土壤,娴熟地运用后现代技法叙事。譬如(下章详述)“魔女”以异质侵凌的艺术风格呈现,强行覆盖动画通常的画风。
\paragraph{本书结构}~{}

本书分三部分。第一部分剖析 TV 动画,十二章对应十二集,每章标题截取捷克剧作家、思想家 Václav Havel\footnote{译注:瓦茨拉夫·哈维尔(1936年10月5日—2011年12月18日),捷克前总统。}名言片段,括号内注明了该集动画标题。选用 Havel 语录的深意,随阅读推进自明。书名亦出自其言:``绝望或是孕育希望的土壤;未经荒诞,难觅生命真义。"\footnote{译注:原文为``Perhaps hopelessness is the very soil that nourishes human hope; perhaps one could never find sense in life without first experiencing its absurdity.'' }\cite{ref7}。

第一部分后设间幕《完美一刻》,探讨本剧对 Goethe《浮士德》叙事元素的运用。

第二部分《外传漫画》篇幅最短,三章分述 TV 动画播映期间及之后推出的三部漫画衍生作。各章标题均引自TV动画台词。

此部分后接第二插曲《牛奶之尸》,解析剧场版《魔法少女小圆【新篇】 叛逆的物语》(下称《叛逆》)中奶酪的隐喻及炼金术“腐化”的概念。

最后一部分共七章,聚焦《叛逆》剧场版。仅选此部是因为前两作系 TV 精编,新画面寥寥,剧情几无新意;而《叛逆》实为 TV 正统续篇\footnote{影片初段看似是一条奇特“平行时间线”,类同第二部分漫画的设定。然而上映的宣发明示:前两部为“回顾”,此部乃“延续”。剧中,麻美与焰谈及“魔兽”;丘比也记得焰告诉过它,小圆为创造新系统牺牲的事;以及小圆最终以“圆环之理”的形态出现在结尾——都证明了角色保有最终话的记忆(在焰重置时间线与魔女之夜决战之后)。}\cite{ref8}。七章分述“叛逆”可能指向的七种对象,故标题皆作“叛逆的\underline{\makebox[1cm]{}}”。

本书为学术分析,非观剧导读。仅在关联分析时简述情节、角色及制作细节,故预设读者熟知原作。成书时,TV 动画在美国可通过 DVD、蓝光碟及 Netflix、CrunchyRoll 流媒体获取;漫画英译版也已或正在美出版。《叛逆》在影院曾上映短暂过,获取较难,但可以购买附英文字幕的日版蓝光碟;美版蓝光碟(含英字与配音)定于2015年4月发行。基于此预设,本书不再标注剧中情节出处,引注仅限《魔圆》系列以外文献。

最后特别说明:本书第二、九等章深入探讨了本剧乃至魔法少女题材中的女性主义议题。相关分析所援引理论植根西方,依托欧洲文化历史与预设。故其适用于西方观众,却未必契合日本文化——即该动画的创作文化和主要受众。此部分论述,当视为西方视角下的解读,其结论未必普适于日本文化。
\paragraph{致谢}~{}

本书成稿仰赖多方襄助。首谢挚友兼封面设计师 Viga Gadson,引我初识《魔圆》。次谢 Unnoun,他将我博客上的剧评持续转至各个论坛,致其成为我笔下的流量之冠,进而催生出更多评述,最终汇集成了本书。三谢责编 Kit Paige,本书学术规范与文字润饰皆赖其功;若有疏漏,皆因我固执,非其失察。四谢 MarkWatches.net 上的 Mark Oshiro 与 SFDebris.com 上的 Chuck Sonnenburg,二人风格迥异的评论帮助我跳脱己见,获得了宝贵的批评意见。最后铭谢博客 JedABlue.com\footnote{译注:原博客因故不可访,今存于 Internet Archive。}上的读者,诸位对初稿的洞见,指正了缺陷、不明之处及我未察的反面观点。尤谢 Alan Jacobs Richardson (01d55),与之就《叛逆》同 E.T.A.Hoffmann\footnote{译注:E.T.A.霍夫曼(1776-1822),德国作家、作曲家。作品多神秘怪诞,以夸张手法对现实进行讽刺和揭露,所描写的人际关系的异化和采用的自由联想、内心独白、夸张荒诞、多层次结构等手法与后来的现代主义文学有很深的渊源。}《胡桃夹子与老鼠王》关联的长篇讨论\footnote{讨论见博客。}\cite{ref9},启发了第二十、二十一章的核心内容。
