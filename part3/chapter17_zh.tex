\chapter{叛逆的爱与救赎}
在《魔圆》的结尾,小圆升华至更高的存在层次,不仅牺牲了生命,还抹消了整个存在,以拯救其他魔法少女免于成为魔女。从焰在动画最后一集对丘比的解释以及《叛逆》中百江渚和沙耶香的评论中可以明显看出,被拯救的魔法少女们仍以某种形式存在于宇宙之外,与小圆同在。无论以何种形式存在,我们知道她们在某种意义上是有意识的,能够做出决定,而且她们同时是魔法少女和魔女(这当然也是她们一直以来的身份)。

在动画及《叛逆》最初的部分中,这种发展被呈现为一种积极的结果,可以说是一个较为圆满的结局。这一点或许让人感到困惑。沙耶香的整个角色线可以说就是让她意识到自我牺牲去拯救他人是错误的;焰拯救小圆的努力同样被描绘为这不断让她们的处境变得更糟。因此,《叛逆》质疑这种救赎也就不足为奇了。

电影第一段展示了魔法少女们的幸福相处,以及她们面对的对手虽具有挑战性但仍可战胜,这一段可以看作是对“小圆世界”的一种讽刺。作为人类痛苦的象征,梦魇是魔兽的扭曲映射。与魔兽类似,梦魇之间也极为相似,但魔兽是令人毛骨悚然的瘦长巨人,梦魇则穿着熊熊套装,手臂还能发射毛绒玩具。击败魔兽可以洁净魔法少女的灵魂宝石,虽然奖励有限,但可以让魔法少女们不再像有魔女的时间线中那样死亡了。然而,在焰的幻想世界中,击败梦魇会产生一种弥漫光辉,净化灵魂宝石。这使得魔法少女们更加积极地想要获得更多这种光辉。此外,我们在小圆创造的魔兽世界中第一次看到了沙耶香的死亡和她队友的哀悼,而焰的梦境世界则保留了沙耶香和她的愿望,并暗示着她和杏子在一起了,顺其自然地让仁美和恭介也能在一起。简而言之,焰的幻想世界比小圆所创造的世界要幸福得多!

它也更滑稽,不仅仅是因为那些熊熊套装。将世俗与诡异相并置,是超现实主义艺术的领域,魔兽和它们徘徊的城市就在这一领域中。脸部缺失或被遮挡的人,正是 Magritte\footnote{勒内·马格里特(1898年11月21日—1967年8月15日),比利时超现实主义画家。因为其超现实主义作品中带有些许诙谐以及许多引人审思的符号语言而闻名。他的作品对于许多观察家对于事先设想现实状况的情况提出挑战,并且影响今日许多插画风格。}在绘画中所展现的惯用手法。同样,魔女虽然更加活泼,但与极端暴力形成对比,无论是魔女对人类和魔法少女的暴力,还是魔法少女对魔女的暴力,这种玩乐带有童稚感的形象与暴力相结合,形成了一种野兽派超现实主义。

然而,梦魇并不涉及暴力行为。它们会破坏财物,但当魔法少女们结束战斗时却没有任何损坏痕迹。面对它们时,魔法少女们使用陷阱和绳缚,或是开火把梦魇引入陷阱,但从未直接攻击梦魇。梦魇的实际击败高度仪式化却又显得随意,开头是由魔法少女们承办的宴会,或是使用类似儿歌的韵律游戏。

无害的暴力、食物、韵律诗、游戏、随意的仪式,这些都是无厘头文学的常见特征。无厘头文学的核心是非传统逻辑,它关注的是那些由最终任意但内部一致的规则所支配的情境(例如游戏、用餐、礼仪);就像梦一样,无厘头文学以一组任意规则替代另一组,并让逻辑自行展开。然而在这种无厘头的世界中,五位魔法少女都幸福快乐。看起来,这样无厘头的世界比小圆创造的魔兽世界更美好。

小圆的净土,她的天堂,同样被描绘为较为劣势的存在。正如我在幕间二中所论述的,小圆的“来世”甚至比焰的梦境世界更差,因为它是一个没有死亡的世界,缺乏腐败、痛苦和朽烂。沙耶香和渚选择与小圆一起概念化,因为她们追寻的东西只能通过腐败和死亡产生,比如沙耶香与杏子的关系,以及渚对奶酪的渴望。

焰的梦境也更加直接地讽刺了小圆的天堂,正如焰没有征得魔法少女们(以及一些普通人类)的同意就把她们拉入自己的世界并将她们困在那里,切断了与外界的联系,人为地让她们感到快乐。她“拯救”了她们,因为她通过对小圆的关爱而开始关心这些人,至少在杏子和麻美的情况下,焰最终信任她们来击杀此岸的魔女\footnote{译注:焰的魔女形态,其性质是背德。背弃时间,反叛时间,指责时间,审判时间。},并开始尊敬甚至可能喜欢上她们。

想要拯救某人必然意味着想要对那人拥有某种权力。通过成为骑士、守护者,沙耶香成为了一个审判者(在列车上面对那两个性别歧视者时,她也成为了处决者)。通过希望成为小圆的守护者,焰最终处于一个不断试图剥夺小圆变成魔法少女选择的位置;而通过希望拯救所有魔法少女免于她们成为魔女的命运,小圆将自己设定为圆神。

换句话说,成为一个救世主(与帮助他人相对,帮助他人需要对方的同意,且帮助者在过程中处于暂时的从属地位)必然意味着在某种程度上是个暴君。救世主的行为没有获得被拯救者同意,因此他们很可能会做错,正如小圆对焰所做的那样。来看看片头曲:焰被描绘成一个石化灰白的形象跪在地上,其他魔法少女们在旁跳舞。她无法融入她们的快乐,她能做到的最接近的事也只是在电影第一段中短暂回归“麻花辫”害羞的形象,但即便如此,她也能感受到某些异样。一旦她的头发再次披散,她在电影剩下的时间里再也没有真正感到快乐,因为她经历了无数的痛苦,只有她记得这些痛苦,这些痛苦从情感上扭曲了她,使她无法被拯救。

相反,她拙劣地模仿小圆,将人们拉入她的迷宫。但这与小圆的行为真的有什么不同吗?小圆牺牲是无私的,而焰牺牲则是自私的吗?哪一种牺牲更大?牺牲存在,还是牺牲灵魂?从未存在过和成为曾珍视的一切的敌人,哪种更糟?

答案显然是:这些都是荒谬的问题。所有的价值都是相对的,哪种更糟完全取决于视角;很可能两位少女都认为她们的牺牲是她们所能做出的最大牺牲,因为小圆非常关心与他人的联系,而焰则更专注于她的目标。

但通过焰行为中小圆扭曲的反映,我们也开始以新的方式看待小圆。一个行为如果让你得到了你想要的一切,还能被视为无私吗?小圆得以与所有人同在;她所有爱的人都安然无恙;击败了所有的魔女;得以成为魔法少女;得以变得意义重大,比任何一个活过的人都更为重要。相比之下,焰选择成为一个“恶魔”,尽力让小圆留在人间,付出的代价是她最珍视的东西,即最终与小圆在一起的机会;现在她们必须成为敌人。那么,难道不是焰才是无私的吗?

当然不是,因为“无私的爱”本身就是一种自相矛盾的说法。这正是描绘焰占有欲的意义所在,并通过这种占有欲揭示出圆的自私。爱一个人意味着想要保护那个人,甚至可能是保护她免于伤害自己。爱意味着想与那个人共度时光,想要那个人也同样渴望与你在一起。如果这种爱以健康的方式表达并且得到了对等的回应,当然爱可以是美好的,无论是浪漫还是其他形式,它都是两人之间最深的纽带。但正如任何一种纽带,它也可以被用来束缚、控制、彰显支配力。这也是虐待关系存在的原因。

在整个动画中,我们看到魔法少女们在承认自己真正的欲望和她们认为自己应该想要的东西之间挣扎。麻美因渴望活下来而饱受折磨,而不是像她认为自己应该的那样,许愿拯救她的父母。沙耶香和杏子是为了别人的利益许愿,而不是为了许愿让这些人感激她们的帮助,结果她们因此遭受了巨大的痛苦。这正是丘比选择少女作为目标的原因,因为从她们拿到第一个玩偶的那一刻起,她们就被灌输要照顾他人,被社会化成认为自己是照料者,负责他人的幸福(119)。社会已经替丘比完成了工作,创造出这样的女孩,她们许下的愿望是社会压力告诉她们该要的东西,而不是她们真正渴望的东西。(当然,最终这一切并不重要;愿望本身已经注定了魔法少女要么变成魔女,要么死亡,只是一个不合适的愿望会加速希望与绝望的循环。)

因此,焰找不到任何办法实现她真正的愿望:与小圆在一起。她转而许愿去照顾圆,第一次是在剧集第10话的“初始”时间线结尾,第二次是在《叛逆》中她成为“恶魔”时。在这两种情况下,她最终都认为唯一的出路是成为“邪恶”。因此,《叛逆》封闭了这个充满循环的动画中最大的循环:晓美焰从一个黑暗的、看似反派的角色演变为一个破坏现状的角色,再到一个黑暗的、看似反派的角色,保持原状。现在她比以往任何时候都更像麻美的黑暗面。