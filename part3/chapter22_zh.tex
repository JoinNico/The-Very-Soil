\chapter{叛逆的我们}
《叛逆》让我们反抗自己的确切时刻大约出现在影片的四分之一处。在此之前,影片展示了一个美好的世界,任何有一丝同情心的观众都必须承认,角色们早已应得这样的幸福。贯穿整部动画,我们跟随这些少女的步伐,伴随她们一起遭受痛苦,并在绝望中祈愿她们能找到某种形式的幸福。

在动画结尾,可以说她们确实达到了某种结局,但不可否认,这个结局是苦乐参半的。焰独自一人,是唯一记得小圆的人。她永远消失了,从未存在过。其余的魔法少女仍在战斗,仍在痛苦中挣扎,最终沉入绝望的深渊,化为魔女;但在她们成为魔女的瞬间,小圆会仁慈地将她们杀死。

沙耶香依旧为一个几乎没有注意到她的男孩而死去。麻美和杏子继续作为魔法少女活跃,我们可以推测麻美的父母依旧死亡,杏子的家人仍然在那场家庭谋杀式自杀中逝去。

这个结局是坦诚的,但并非幸福。它描绘了一个新的、更好的世界诞生,但这世界远非完美。

这种完美世界出现在《叛逆》开头。五位魔法少女都还活着,作为一个团队并肩作战。她们的内部矛盾简化为沙耶香与杏子之间的调侃戏谑。焰因时间旅行造成的心理创伤似乎已痊愈:她又回到了那个羞涩、缺乏自信,但更加愉快、和蔼的戴眼镜、扎着双马尾的自己,而小圆也变得更加欢快和自信,变得更像焰在第10集开头第一次见到的那个样子。

梦魇作为威胁的存在几乎令人发笑。如果仁美的梦魇为例,它们对魔法少女们没有任何身体上的威胁,也没有在心理上折磨她们,甚至可以简化为字面意义上的“萌物”。更重要的是,在杀死梦魇后,它们会释放出大量净化灵魂宝石的光辉,如我之前所提到的,这不仅允许少女们协作战斗,还为她们提供了足够的能量,避免灵魂宝石染黑并死亡。于是,少女们得以发挥魔法力量,视觉效果华丽惊艳,作为一个团队战斗,面对的挑战刚好让她们感到自己有用,但却永远不会遭受动画中曾经经历过的恐怖。

这正是我们,作为观众所共同期望的。我们中的大多数人都明白,原本的结局或许在美学上更好,但有着许多的同人作品,证明了那些或因缺少美学经验、或因过于投入而不在乎的人确实希望这些女孩们拥有更好的结局。而现在,电影如我们所愿,正好给了我们想要的,直到焰开始发现真相。

如同《失乐园》之前所做的那样(详见第十九章),影片巧妙地让我们支持一个正试图摧毁我们乐园的人。焰知道这个美好的世界并不真实,因此我们也知道,当她调查这个世界时,她必将摧毁它。从与杏子的首次对话开始,世界变得越来越不现实,直到她们意识到被困在这座城市时,世界被简化为红色色块和白色线条的抽象图景,唯一可辨认的物体是辆公共汽车。不久之后,焰又变成了我们熟悉的那个摘掉眼镜、直发披肩、冷酷寡言的女孩,熟悉的面孔让我们感到兴奋,但也意味着这个美好的世界正在加速崩塌。

接着,我们看到了动画前几集中预示过的战斗场景,焰和麻美大打出手。随后的战斗在视觉上令人惊叹,焰与麻美都使出了她们各自的力量与武器库中的所有武器。这场战斗既令人兴奋,又充满戏剧性,动画和配乐精美绝伦,但它从根本上是错误的。作为一个场景,这是段冗长的情节,推动剧情发展的作用不大,人物塑造和主题几乎没有任何进展;它虽充满了令人兴奋的元素,但明显是为了迎合观众的需求,属于纯粹的迎合性内容,而这些内容在动画中大多是被故意回避的。随后,焰开枪射击自己的头部,麻美化作缎带,原本的迎合瞬间转变为恐怖。

得到我们想要的,带来的往往是失望与恐惧。影片高潮部分,这一点体现得最为明显:焰和小圆重聚,然而一切都变得极其糟糕,最后结果是所有魔法少女都得以解脱、活着,而小圆也不再需要成为魔法少女;但这个腐败的世界由一个恶魔般的焰统治,她将小圆囚禁了起来。

我们买了票,签订了契约,得到了我们的愿望,而这些愿望最终化为灰烬。欲望不可避免地引向痛苦。

为什么?因为我们可能渴望幸福,但我们需要真相。这并不是说绝望比幸福更接近真相,而是说《魔圆》的真相在于熵与不可避免的衰败,整部动画始终将物理熵与衰败与抑郁和绝望联系在一起。如果要以一个简单、无瑕疵的幸福结局收尾,给予我们未被腐化或夺走的愿望,那将是对作品本身的背叛。

因此,影片迫使我们拒绝自己对这部动画的愿望。那些沉迷于黑暗与高难度的人,必须忍受充满粉丝服务的迎合性内容,这让他们不愿意认同自己对该动画的喜爱。而那些享受这种粉丝服务的人,必须面对它带来的后果。两者都必须应对影片最后一段深具双重意义的情节:焰创造了一个比动画中的世界既更黑暗又更明亮的世界,却比电影的梦境世界更为连贯。

因此,该动画将观众置于魔法少女的境地。追逐我们对动画的愿望,最终导致它变得悲剧性。我们的愿望变成了诅咒,一步步走向深渊,直到影片高潮时,我们发现自己在希望焰魔女撕裂这个世界。然后当她真的这样做时,我们必须承受自己愿望带来的现实。

通过让我们反对自己,并展示我们对动画的愿望如何背叛了我们,影片试图做最后一次努力,将同理心强加给我们。正如该动画在最初几集中所做的那样,它提供了视觉盛宴和粉丝服务来吸引我们,而当陷阱诱饵已布下,它让我们为角色们感到痛苦。更重要的是,它让我们感同身受。这是同理心,而不是同情心。通过将我们置于与她们相似的处境。焰将小圆分裂的那一刻的困惑、疏离与错位感,正是成为魔法少女的感受的一小部分。

我之前提到,这个动画是关于熵与衰败、抑郁与绝望的,确实如此。但很容易忘记,它也关乎其他事物,而通过让我们反对自己,它提醒我们那些被忽视的事物。

这不仅仅是关于佛教、德国文学或魔法少女题材的作品。这不仅仅是关于熵与痛苦,或关于主题的复杂性,或是作者隐含的心理问题。这甚至不仅仅是关于那些由动画师创造、由配音演员赋予生命的角色。这也是关于我们,关于我们的恐惧与希望,关于我们的愿望与绝望,关于我们的同理心。

归根结底,正如开篇,《魔法少女小圆》讲述的,是一个关于人的故事。