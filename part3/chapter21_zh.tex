\chapter{叛逆的丘比}
每个故事背后至少都存在一个叙事者。正如我在五章中所论述的,有理由将丘比部分等同《魔圆》的创作者,尤其是虚渊玄。因此,尽管焰在电影结尾声称自己变成了“恶”,与小圆在动画结尾化身为“希望”相类似,但影片中的两个早期场景暗示我们应该超越这些表象,找到真正的反派,从而发现反抗的真正暴君。

在电影的相对开头,焰开始意识到魔法少女们的记忆被篡改,她发现她们被封印的空间类似于魔女的结界。认出贝贝是魔女夏洛特后,她推断是贝贝的作为。接着出现了一个场景:小圆、麻美和焰围坐在麻美家的餐桌旁,桌子同在动画中一样,呈等腰三角形。焰y用小圆的评论作契机,开始询问麻美关于贝贝的来历,贝贝对此则表现出困惑和担忧。整个场景中,焰试图证实她的怀疑,即贝贝是封印空间的创造者,是这个故事的反派。但从多个俯视镜头以及桌子中线的角度来看,桌子始终指向丘比,尤其是在那些强调桌子指向丘比的镜头中,它越过了贝贝,直接指向丘比。

相应地,在影片结尾,魔法少女们似乎最初是在对抗焰的使魔军队。然而,当她们成功打破结界后,巨大的丘比们从天空(她们的隔离区)俯视下来,那才是囚禁魔法少女们(包括焰在内)的真正监狱。这再次提醒我们,要超越焰,看到真正的敌人。
%TODO:“红润的脸颊” 红润的糖果玩偶???
在上一章中,我简要讨论了 E.T.A.Hoffmann 的《胡桃夹子与老鼠王》。在这里可以更全面地展开讨论,因为影片中的最终战斗在许多方面模仿了故事中玛丽的玩具与老鼠王之间的战斗:玩具们由一个下颌骨断裂的生物指挥,从柜子里走出来,与邪恶的类鼠生物战斗。考虑到老鼠王有七个头和七顶王冠(这些特征在故事中被明确提及),明显让人联想到《启示录》第十三章中的七头十角兽\footnote{译注:七首十角的大红龙,它是魔鬼撒旦的化身,为欲吞食身穿太阳的妇人所生之子,而在争战中被打落在地,后为天使捉住,以大链捆缚,丢入无底坑,禁锢一千年。}。但在《魔圆》中,这多个头共享一个身体的形象反映在丘比身上则是,单一意识通过多个身体表现出来。就像老鼠王是玛丽(玛利亚)的敌人,兽是神的敌人,丘比是小圆(观音)的敌人,因此充当她骑士角色的焰便是胡桃夹子。最后,老鼠王要求用玛丽的糖果和玩具作为赎金,以饶恕胡桃夹子,就像丘比要求魔法少女们牺牲她们的生命和灵魂来满足它对能量的需求,食物是所有生物最基本的能量来源。(值得注意的是,这还包括一种可以食用的玩偶,是玛丽最喜欢的玩具,与胡桃夹子的真身,“真正的”朵谢梅的侄子极为相似,尤其是在“红润的脸颊”这句话中。如果胡桃夹子代表了现在的焰,那么红润的糖果玩偶则代表了最初的焰,那位小圆救下的弱小女孩,最终却无意间将她引入了丘比的轨道。)

最有趣的是,在《胡桃夹子与老鼠王》中,叙事者这个角色充满了模糊性。朵谢梅把胡桃夹子作为礼物送给玛丽,并告诉她,自己起源和诅咒的故事,但他多次暗示自己并不值得信任,甚至可能与老鼠王结盟。最明显的是,他默许或导致了时钟的敲响,召唤老鼠王进入玛丽的房间,迫使胡桃夹子召唤玩具军队作战。后来,虽然朵谢梅在给受伤的玛丽讲故事时显得很友善,试图安慰她,但他却轻视了玛丽关于她目睹玩具战斗和胡桃夹子击败老鼠王的说法,称其为梦幻与想象。

朵谢梅也把自己插入了胡桃夹子诅咒的故事中,正如丘比插入了焰的魔女结界(这只有在结界存在于丘比的隔离区内时才有可能)。他讲述的故事中,他几乎成了主角,踏上了拯救公主碧丽波的冒险。然而,随着读者对“真实”朵谢梅的怀疑,内在故事中的朵谢梅同样可以被解读为一个经典的童话反派,为了自保而牺牲了自己的侄子。我们只能凭朵谢梅的一面之词来了解仪式为何失败,或许它本就是为了把诅咒从公主转移到胡桃夹子身上?他的故事不可信,尽管玛丽的经历发生在那些可以轻易被当作梦境的时间和地点,但她的经历总是被以与“现实”事件相同的方式叙述。而朵谢梅的故事则是由角色讲述出来的,用来逗小孩开心的虚构故事。在《胡桃夹子老鼠王》的背景下,朵谢梅的胡桃夹子故事比起玛丽的梦境更不可能是真实的。结局进一步暗示玛丽的梦比朵谢梅的故事更加“真实”,因为她能够进入胡桃夹子的世界,而他对朵谢梅的态度则充满轻蔑。
%TODO:发条湖???
朵谢梅是否比他表现出的更具反派色彩,取决于读者的偏好和解读。然而,显然胡桃夹子在玛丽进入他的世界并治愈他时(就像小圆穿越结界,拯救焰脱离魔女身份一样),公开对朵谢梅不屑一顾,警告玛丽他无法兑现为她打造发条湖的承诺,并表示她的创造力超越了他。而最终,玛丽的幸福结局是与胡桃夹子一起逃离“现实”,进入胡桃夹子的世界。

尽管剧情细节并不完全契合,尤其是事件顺序,但角色对位关系却十分清晰。焰负责创造她结界中的意象,而与她关联的胡桃夹子主题表明了她将自己视为那个角色,背负起公主的诅咒,并期望有一天能被玛丽拯救。小圆既是公主,也是玛丽(值得注意的是,在胡桃夹子的世界里,玛丽在池中看到自己的倒影是公主,两者在 Hoffmann 的故事中很可能是同一个角色),而丘比则融合了老鼠王和朵谢梅的元素。

需要指出,这些角色映射并不一定反映他们的真实本质,因为这些都是从魔女结界中的意象得来的,而结界早已被确立为反映魔女精神状态的幻觉迷宫,替代并扭曲了现实。但这一切深刻揭示了焰的动机;对玛丽来说,幸福的结局并不是逃往某种涅槃或天堂,而是逃入胡桃夹子的世界,在那里他们可以一起生活。焰创造了她的世界,以便与她的玛丽,小圆,在一起,于是进行了一场邪恶而绝望的尝试,完成她的胡桃夹子故事。她变得比以往更加黑暗和丑陋。唯一的解救方式就是有人能爱她原本的样子。换句话说,就是小圆要了解焰所做的一切,了解她变成了什么,依然不放弃她。

然而,朵谢梅或许会笑到最后。作为故事中的叙事者,他必然在某种程度上反映了他自己的叙事者 Hoffmann ,而玛丽与胡桃夹子在故事结尾时仍处于 Hoffmann 的掌控之中,表明他可以继续写下他们的故事,直到选择停止。而丘比就是朵谢梅,就是 Hoffmann ,就是虚渊玄;他看似被打败,在动画结局中它也曾看似被打败。最后一幕它凝视的眼睛提醒我们:只要我们还在看,它就在看。只要我们愿意接受更多关于小圆、焰及其他人的故事,丘比就仍然有威胁她们的力量。