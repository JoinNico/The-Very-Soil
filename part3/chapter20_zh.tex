\chapter{叛逆的焰}
我们之前简要提到过,在《魔圆》系列电影中有一个反复出现的意象,即在一片草地上的两把白色椅子,焰和小圆坐在椅子上并肩而坐。在前两部电影的片头中,她们相互依偎,甜美可爱,纯洁无邪。而到了第三部电影,这一幕迅速变得不再那么甜美。

在《叛逆》第二幕高潮部分,焰即将变成魔女时,她再次回到了这张椅子所在的场景处。然而这次,小圆从椅子上站起,向一旁跌落,溅成了一摊粉色的污迹,而焰只能无助地伸手去抓她。焰蹲在一旁,双眼睁大,充满震惊与恐惧,周围则是一群高大、瘦长的焰俯视着她。接着,一只巨大、狂怒的焰之拳砸向了蹲在小圆的残迹旁哭泣的焰。

小圆已经消失,她的身份被替换为一种模糊的抽象存在。焰失败了。现在,焰站在焰的面前,开始审判她自己,发现自己有罪。愤怒与悲伤终于爆发,击碎了自我,变成了一个同样抽象且深奥的存在:魔女。

同沙耶香、之前的所有魔女一样,焰的魔女形态是无尽的自我惩罚与循环,一场心理剧,在其中反复上演将她引向绝望的事件,并因这些失败而惩罚自己。她试图射杀自己,而射中的那个“自我”变成了她不得不亲手结束生命的小圆。她不能死,也不配像小圆那样死去,因为她没能拯救小圆。

不仅仅是没能拯救小圆;焰是让小圆消失的原因。她在时间中的循环使得小圆有能力成为“圆环之理”,从而将小圆从现实中抹除。而她与丘比的对话,给了孵化者们足够的信息来构建如今束在小圆身上的圈套,而它们正是利用焰来制造这个圈套。焰是小圆最大的弱点。

焰的魔女形态是最具象征意义的。她戴着尖顶黑帽,拥有突出的鼻子和下巴,除了她是一个数百英尺高的骷髅之外,看起来相当典型的万圣节女巫装扮。焰了解魔女以及她们的来历,然而她依然无法躲开这个圈套,甚至故意去拥抱它,以此来挫败丘比。与沙耶香不同,沙耶香自认为是骑士,因此即使变成魔女仍保留骑士的外形,而焰知道她正在选择变成什么样子。同样,她有意牺牲自己,以此告诉丘比,她相信麻美和杏子会杀了她。因此,她的使魔将其引向断头台,象征着她的牺牲,审判着她的罪行。

同时,她的周围也充满了胡桃夹子相关的意象。她的使魔之一是带有胡桃夹子下颚的大牙齿\footnote{译注:Luiselotte 的职责是驱逐老鼠。她们是狩猎着白色老鼠的蛀牙骑兵;其中亦有部分协助着锡兵队,驱逐那些无理阻碍送葬之人。}。另一个使魔类似玩具士兵\footnote{译注:Lotte 的职责是执行处决。组成送葬队伍的士兵,将魔女带到断头台边。憎恨愚者的愚者,被它们所无情断罪。笨重的头脑不知变通,亦完全不听魔女所言。最讨厌的东西是白色的老鼠。},但她们戴着高高的毛帽,像传统的圣诞胡桃夹子(在原作《胡桃夹子与老鼠王》中,孩子们立刻认出胡桃夹子是士兵(127))。当她第一次意识到自己就是困住魔法少女迷宫中的魔女时,一个露出微笑的嘴巴正咬着核桃的形象出现了。她失去了头的一半,只剩下颚,这与 E.T.A. Hoffmann 的故事《胡桃夹子与老鼠王》和 Tchaikovsky\footnote{译注:彼得·伊里奇·柴可夫斯基(1840年5月7日—1893年11月6日),俄国作曲家。其许多作品在有生之年就享誉国际,至今仍是最重要的浪漫主义音乐代表之一。} 著名的芭蕾剧《胡桃夹子》中的胡桃夹子形成了呼应,后者也失去了下颚。她许多使魔的玩偶外形(尤其是那些破布娃娃,在补充资料中被命名为芭蕾剧中主角的名字“Clara\footnote{译注:Clara Dolls 的职责是哭泣。她们是来自伪街的孩子们,假扮眼泪的换装人偶,从而让葬礼更加喧闹。到场的人偶有:自大、阴沉、欺骗、冷血、自私、毒舌、愚钝、嫉妒、懒惰、虚荣、懦弱、愚蠢、自卑、固执;最后的“爱”却并未来临。拥有着不亚于魔法少女的力量。}”(128))、与使魔从玻璃橱柜中破裂而出的画面、以及时钟机制的突显都让人联想到《胡桃夹子与老鼠王》的原作,在故事中,胡桃夹子在钟声敲响时,率领橱柜中的玩具军队对抗老鼠王的军队。

在最基础的层面上,没有下颚的胡桃夹子象征着无用,一个失去了功能的物品。更深的联系在于,如果我们回想起 Hoffmann 故事中胡桃夹子的来历故事。他曾是被选中的人,被预言为唯一能拯救被老鼠王诅咒的公主的人。他必须完成一个复杂的仪式来救她,但在最后关头,却被鼠王绊倒,诅咒反而落到了他身上。这正如焰,在她相信自己已经帮助小圆逃脱了命运后放松警惕,却因为丘比的干预而最终失败。换句话说,这又是一种自责和惩罚自己的方式。

然而,魔法少女们拒绝配合。她们拒绝与焰一起审判她自己。她们拒绝憎恨她,拒绝杀死她。相反,她们努力解救她,打破迷宫与孵化者的圈套,以便小圆能将她带往魔法少女的天国。尽管焰疯狂恳求,她们仍坚持原谅她,拒绝审判,并要求她也拒绝这个审判。她们希望焰能原谅自己,并解救自己。

但从电影一开始,焰就在与焰自己战斗。在电影的第一幕中,焰试图寻找那个神秘且看不见的暴君,统治着魔法少女们所处的这个看似幸福的世界,意图将其摧毁。正是发现自己就是那个暴君,使得焰对自己下了诅咒,并完全变成了魔女;这一切都是她对自己的反抗。

这种反抗在影片结束时仍未终结。焰称自己为邪恶,并拥抱了那身穿着暴露、带有黑翼的恶魔形象。然而,说“我是邪恶的”与说“我应当被惩罚”有何区别?这不过是她内疚的另一种表达形式,是她折磨自己的新方式。

她已将自己提升为一个宇宙级别的存在,一个具有几乎无限物质现实操控能力的造物主:她可以篡改沙耶香的记忆,让死者复活,重塑小圆家的整个历史来扭转第一集的事件。尽管如此,她选择了创造一个自己孤独无依的世界,断绝了与其他魔法少女的友谊,选择让沙耶香在记忆消失前斥责她。

在她所创造的新现实中,焰唯一真正展现出的情感是恐慌,发生在小圆预示要重新与“圆环之理”连接时。换句话说,小圆几乎唤回了一个充满希望与宽恕的宇宙级存在,能够结束焰的苦难。然而,焰绝对不能允许这发生;她必须因未能拯救小圆、让事情对小圆来说变得更糟而受苦。她必须永远将小圆置于一种纯洁与安全的状态,切断她的潜力,因为保护小圆是焰对“善”的唯一理解。因此,未能保护小圆则是她对“恶”的唯一定义。

这一切本可以结束。如果其他魔法少女杀死了她,焰也将不再受任何惩罚,痛苦也将终止。但她们以残酷的仁慈强迫她继续存在,迫使她找到另一种方式来继续保护小圆,并惩罚自己。她因此恨她们,恨她们未能像她恨自己那样憎恨她。在她的新世界中,她以消极攻击的方式表达自己的仇恨。在麻美身后摔碎了一个茶杯,回忆起麻美与夏洛特战斗时的死亡;在沙耶香记忆消逝时嘲弄她,模仿她变成魔女时的自我丧失;骗杏子浪费食物。

在片尾彩蛋中,她从一旁的悬崖上跃下,旁边是一把白色椅子,呼应了之前小圆从椅子上倾倒的场景。她对自己的仇恨没有改变。唯一改变的是,现在她有了让魔法少女们也恨她的能力,将自己定位为她们的敌人,希望她们最终能完成这一工作。

自从电影上映以来,关于焰的新身份一直存在争论。她是英雄还是反派?这里有了答案:是的,焰既是《叛逆的物语》的反派,也是与反派战斗的英雄。

这里还有另一个答案:不,其实焰是反派的受害者,英雄必须拯救她。

她的魔女结界已经扩展到了宇宙的范围。现在,她就是整个故事。