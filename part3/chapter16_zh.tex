\chapter{叛逆的分析}
要讨论《剧场版 魔法少女小圆 [新篇] 叛逆的物语》并不容易。与动画一样,这部电影在符号学上极为复杂(也就是说,它的画面和事件可以引发许多不同的解读)。然而,与动画大部分故事场景,普通又刻板,以与魔女迷宫复杂元素形成对比不同的是,剧场版的故事场景在视觉上显得更为复杂。换句话说,剧场版不仅每个画面都像动画中的内容一样需要解读,而且它所包含的图像信息量也远远超过了其五个半小时电视动画所暗示的量!

因此,分析这部电影几乎是让人不知从何下手。要将电影作为一个有机整体来解读,至少在章节长度的文本当中几乎是不可能的。为了合理解读它,要么选择一个主题,追踪它在电影中的发展,但这意味着只能处理单个主题,最多也只能看到其与其它主题的交互;要么选择一个场景,细致分析其复杂性,但这意味着除了一些与所选场景有关联的内容,其它的就被忽视了。

比如,我们可以考虑一个相对简单的问题:影片标题中,“叛逆”到底指什么?是丘比对小圆新世界的反叛?是焰对迷宫中世界的反抗,尽管她并没有意识到自己是创造者?还是焰对小圆的叛逆?或者,影片本身就是一种叛逆,如果是,那它反抗的又是谁或什么呢?

即使是像“标题是什么意思?”这样看似简单的问题,也只会引出更多疑惑,而且每个问题都有多种合理的解答,每个解答也都可以成为一篇独立文章。因此,本书剩余的章节将由一系列关于“叛逆”的文章组成。有的文章可能是对单个场景的分析,有的可能追踪某一主题,或是探讨某个角色的转变。但无论是哪种方式,它们都试图回答同一个问题:“叛逆的是谁或什么?”我认为这种方法是唯一可行的途径,因为这部电影无法简单分析。

而反抗的近义词是……你应该明白了\footnote{译注:原文为``And a near-synonym for defiance is... well, you get the idea.''译者认为这是作者的搞怪,``defiance(反抗)''与``rebellion(叛逆)''同义,故做此译。}。

那么,让我们来看一个具体又简短的场景,这很好地展示了这部电影的分析难度,即片尾彩蛋。在彩蛋前的片尾中,电影情节以高度风格化的形式展现出来,焰和小圆被缓缓播放的职员表分隔开,然而在最后,她们手牵手一起奔向远方。考虑到焰在片尾前最后几场戏中的形象,这一结局显得出人意料地充满希望。这是对未来的预示,还是说只是焰的梦境?她们消失在远方是表明她们即将逃脱束缚,还是象征她们在一起的可能性正在消失?

这并不重要,因为彩蛋否定了片尾动画。(或者它真的否定了吗?如果片尾是焰的梦境,而彩蛋是现实,或者片尾是预示,而彩蛋则是她的恐惧……)我们可以先讨论一下彩蛋的通常功能。它通常出现在大制作的动作系列片(漫威电影宇宙已将彩蛋提升到了艺术形式的层次),或是喜剧片中。彩蛋通常有两种功能:要么收尾之前影片中设置的笑话(可能最好的例子是《空前绝后满天飞》中的出租车乘客),要么引发观众对下一部作品的期待,并暗示其情节走向(比如《复仇者联盟》Samuel L. Jackson\footnote{塞缪尔·杰克逊(1948年12月21日—),美国男演员及监制。}的现身)。

然而,这里的彩蛋似乎既不属于前者,也不属于后者。它的主要作用是引发疑问,并对焰在片尾前最后几场戏中选择的自我呈现方式产生质疑。诚然,从某种意义上讲,它可以被视为一种“画龙点睛”,因为它回收了电影中反复出现的那两把椅子意象的伏笔。在前两部总集篇电影片头中,我们看到小圆和焰并肩坐在白色椅子上,置身于一片花海之中,搂抱着亲昵交谈。在《叛逆》彩蛋中,焰独自坐在类似的椅子上,椅子处在悬崖边缘。椅子、悬崖和半个月亮排成一线,一幅被切割成两半的画面,仿佛一半月亮一半世界,包括小圆和她的椅子,已被切实地切割掉,替换成了空虚的黑暗。

焰的决意让小圆也成为她的敌人?她的悔恨?还是对观众的一种残酷提醒,提醒他们已失去的东西?

电影最后的几场戏中,焰似乎几乎完全掌控局面。一支由使魔组成的军队听命于她;她可以重写沙耶香的记忆,切断她与人鱼魔女形态的联系;她甚至可以阻止小圆恢复她的“佛性”,即“圆环之理”。焰是这个新世界的创造者,早在电影的某个时刻,她便重写了现实;因此,推测她现在是宇宙中最强大的存在并不为过(但她与“圆环之理”相比如何,尚有争议,因为她似乎无法完全控制的两个实体正是她自己和小圆,注意她在片尾前不久,小圆记起自己的佛性时显而易见的惊慌)。

那么,为什么焰在彩蛋中看到丘比的靠近时会显得惊讶呢?她脸上的表情可以解读为不安或希望;考虑到椅子的象征,她是否一时以为那是小圆?她希望是小圆,还是害怕是小圆?她怎么会不知道那是丘比呢?

再来看看丘比的状态:凌乱、颤抖。在动画和电影中,特写丘比眼睛常常提醒着观众,它正在观察一切,这也让它成为了一个更加不祥和邪恶的存在。然而,这次的特写却展示出它凌乱的皮毛、黯淡的眼睛以及明显的颤抖。丘比第一次展现出了情感上的变化,但这反而让观众更加不安,即使是丘比,在这个场景中,也被转化为某种“错误”的存在。它不再是一个邪恶的角色,而是一个可怜的形象,被焰重写宇宙的力量击败、打破。对它和它的同类来说,这是最坏的情况:虽然小圆害怕、不再信任它,但她几乎没有能力去憎恨;而焰则不同,充满了愤怒和悲伤,她对孵化者们,尤其是对见泷原市的这只丘比,完全有可能发泄这种情绪。

但是彩蛋的逻辑表明,丘比眼睛特写是某种预示。它是这一场景中最具彩蛋特色的镜头,令人联想到恐怖电影中的结尾,通常以人们以为已死的凶手突然睁开眼睛为结束。而不幸的是,它依然无法捉摸不透,是在策划对焰的反击?还是仅仅在观察并等待机会?或者它真的已经被击垮,那可怜的形象表明它作为反派的角色已被焰剥夺?

然后是焰的舞蹈,与她新灵魂宝石共舞,舞蹈风格和音乐都让人联想起电影开头她的芭蕾式变身场景。这颗宝石形似国际象棋中的国王,一个几乎没有力量但却是棋局中最重要的棋子。这是否意味着焰在电影最后成为了棋盘上的“皇后”,与她世界中最重要但现在已经无力的小圆共舞?这种解读是可以得到支持的:电影前面焰的新灵魂宝石是由她旧宝石的碎片和与小圆头发颜色相同的线轴所制成的。但这只有在线轴代表小圆本人或她与焰的联系时才成立。另一种可能的解读是:这根线象征了化身为人类的小圆与无所不在、难以捉摸的小圆之间的联系。在这种情况下,焰并不是在怀念她的“另一半”,而是在沉浸于她所构建的牢笼之中。换句话说,这两种解读的区别在于,是将焰视作在痛苦和困惑中强装镇定,还是视作一个阴森、控制欲极强的跟踪狂。

在她的舞蹈结束时,焰侧身倾倒,几乎掉下悬崖。她坠落时的姿态让人想起焰变成魔女时,小圆从椅子上以类似方式侧身倾倒的画面,这显然是呼应了小圆的自我牺牲,也体现了焰对未能阻止她的悔恨逐渐加深的过程。那么,焰是在通过复制小圆的行动来试图与她重聚吗?是在牺牲自己以确保小圆不再需要这么做吗?还是在嘲讽小圆的牺牲,以此标志她已经堕落到了一个地步,甚至曾驱使她堕落的爱也变得不再重要?又或是,这是一个陷入绝望的角色做出的徒劳自杀举动(不清楚如果焰死了,宇宙会发生什么,但几乎可以确定至少小圆会重新连接到焰试图阻止她恢复的佛性)?或者(可能也是或其他解读)这是对塔罗牌中愚者的引用?愚者牌通常描绘成一个愚人正准备踏下悬崖,身边伴随着一只白色小动物。

当然,我们可以详细探讨这些问题,以及其他(例如,月亮恰好处于新月与满月之间的状态,这对应了沙耶香变成魔女时的新月和焰变成魔女时的满月,月亮这一变化有什么意义?)。这些探讨可能需要成千上万字,且最终结论只会是,这个场景有意保持模棱两可,不过这样的探讨是可以进行的,这并不是重点。重点是,这只是一个两小时电影中的九十秒片段,甚至还不是视觉或符号学上最密集的九十秒(那些我认为是焰变成魔女和她与小圆共同摧毁孵化者屏障的场景)。

不,重点是:这部电影极其复杂,并且充满歧义,对分析者提出了极大的挑战。

很好。那我们就来尝试解读吧。