\chapter{叛逆的神明}
在古代地中海地区有一个流传的神话。故事讲述一个名为“闪亮之子”(希伯来语:Helel,希腊语:Phaethon)的存在试图篡夺太阳或至高神的位置,结果因其僭越行为而被惩罚、堕落。这一神话在我们的文化中颇为熟悉,主要是因为希腊版本的影响。相比之下,闪米特\footnote{译注:闪米特人起源于阿拉伯半岛和叙利亚沙漠的游牧民族,词汇灵感来自《圣经》诺亚的长子 Shem(闪)。}版本较不为人知,部分原因是因为其中一份书面记载在翻译中丢失了:

“明亮之星,早晨之子啊,你何竟从天坠落?你这攻败列国的何竟被砍倒在地上?你心里曾说:‘我要升到天上;我要高举我的宝座在神众星以上;我要坐在聚会的山上,在北方的极处。我要升到高云之上;我要与至上者同等。’然而,你必坠落阴间,到坑中极深之处。”(122)

在这里,英文中的“晨星\footnote{译注:也即中译本中的“明亮之星”。}”是用来翻译希伯来语中的``Helel''。我们可以很容易想象这个神话的情景:天上最明亮的星星拒绝与其他星星分享天空,试图在黎明前跳到天空最高处。然而,日出时它便被太阳的光辉抹去,这种宇宙间的僭越行为每日重复,像是一个永无止境的循环。

可能大家更熟悉另一种翻译,即《钦定版圣经》\footnote{译注:《钦定版圣经》又称《詹姆斯王圣经》,由苏格兰及英格兰国王詹姆斯六世及一世的命令下翻译的英文版本圣经,于1611年出版,自诞生至今一直都是英语国家极受推崇的圣经译本。}中的“路西法(Lucifer)”,是“晨星”在中世纪拉丁语中的名称。

这就是圣经关于路西法的全部故事。其余的都是民间传说,也就是所谓的“同人创作”。例如,路西法曾是天使,他是《约伯记》\footnote{译注:《约伯记》是《希伯来圣经》的第18本书、基督教《旧约圣经·诗歌智慧书》的第一卷,第一部诗篇性著作。约伯这个名字的含义是“仇视的对象”。}中的撒旦,他是《创世纪》\footnote{译注:《创世纪》是《希伯来圣经》的第一卷书,开篇之作。介绍了宇宙的起源(上帝创造天地),人类的起源(上帝创造了亚当和夏娃)和犹太民族的起源,以及犹太民族祖先生活的足迹。本书也是上帝全部计划中的开始,展示了上帝的创造是怎样的完美,人类是怎样堕落的,一个民族是如何被上帝拣选发展壮大的。}中引诱夏娃的蛇,他是《启示录》\footnote{译注:《启示录》是《新约圣经》收录的最后一个作品,写作时间约在公元90至95年。内容主要是对未来的预警,包括对世界末日的预言:接二连三的大灾难,世界朝向毁灭发展的末日光景,并描述最后审判,重点放在耶稣的再来。}中的兽或龙;这些内容在圣经中其实并未明确提及。圣经只讲述了一个骄傲的存在试图上升、最后被打落的故事。(当然,这并没有问题。宗教文本只是宗教这个复杂的思想、行为和制度体系中的一部分。)

如果我们在《叛逆》中寻找这样一个具有僭越行为的角色,那么丘比无疑是这个形象。它明确表示,将焰置于屏障中的目的是“干预”,即控制并篡夺创造现行宇宙的小圆。为此,它被小圆的使者,焰,愤怒地惩罚。结果就是,新的宇宙似乎暗示着丘比的力量几乎被焰作为主动参与的造物主彻底剥夺。

然而,如果我们借鉴《失乐园》中著名的“同人创作”版本,我们可以得出另一种解读。在讨论 Milton\footnote{译注:约翰·弥尔顿(1608年12月9日—1674年11月8日)英国诗人、政论家,民主斗士,被称为英国文学史上伟大的六位诗人之一。代表作品有长诗《失乐园》、《复乐园》和《力士参孙》。}的史诗《失乐园》时,首先要理解史诗的定义。作为一种文学体裁,史诗大多可以追溯到荷马的作品。一般来说,史诗的定义包括主题和结构上的几个共同特征,最重要的是:它涉及国家、宇宙或全球范围内的事件;描写一位超凡脱俗、通常拥有超自然能力的英雄的事迹;并使用一种独特的风格,使之超越普通的叙述形式。此外,史诗通常以祈愿和主题声明开篇,故事从中途开始(in medias res\footnote{译注:in medias res 意为在事件中、从中间开始,也是文学术语:拦腰法。其是文学与艺术的叙事手法,故事从某个中间点开始阐述,而不是从最初。拦腰法为了填补背景故事,经常必须伴随使用倒叙和跳叙以解说早前的事件、纠葛、登场人物、故事背景。最常见的情况,就是主轴故事的前传、外传、衍生。}),并包含冗长的独白,至少会有一段跳叙,描述故事开头之前的事件。

《失乐园》之所以如此引人入胜,部分原因在于撒旦这个角色:他是一个卑鄙无耻的反派,撒谎、欺骗、自私自利的操纵者;但他同时也是故事的史诗英雄,围绕他的冒险展开叙述。他极具魅力,擅长说服,甚至让一些著名的评论家(如 Blake,认为 Milton 无意中站在了“魔鬼一方”(125))。然而,撒旦的目的却是征服世界、奴役和灭绝人类!(而且他成功了。他的儿子兼孙子“死亡”和女儿兼妻子“罪恶”在故事结尾建造了一座通往地球的桥梁,人类因“堕落”而受到惩罚,亚当和夏娃的后代被赋予了死亡,从此人类都注定一死。)

这些矛盾使撒旦在某种意义上成为一个道德上暧昧的角色。在结构上他是英雄,但在叙事中他却是反派。那么,这与《叛逆》有何关系呢?

想想看:《叛逆》以对“轮回之法”的祈愿和声明开篇,提出了在无望的世界中循环绝望以及逃离虚无的主题。观众被抛入一个与动画结局完全脱节的境况中,直到电影的后半段,才通过丘比的冗长解释了解到发生了什么。这位女主角不仅是拥有魔法的少女,她的力量远超普通魔法少女,因为她是困住剧中所有角色的迷宫中的魔女。最终,她通过爱,变成了恶魔。整个故事因此扩展到了宇宙的维度。至于与普通动画电影不同的独特风格,详见第十六章。

因此,《叛逆》是一部史诗。更重要的是,这是一部关于焰如何从第十二集里小圆的“最好的朋友”变成电影结尾时将她视为敌人的史诗。换句话说,这是一部关于小圆魔法少女宇宙中最忠诚的追随者堕落成恶魔的史诗,她同时征服了物质宇宙,明确表示目标是引导它走向毁灭;然而,她的道德状态仍然充满歧义,至今在互联网上关于这些讨论仍然不绝于耳。

正如动画不仅仅是一个浮士德式交易的故事,而且在许多方面重述了歌德的《浮士德》故事,包括一些相对隐晦的元素,比如时间旅行一样,《叛逆》不仅仅是一个关于僭越与堕落的故事;它在结构上以及通过史诗英雄与道德堕落的并置,颇似《失乐园》。