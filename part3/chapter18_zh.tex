\chapter{叛逆的小圆}
名符其实。

在《叛逆》电影中,被代入焰迷宫的人物是有规律的。魔法少女们作为焰生活中的重要人物,出现当然是合理的,至少在最近的时间线里,她们是焰的队友。小圆的家人也在其中,虽然有些勉强,但因为他们对小圆很重要,而小圆对焰很重要,所以也不是完全没有理由。更牵强的是仁美和恭介,然而仁美对小圆重要,恭介对仁美重要,所以也能理解。然而,焰将和子(班主任)和中泽(普通同学)带入迷宫又有什么可能的理由呢?她确实将他们都带入了迷宫。当焰在数学课上开始怀疑周围人的真实性时,中泽和其他魔法少女是仅有的几个有正常面孔的人,最终当迷宫被打破时,中泽和和子都被描绘为昏倒在沙发上。

答案隐藏在民间传说中:一个知道某人名字的女巫可以利用这个名字对那个人施展魔法(120)。考虑一下被带入迷宫的人,再考虑一下整个动画。被焰带入结界的人也就是在系列中焰可能听到或看到的名字,而焰很可能知道他们的名字。(虽然小圆父亲的名字并没有在荧幕上告知,但小圆叫他“爸爸”,而剧中又没有其他还在世的父亲出场,“爸爸”这个称呼足以作为名字发挥作用。)

之所以“名符其实”,是因为在魔法逻辑(魔法逻辑在很大程度上是叙事逻辑)中,没有符号与被指称对象的区别。在某种意义上,名字就是被命名之物,因此操控名字就是操控事物(121)。由此推断,如果两样东西有同样的名字,那么在某种意义上它们必须相同,一个可以替代另一个。

这一切绕来绕去,最终就是在说,当焰抓住小圆的手臂,将小圆女孩的形象从概念小圆中撕裂出来时,这不仅仅是焰对小圆的反抗,也是《叛逆》对《魔圆》的反抗。

为什么电影不该反抗动画呢?曾几何时,如果一个人想讲故事,他们就会站起来讲述。口述故事是短暂的,每次讲述都会有一点改变。的确会有一些传统,但讲述者可以确信,他们的创意变化不会被视为错误或背叛,而是作为一种高超的表演润色而被接受。

大众识字给这种形式的讲故事带来了致命一击,而广播、电影和电视则让其轰然倒塌。这种讲故事的方式仍然存在(因为没有任何一种艺术形式会真正消亡),但只作为一种好奇心的对象,成为参加文艺复兴节时惊叹事物,或带孩子去公共图书馆听的东西。大多数时候,当我们想要看故事时,我们会寻找已经包装好的书或 DVD 碟片。

当作者想要讲述一个系列或轮回故事,重复使用同样角色或设定时,这种现象带来了挑战。作者希望探索和创造,而在口述故事时代他们可以自由地这么做,毕竟没人特别期待《列那狐传奇》\footnote{译注:《列那狐传奇》的故事采用把动物人格化的方法,用动物世界来影射人类社会,反映市民阶层形成后的封建社会的世态人情,阶级矛盾和斗争。}必须与彼此一致,或者抱怨“嘿,当他诱惑列那时,宙斯是天鹅,为什么现在他变成了金色的雨?”毕竟,如果列那的故事可以随每次讲述而改变,那么当你听到一个完全不同的故事时,为什么还要期待它仍是一样的呢?

口述故事是活的、成长的、变化的东西。相比之下,书写或拍摄的故事钉在纸上或屏幕上是死的,无法改变或成长,永久地固定在一次讲述中。观众可以改变和成长,所以他们对故事的视角可以随时间改变,但故事的实际创作者却被剥夺了这种成长的机会。即使在制作续集时,观众,尤其是“婆罗门”观众,会要求连贯性;也就是说,他们要求忠于死去故事的专制统治。令人惊讶的是,很多创作者不选择反抗!

于是《叛逆》对动画剧集表示口头上的尊重。剧集中所有的事件在这里都确实地发生了,并得到了那些痴迷于连贯性的人所谓的“尊重”,也就是严格遵守“不应与之前的设定矛盾”的字面法则。甚至电影结构也模仿了动画的结构:分成三部分,第一部分假装是“正常”的魔法少女剧,直到与麻美的暴力冲突陡然打破这种假象。第二部分(诚然,它与第一部分的重叠比动画更强)跟随一名魔法少女的视角,逐渐意识到她其实正与自己对抗,并且从一开始就是魔女。最后,第三部分涉及与大量魔女的最终决战,之后现实被重写,新的秩序建立。

然而,动画是跟随小圆的,而电影则是跟随焰的,这就是所有区别所在。小圆是一个耐心、谨慎但非常乐观的角色。她直到剧集的最后才采取行动,但当她这么做时,是果断的,意图彻底解决她认为是宇宙中最主要的问题。而焰则是一个愤世嫉俗、冲动且对抗性强的角色;她一次又一次地投身于冲突,直到最终她模仿小圆的行动,创造了一个世界,在这个世界里,她必须不断地应对小圆的自我牺牲倾向。

这并不意味着,在电影的大部分时间里,焰都是有意识地在反抗小圆的秩序。焰最初的位置,与在动画中一样,是质疑和扰乱现状的那个人,没错,但那个现状(如前所述,由麻美代表并捍卫)是焰自己的梦境。焰试图将世界恢复到她记忆中的样子,也就是动画中的世界。她只有在小圆告诉她,离开她所爱之人是非常痛苦的之后,才开始有意识地反抗。换句话说,焰意识到小圆的自我牺牲的的确确带来了牺牲。当然,对焰来说小圆的牺牲是不可想象和不可原谅的,即使是小圆自己在做出牺牲。

即便如此,焰直到电影后期才开始采取行动,因为在那之前她都没有机会。真正大部分时间反抗小圆,进而反抗《魔圆》的角色是丘比,它策划了整个局面,试图篡夺循环法则并带回魔女。值得记住的是,丘比在许多方面是一个(异常不讨人喜欢的)作者化身,因此,它对小圆的反抗可以被看作是创作者对《魔圆》的反抗。

然而,丘比的反抗并不令人惊讶。毕竟,它是剧中的反派,而一个不知悔改的反派在续集中依然存在,至少可以假定它会尝试重新扮演反派角色。相比之下,焰对小圆表现出了极度、执着的忠诚,因此电影小心翼翼地精心布局了她的反叛元素:她有动机,来源于她与小圆在花丛中的对话,以及她意识到自己“永远不该允许”小圆牺牲自己。她有主意,当丘比揭示小圆可以重新进入世界时,焰意识到了小圆作为循环法则记忆和力量的佛性是可以由他人储存的。而她也有机会,当小圆降临到“现实世界”来拯救焰,防止她成为魔女时,正如丘比所说,“可感知的事物就可以被干涉”。

于是焰作为魔鬼的形象崛起,将“神”从她的天堂中拉下,并带入尘世。她是终极的“坏女孩”,她的使魔们引用《失乐园》\footnote{译注:《失乐园》是英国政治家、学者约翰·弥尔顿创作的史诗,讲述了叛逆之神撒旦,因为反抗上帝的权威被打入地狱,却毫不屈服,为复仇寻至伊甸园。亚当与夏娃受被撒旦附身的蛇的引诱,偷吃了上帝明令禁吃的知识树上的果子。最终,撒旦及其同伙遭谴全变成了蛇,亚当与夏娃被逐出了伊甸园。}中的诗句,伴随着尼采式的咏唱,向她投掷番茄,将她标榜为撒旦。她接受了“恶魔”、“邪恶”和“敌人”的标签,并明确表示打算将其付诸行动。这又带我们进入了另一个叛逆……