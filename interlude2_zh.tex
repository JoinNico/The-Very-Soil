\interlude{牛奶之尸}
《魔圆》中最为突出的主题之一便是衰败。显然,熵是衰败的一种形式,而魔法少女/魔女则被塑造成对抗衰败的武器。然而,作品中还展现了其他形式的衰败:城市在剧集中逐渐崩坏,从第一集明亮整洁的空间,到第十二集化为废墟的景象。尤为突出的是,魔法少女们的精神状态也在不断衰败。这一点在沙耶香的堕落中表现得最为明显,其他魔法少女也显露出严重抑郁的迹象,例如在麻美和小圆交心时出现的疑似抗抑郁药“百忧解”的喷泉,以及杏子不断用食物麻痹情感的行为。魔女系统的核心目的,正是让魔法少女在情感上逐渐衰败,直至她们化为魔女;从某种意义上说,丘比的系统所做的,不过是将宇宙的物理衰败转化为少女们的情感衰败。

这种无处不在的衰败与作品的佛教根源紧密相连。佛教四圣谛中的首谛便是“苦谛\footnote{译注:根据对现实的深刻观察,佛总结出人生的八大痛苦:生、老、病、死、爱别离、怨憎会、求不得、五蕴炽盛。 世间有情悉皆是苦,有漏皆苦,即所谓“苦谛”。}”,简单翻译为“苦难”。苦分为三种:一是显而易见的苦,如疾病、衰老和死亡;二是因执着于随时间流逝而不断变化的事物而产生的焦虑之苦;三是所有物质事物因其短暂性而固有的根本之苦(108)。

最后一种苦与熵直接对应,即所有物质事物必然走向衰败的原则(109)。这种衰败的必然性看似应成为绝望的源泉,但佛教提供了解决之道。其主要方法是超脱:逃离这个世界,便是逃离不可避免的绝望轮回(110)。这正是小圆在剧终时,以观世音菩萨的身份为魔法少女们打开的门。然而,这是唯一的解决之道吗?在这个无常世界中,是否能获得幸福?

西方文化最初也给出了否定的答案。基督教提供的解决之道是逃离此世,进入天堂,并附加了一个观念:在未来的某一时刻,上帝将毁灭这个充满苦难的世界,并以一个更好的世界取而代之。然而,在中世纪和文艺复兴时期,出现了新概念,为衰败与绝望提供了另一条出路:腐化。

腐化是一个炼金术概念,本质上是发酵的另一种说法,但它逐渐指代死亡与腐烂如何带来新生(111)。试想一块腐烂的水果,它对人类的感官来说是令人厌恶的,漆黑、丑陋且散发着恶臭,但与此同时霉菌和蛆虫爆发,也是新生命的狂欢。这些新生命又为“更高级”的生命形式提供养分,直至最终,连最高贵的生物也依赖腐烂而存在。

这不仅仅是生物学的生命周期,更是炼金术士最深奥的精神教义之一:死亡孕育生命。腐烂与创造本是一体。衰败即是进化。

换句话说,花朵在墓园中绽放。其中一种是彼岸花,它在《叛逆》中作为花冠点缀着焰的魔女形态。它们通常呈鲜红色,与大多数花不同,它在开花前会先落叶;它象征着因命运和死亡而分离的爱人,常被种植于日本的墓园中(112)。这与焰的痛苦:因命运而与挚爱的小圆分离,之间的联系显而易见。

然而,种植这些花的行为表明人们仍然铭记着逝去的爱人;爱可以在物质存在衰败后依然延续。正是这种起源于一个被毁灭宇宙的爱将小圆带回了焰的虚幻世界。也带回了另外两个存在:夏洛特和沙耶香。她们因对小圆的忠诚与责任而回归,但后来她们也表明了额外的意图。

不出所料,鉴于夏洛特在《叛逆》中一直痴迷于奶酪,她回归的动机正是为了奶酪。奶酪是腐化的绝佳象征,既是一种美味且营养丰富的食物,本质上又是腐烂的牛奶。夏洛特并不是唯一带着这种动机返回的人;小圆和她的仆从们也都是为某种从衰败中诞生的珍贵之物而回归的。对小圆而言,是她与焰的关系,这种关系在多个时间线中逐渐演变:小圆从一个阳光、开朗的魔法少女,变为动画中那个完全被动的人物,而焰则从怯懦中逐渐封闭自我,高冷起来。另一方面,沙耶香回归是为了她与杏子的关系,这段关系根植于杏子在沙耶香精神状态迅速恶化时试图拯救她的努力。

换言之,腐化的产物是具有价值的。衰败并非纯粹的邪恶。

那么,如何理解小圆那完美无缺、无衰败的涅槃世界中必然没有奶酪这一事实?这既是因为无法制作奶酪,也因为夏洛特没有理由离开。在一个没有衰败、没有腐化的世界中,由腐烂创造的美与生命都无法存在。小圆的世界中既没有奶酪,也没有历经磨难的友谊,因此它不能被视为解决衰败问题的充分方案。

只有时间,以及《叛逆》几乎不可避免的续作,才能告诉我们焰的解决方案是否更为高明。